Reconfigurable architectures have gained popularity in recent years as they allow the design of energy-efficient accelerators. Fine-grain fabrics (e.g. FPGAs) have traditionally suffered from performance and power inefficiencies due to bit-level reconfigurable abstractions. Both fine-grain and coarse-grain architectures (e.g. CGRAs) traditionally require low level programming and suffer from long compilation times. We address both challenges with {\it Plasticine}, a new spatially reconfigurable architecture designed to efficiently execute applications composed of parallel patterns. Parallel patterns have emerged from recent research on parallel programming as powerful, high-level abstractions that can elegantly capture data locality, memory access patterns, and parallelism across a wide range of dense and sparse applications.

We motivate Plasticine by first observing key application characteristics captured by parallel
patterns that are amenable to hardware acceleration, such as hierarchical parallelism, data
locality, memory access patterns, and control flow. Based on these observations, we architect
Plasticine as a collection of \emph{Pattern Compute Units} and \emph{Pattern Memory Units}. Pattern Compute
Units are multi-stage pipelines of reconfigurable SIMD functional units that can efficiently execute
nested patterns. Data locality is exploited in Pattern Memory Units using banked scratchpad memories and
configurable address decoders. Multiple on-chip address generators and scatter-gather engines make efficient use of DRAM
bandwidth by supporting a large number of outstanding memory requests, memory coalescing, and burst
mode for dense accesses.  Plasticine has an area footprint of 113 $mm^2$ in a 28nm process, and
consumes a maximum power of 49 W at a 1~GHz clock. Using a cycle-accurate simulator, we demonstrate that
Plasticine provides an improvement of up to 76.9$\times$ in performance-per-Watt over a conventional FPGA over a wide range of dense and sparse applications.
