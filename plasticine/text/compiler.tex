\section{SPATIAL Compiler}
\label{compiler}
The SPATIAL compiler transforms SPATIAL programs to PISA in two phases: high-level
compilation and low-level compilation. The high-level compiler represents applications
internally using a high-level dataflow IR. After a series of analyses, optimizations, and scheduling, the
dataflow IR is transformed to a low-level IR called \emph{Plasticine IR},
or \emph{PIR}. The low-level compiler accepts as input an architecture specification, and
transforms \emph{PIR} to \emph{PISA} configuration using an iterative mapping algorithm.
The following sections describe our intermediate
representations, optimizations, scheduling, placement, and routing.

\subsection{Example}
Insert picture to show program transformations on example in Section~\ref{spatial} during various
compiler stages.

\subsection{High-level Dataflow IR}
We have a high-level dataflow IR.

\subsection{Plasticine IR}
We have a low-level IR called \emph{Plasticine IR}, or \emph{PIR}.

\subsection{High-level Compilation}
\subsubsection{Optimizations}
\subsubsection{Scheduling}

\subsection{Low-level Compilation}
\subsubsection{Register Allocation}
\subsubsection{Iterative Place-And-Route}
