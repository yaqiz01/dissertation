\section{Algorithmic Improvements}
\label{sec:algo}
To fix these deadlock problems, we use a conflict set of edges that result in possible deadlocks and reduce the problem to graph coloring.
When we color the conflict graph, we use the following correspondences:
\begin{itemize}
  \item Routes in the program graph become nodes in the conflict graph.
  \item Edges are added to the conflict graph to break cycles.
  \item Nodes in the conflict graph are colored by assigning VCs.
\end{itemize}

The deadlock avoidance algorithm proceeds as follows:
\begin{enumerate}
  \item For all nodes, find the set of routes arriving at the node.
  \item Add a fully connected set of edges to ensure that those routes are placed in different VCs.
  \item Use graph coloring to allocate VCs.
  \item Produce the full holds/waits graph, including dependences through nodes and static links.
  \item Reduce the holds/waits graph to the cyclic subgraph.
  \item If there are no cycles, terminate.
  \item If all cycles are ``program cycles,'' terminate. A program cycle is a set of dependences between nodes that is present in the original program, which is assumed to be deadlock-free. 
        If there are any cycles in the network graph that don't include nodes, there are non-program cycles.
        Otherwise, the check is done by building the set of direct node dependences using BFS, which are dependences that can be constructed between nodes without traversing other nodes.
        Then, this list is checked against the input program. 
  \item Otherwise, find a possible conflicting pair of holds/waits dependences. This is a pair of routes, where one route waits on a resource that another holds. This is guaranteed to be possible, because if all routes were assigned to different VCs, there would be no non-program cycles. Add this pair as an edge to the conflict graph, and go to step 3.
\end{enumerate}
