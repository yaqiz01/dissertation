\section{Future Work}
\label{sec:future}
\subsection{Multitenancy}
%\begin{outline}
%  \1 Do an evaluation of app combinations, and plot the speedup of the combined apps vs. sequential execution
%\end{outline}
One promising future direction for this work is enabling fine-grained resource sharing for multitenant CGRAs.
Because static networks are slow to place, and placement is not guaranteed to succeed, the state of the art presents no compelling options for placing multiple applications onto a single chip.
One possibility, which is potentially highly inefficient, is to partition the chip and run smaller versions of each application on the partitions. 
However, this does not account for the significant disparity in resource usage between applications: two applications, one DRAM-bound and one compute-bound, could each only run at half speed with this approach.
Another possibility is to place both applications together onto the same chip, arbitrarily sharing the same resources. 
However, as congestion increases, there is no guarantee that this will complete for any given chip---if it does not complete, the system can not run the applications in parallel.

\subsection{Fault Tolerance}
%\begin{outline}
%  \1 Do an evaluation of placement with several randomly disabled PCUs for several apps, and show that performance is maintained
%\end{outline}
Another promising direction is producing more fault-tolerant CGRAs.
Many current fault tolerance schemes attempt to replace ``like for like,'' such as swapping in a spare SRAM row to replace a defective one \cite{}.
However, to reap the benefits of SIMD parallelism, the largest structures in Plasticine are on the order of \SI{1}{mm^2}; it is not possible to swap in another row if one PCU or PMU is defective.
Therefore, applications for future CGRAs will need to adapt to defects in arbitrary locations on chip whose exact locations become known only at runtime.
This necessitates a fast placement algorithm that is guaranteed to succeed---impossible with a purely static network.
