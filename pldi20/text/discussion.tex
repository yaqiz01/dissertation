\section{Discussion \& Future Work}
\label{sec:discussion}

Currently our compiler map the entire program graph onto the accelerator spatially.
If the program graph cannot fit on the hardware, the programmer needs to break the applications
into multiple kernels, materialize intermediate results into DRAM, and reconfiguring the accelerator.
The partitioning algorithm described in \Cref{sec:splitmerge} can be used to find region of the program 
to partition within minimum live out variables. Doing this automatically can also enable overlapping
of reconfiguration with materializing and prefetching of the data.
The same algorithm can be used to partition the program on multiple chips. Since the control-flow
is completely distributed, the program tolerant arbitrary amount of delay from PCI links between
accelerators.
There is also the opportunity of mapping multiple logical memories with the same on-chip scratchpad. 
This can be achieved in pure software without hardware support. Compiler can perform live range analysis
on the memories and map logical memories with no overlapping live range on the same scratchpad. 
The last actor that access the old memory can pass the token to the first actor that accesses the new memory
to override previous data.
Fine-grain reconfiguration of the compute would requires hardware supports to load multiple configurations in PB.
In order to time-multiplex the computation hardware, such as using the same actor for different
layers of a DNN, currently the user need to write a loop around the reusable section and change dimension
of the kernel based on loop iterator. 
Finally, the distributed control-flow can be also applied on FPGAs to improve clock frequency on large designs.
