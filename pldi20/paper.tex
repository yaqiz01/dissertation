%%
%% This is file `paper.tex',
%% generated with the docstrip utility.
%%
%% The original source files were:
%%
%% samples.dtx  (with options: `sigconf')
%% 
%% IMPORTANT NOTICE:
%% 
%% For the copyright see the source file.
%% 
%% Any modified versions of this file must be renamed
%% with new filenames distinct from sample-sigconf.tex.
%% 
%% For distribution of the original source see the terms
%% for copying and modification in the file samples.dtx.
%% 
%% This generated file may be distributed as long as the
%% original source files, as listed above, are part of the
%% same distribution. (The sources need not necessarily be
%% in the same archive or directory.)
%%
%% The first command in your LaTeX source must be the \documentclass command.
\documentclass[sigplan,10pt,preprint]{acmart}

%% Includes for the paper
%%
%% Font and spacing
\usepackage{xspace} 
\usepackage{enumitem} % left margin for itemize
\usepackage{color, colortbl}
\usepackage{xcolor}
\usepackage{textgreek}

%%
%% Reference 
\usepackage[capitalise]{cleveref}

%% 
%% Math 
\usepackage{siunitx}
\usepackage{amsmath}

%% 
%% Listings
\usepackage{listings}
\usepackage[utf8]{inputenc}

%% 
%% Tables
\usepackage{booktabs}  
\usepackage{array}
\usepackage{multirow}
\usepackage{diagbox}

%%
%% Format
\usepackage{balance}
\usepackage{xurl}

%%
%% Footnotes
\usepackage[multiple]{footmisc}

%%
%% Outlines
\usepackage{outlines}

%% subfigure
\usepackage{subcaption}

\usepackage{booktabs}
\usepackage{multirow}
%\usepackage[numbers,sort&compress]{natbib}
\usepackage[us,12hr]{datetime}
\usepackage{amsmath}
\DeclareMathOperator*{\argmax}{argmax}
\usepackage{algpseudocode}
\usepackage{outlines}
\usepackage[normalem]{ulem}
\usepackage{siunitx}
\usepackage{subfig}
\usepackage{minted}
\newcommand\info[1]{\textcolor{red}{(comment: #1)}}
%\newcommand{\gist}[1]{{\color{blue} GIST: #1 \\}}
\newcommand\TODO[1]{\textcolor{red}{(TODO: #1)}}
\newcommand{\yaqi}[1]{{\color{blue} #1}}
\newcommand\dash{\textnormal{-}}
\newenvironment{hangingpar}[1]
  {\begin{list}
          {}
          {\setlength{\itemindent}{-#1}%%'
           \setlength{\leftmargin}{#1}%%'
           \setlength{\itemsep}{0pt}%%'
           \setlength{\parsep}{\parskip}%%'
           \setlength{\topsep}{\parskip}%%'
           }
    \setlength{\parindent}{-#1}%%
    \item[]
  }
  {\end{list}}
\usepackage{listings}           % Code listing
\usepackage{courier}
\definecolor{vbgray}{gray}{0.9}
\definecolor{darkgreen}{RGB} {0, 100, 0}
\definecolor{darkred}{RGB} {255, 0, 0}
\definecolor{orange}{RGB} {255, 128, 0}
\def\tick{{\color{darkgreen} \textbf{\tikz\fill[scale=0.5](0,.35) -- (.25,0) -- (1,.7) -- (.25,.15) -- cycle;}}}
\def\ytick{{\color{orange} \textbf{\tikz\fill[scale=0.5](0,.35) -- (.25,0) -- (1,.7) -- (.25,.15) -- cycle;}}}
\def\x{{\color{darkred} {$\bm{\times}$}}}

\definecolor{red}{RGB}{255,0,0}
\definecolor{vbgray}{gray}{0.9}
\definecolor{darkgreen}{RGB} {0, 100, 0}
\definecolor{darkred}{RGB} {255, 0, 0}
\definecolor{blue}{RGB} {0, 135, 255}
\definecolor{yellow}{RGB} {224, 173, 0}
\definecolor{codegreen}{RGB}{52,123,0}
\definecolor{codegray}{rgb}{0.6,0.5,0.5}
\definecolor{codepurple}{rgb}{0.58,0,0.82}
\definecolor{backcolour}{rgb}{0.95,0.95,0.95}
\definecolor{lightback}{rgb}{0.95,0.95,0.95}
\def\tick{{\color{darkgreen} \textbf{\tikz\fill[scale=0.5](0,.35) -- (.25,0) -- (1,.7) -- (.25,.15) -- cycle;}}}
\def\ytick{{\color{orange} \textbf{\tikz\fill[scale=0.5](0,.35) -- (.25,0) -- (1,.7) -- (.25,.15) -- cycle;}}}
\def\x{{\color{darkred} {$\bm{\times}$}}}

\lstdefinelanguage{Spatial}{
  basicstyle=\fontsize{7}{7}\selectfont\ttfamily,frame=tlbr,framesep=4pt,framerule=0pt,
  tabsize=2,
  basewidth={0.55em, 0.4em},%
  numbers=left,
  showspaces=false,
  keywordstyle=\bfseries,
  breaklines=true,
  columns=fixed,
  %xleftmargin=0.25in,
  firstnumber=auto,
  showstringspaces=false,
  escapechar=@,
  escapeinside={(*@}{@*)},
  morestring=[b]",
  morestring=[b]',
  morecomment=[l]{//},
  morecomment=[s]{/*}{*/},
  backgroundcolor=\color{backcolour},
  commentstyle=\color{codegreen}, %\bfseries,
  numberstyle=\tiny\color{codegray},
  stringstyle=\color{codepurple},
  keywordstyle=[2]\color{blue},
  keywords=[2]{val, def, type},
  keywordstyle=[3]\color{yellow}\bfseries,
  keywords=[3]{Float, Float8, Int, String, T, Void, Bit, Half, FltPt},
  keywordstyle=[4]\color{orange}\bfseries,
  keywords=[4]{Matrix, Array},
  keywordstyle=[5]\color{blue}\bfseries,
  keywords=[5]{StreamIn, StreamOut, DRAM, ArgIn, ArgOut, HostIO, RegFile, Reg, SRAM, SRAM1, SRAM2, FIFO, LIFO, LUT, LineBuffer},
  keywordstyle=[6]\bfseries,
  keywords=[6]{enq, deq, load, store, scatter, gather, :=, push, pop, peek},
  keywordstyle=[7]\color{magenta},
  keywords=[7]{until, par, by, value},
  keywordstyle=[8]\color{red}\bfseries,
  keywords=[8]{C0,C1,C2,C3,C4,C5,C6,C7,C8,C9,C10},
  keywordstyle=\color{magenta}\bfseries,
  morekeywords={Foreach,Reduce,MemReduce,MemFold,Fold,Accel,Stream,FSM,Sequential,if,else,Parallel,Pipe, DummyPipe}
}



%%
%% Setup annotations and comments
\if\showcomments1
    \usepackage[textsize=tiny]{todonotes}
\else
    \usepackage[disable]{todonotes}
\fi

%%
%% end of the preamble, start of the body of the document source.
\begin{document}

%%
%% The "title" command has an optional parameter,
%% allowing the author to define a "short title" to be used in page headers.
\title{\name{}: A Compiler for Scaling Reconfigurable Dataflow Accelerators\vspace{-10pt}}

%%
%% The "author" command and its associated commands are used to define
%% the authors and their affiliations.
%% Of note is the shared affiliation of the first two authors, and the
%% "authornote" and "authornotemark" commands
%% used to denote shared contribution to the research.
%\author{Ben Trovato}
%\authornote{Both authors contributed equally to this research.}
%\email{trovato@corporation.com}
%\orcid{1234-5678-9012}
%\author{G.K.M. Tobin}
%\authornotemark[1]
%\email{webmaster@marysville-ohio.com}
%\affiliation{%
%  \institution{Institute for Clarity in Documentation}
%  \streetaddress{P.O. Box 1212}
%  \city{Dublin}
%  \state{Ohio}
%  \postcode{43017-6221}
%}
%
%\author{Lars Th{\o}rv{\"a}ld}
%\affiliation{%
%  \institution{The Th{\o}rv{\"a}ld Group}
%  \streetaddress{1 Th{\o}rv{\"a}ld Circle}
%  \city{Hekla}
%  \country{Iceland}}
%\email{larst@affiliation.org}
%
%\author{Valerie B\'eranger}
%\affiliation{%
%  \institution{Inria Paris-Rocquencourt}
%  \city{Rocquencourt}
%  \country{France}
%}
%
%\author{Aparna Patel}
%\affiliation{%
% \institution{Rajiv Gandhi University}
% \streetaddress{Rono-Hills}
% \city{Doimukh}
% \state{Arunachal Pradesh}
% \country{India}}
%
%\author{Huifen Chan}
%\affiliation{%
%  \institution{Tsinghua University}
%  \streetaddress{30 Shuangqing Rd}
%  \city{Haidian Qu}
%  \state{Beijing Shi}
%  \country{China}}
%
%\author{Charles Palmer}
%\affiliation{%
%  \institution{Palmer Research Laboratories}
%  \streetaddress{8600 Datapoint Drive}
%  \city{San Antonio}
%  \state{Texas}
%  \postcode{78229}}
%\email{cpalmer@prl.com}
%
%\author{John Smith}
%\affiliation{\institution{The Th{\o}rv{\"a}ld Group}}
%\email{jsmith@affiliation.org}
%
%\author{Julius P. Kumquat}
%\affiliation{\institution{The Kumquat Consortium}}
%\email{jpkumquat@consortium.net}

\author{Paper \#169}
\affiliation{\institution{\pageref{lastpage} Pages Body, \pageref{totalpage} Pages Total}}
%\email{jsmith@affiliation.org}

%%
%% By default, the full list of authors will be used in the page
%% headers. Often, this list is too long, and will overlap
%% other information printed in the page headers. This command allows
%% the author to define a more concise list
%% of authors' names for this purpose.
%\renewcommand{\shortauthors}{Trovato and Tobin, et al.}

%%
%% Setup todo notes for comments and annotations
\if\showcomments1
    \setcounter{page}{0}
    \listoftodos {}
    \clearpage
\fi

%%
%% The abstract is a short summary of the work to be presented in the
%% article.
\prefacesection{Abstract}

%In this talk, I will talk about architectural design and compilation techinques that improves the scaling efficiency of a RDA developed 
%at Stanford--Plasticine.
%Staring with a static-dynamic hybrid network, we show that the hybrid network can improves energy efficiency while providing
%guaranteed success in placement and routing.
%Next, I will talk about compiler techinques that convert applications' complex control hierarchies, such as nested loops and branch conditions, into a streaming dataflow representation that can be efficiently executed by Plasticine with distributed on-chip resources.
%The compiler implements (a) a peer-to-peer (p2p) control paradigm inferred from an imperative programming style that minimizes synchronization overhead, and (b) a mapping strategy that decomposes the computation and memory in a program across a distributed heterogeneous resources.
%By applying these techniques, we show that Plasticine is able to outperform state of art accelerators, such as GPUs and FPGAs, in
%both performance and performance/Watt in various dense, sparse, and streaming applications.

With the slowdown of Moore’s Law, specialized hardware accelerators are gaining tractions 
for delivering 100-1000x performance improvement over general-purpose processors. 
As the performance scaling in multicores is coming to a limit~\cite{multicorescale}, a new
class of accelerators--reconfigurable dataflow architectures (RDAs)--is promising in 
offering high-throughput and energy-efficient acceleration that keeps up with the performance demand.
Instead of dynamically fetching instructions like in traditional processors, RDAs have flexible datapath 
that can be statically configured to spatially parallelize and pipeline the program across
distributed on-chip resources. 
The pipelined execution model and explicitly-managed scratchpad in RDAs eliminate
the performance, area, and energy overhead in dynamic scheduling and conventional memory hierarchy.

To adapt to the compute intensity in modern data-analytic workloads, particularly in the deep learning
domain, RDAs are increasing to a scale that was unprecedented before, compared to the classic coarse-grained
reconfigurable architectures (CGRAs).
With an area footprint of $133\text{mm}^2$ at 28nm, 
Plasticine is a hierarchical RDA with 12.3 TFLOPs of compute power~\cite{plasticine}.
Prior work has shown an up to 76x performance/watt benefit from Plasticine over a Stradix V FPGA 
due to advantage in clock frequency and resource density.
The increase in scale introduces new challenges in on-chip network design to maintain 
the throughput and energy efficiency of RDAs.
Furthermore, targeting and managing RDAs at this scale also require new strategies in mapping, 
memory management, and flexible control to fully utilize the compute power of the accelerator. 

In this work, we focus on two aspects of the software-hardware co-design that impact the usability
and scalability of the Plasticine accelerator. 
One of the biggest challenges that hinders the adoption of these
accelerators is the low-level declarative configuration interface that requires the programmers to
have detailed knowledge about the underlying microarchitecture implementation and hardware
constraints. To address the programmability challenge, we introduce a compiler stack that provides a high-level
programming interface that efficiently translates imperative control constructs to streaming
dataflow execution with minimum synchronization overhead on an on-chip distributed architecture. The
compiler handles the hardware constraints systematically with resource virtualization. To address
the performance challenges, we present a comprehensive study on the on-chip network design for 
reconfigurable dataflow architectures that sustain performance in a scalable fashion with high energy efficiency.
 % 1/4 page

%%%
%%% The code below is generated by the tool at http://dl.acm.org/ccs.cfm.
%%% Please copy and paste the code instead of the example below.
%%%
%\begin{CCSXML}
%<ccs2012>
% <concept>
%  <concept_id>10010520.10010553.10010562</concept_id>
%  <concept_desc>Computer systems organization~Embedded systems</concept_desc>
%  <concept_significance>500</concept_significance>
% </concept>
% <concept>
%  <concept_id>10010520.10010575.10010755</concept_id>
%  <concept_desc>Computer systems organization~Redundancy</concept_desc>
%  <concept_significance>300</concept_significance>
% </concept>
% <concept>
%  <concept_id>10010520.10010553.10010554</concept_id>
%  <concept_desc>Computer systems organization~Robotics</concept_desc>
%  <concept_significance>100</concept_significance>
% </concept>
% <concept>
%  <concept_id>10003033.10003083.10003095</concept_id>
%  <concept_desc>Networks~Network reliability</concept_desc>
%  <concept_significance>100</concept_significance>
% </concept>
%</ccs2012>
%\end{CCSXML}
%
%\ccsdesc[500]{Computer systems organization~Embedded systems}
%\ccsdesc[300]{Computer systems organization~Redundancy}
%\ccsdesc{Computer systems organization~Robotics}
%\ccsdesc[100]{Networks~Network reliability}

%%
%% Keywords. The author(s) should pick words that accurately describe
%% the work being presented. Separate the keywords with commas.
%\keywords{datasets, neural networks, gaze detection, text tagging}
%\keywords{Streaming Data-flow Accelerator, Reconfigurable Spatial Accelerator, Spatial Accelerator Mapping}

%%% A "teaser" image appears between the author and affiliation
%%% information and the body of the document, and typically spans the
%%% page.
%\begin{teaserfigure}
%  \includegraphics[width=\textwidth]{sampleteaser}
%  \caption{Seattle Mariners at Spring Training, 2010.}
%  \Description{Enjoying the baseball game from the third-base
%  seats. Ichiro Suzuki preparing to bat.}
%  \label{fig:teaser}
%\end{teaserfigure}

%%
%% This command processes the author and affiliation and title
%% information and builds the first part of the formatted document.
\maketitle

%% Removes running page headers
\if\removepageheaders1
    \pagestyle{plain}
\fi

\section{Introduction}
\label{sec:introduction}

With the end of Dennard Scaling~\cite{dennard}, the amount of performance one can extract from a CPU is reaching a limit.
To provide general-purpose flexibility, CPU spends the majority of energy on overheads, including dynamic-instruction execution, branch prediction, and a cache hierarchy, and less than 20\% of the energy on the actual computation~\cite{mark}.
Even worse, the power wall is limiting the entire multicore family
to reach the doubled performance improvement per generation enabled by technology scaling in the past\cite{multicorescale}.

For this reason, many recent efforts are spent on leveraging application domain knowledge in hardware design to enable 
continued performance scaling while meeting the power budget\cite{turinglecture}.
Examples include widely adopted General-Purpose Graphics Processing Units (GPGPUs) in the deep-learning domain and machine learning (ML) accelerators, such as Tensor Processing Units~\cite{tpu} and EIE~\cite{eie}, providing orders of magnitude acceleration over a CPU.
However, massive threads in GPU and the highly specialized datapath in ML accelerators often cause severe
underutilization of the hardware due to variation in application and data characteristics\cite{tz_rnn}.

\begin{figure}[!tbp]
  \centering
  \includegraphics[height=0.1\paperheight]{figures/parpipe.pdf}
  \caption{Scaling regimes for processor (CPU and GPGPU) and spatial architectures (FPGA and RDA).}
  \label{fig:scaling}
  \includegraphics[width=1\columnwidth, height=0.1\paperheight, keepaspectratio]{figures/rda_arch.pdf}
  \caption{RDA High-Level Model}\label{fig:arch}
\end{figure}

Reconfigurable spatial architectures overcome this limitation by changing its datapath based on applications' needs.
Applications are configured at the circuit-level without dynamic instruction fetching and decoding, hence improving energy-efficiency.
In addition to instruction, data, and task-level parallelism explored by processor architectures, spatial architectures also explore instruction and task-level
pipelining that further increase the compute throughput.
Pipelining at various granularity enables spatial architectures to achieve high throughput on program phases without massive parallel workload, and allocate resource proportional to compute intensities of program phases.
As the whole program is pipelined, performance is bounded by the bottleneck stage with the highest latency.
Initially, the overall throughput can be doubled by only parallelizing the bottleneck stage without doubling the resource of the entire program, which gives a super scaling in performance to allocated resources, shown in \Cref{fig:scaling}.
After stage-balancing, programs enter the linear scaling regime with proportional performance to resource increase, similar
to the best achievable scaling on processor architectures.
Eventually, communication overheads cause sub-linearly scaling in all architectures.
Many applications inherently contain imbalanced workloads in their program structures. 
For example, layers of a deep neural network vary highly in operational complexity.
Spatial architectures can take advantage of imbalance workloads in a program to achieve
high marginal performance improvement per resource increase in the super scaling regime.
For applications, such as large image network like ResNet\cite{resnet}, RDAs tend to run out of resource before
balancing the pipeline, which leaves the program entirely in the super scaling regime.

One example of a spatial architecture is Field Programmable Gate Arrays (FPGAs) that support fine-grain, 
bit-level reconfiguration with a soft logic fabric~\cite{fpga-survey}.
Although around for a long time, FPGAs are not broadly used on high-level applications due to their low-level programming interface 
and low resource density with high routing overhead.

Lately, Reconfigurable Dataflow Accelerators (RDAs)~\cite{plasticine, ti} are emerging as a new class of spatial accelerators that retain the desired level of flexibility and energy efficiency without the fine-grained reconfigurability overhead.
RDA provides high resource density and compute throughput with a hierarchical streaming network, operating at a fixed and high clock frequency at large chip sizes.
Recent studies~\cite{plasticine} have demonstrated a promising acceleration of dense, sparse, and streaming applications using RDAs.

Today, these RDAs are controlled using centralized-control schemes to manage and schedule the spatially distributed compute and memory resources.
However, to accommodate applications with ever-increasing sizes (\eg, large deep neural networks), architects are building ever-larger RDAs.
At scale, the centralized schemes incur high synchronization overhead and communication hot spots. 
To achieve complex control schemes, such as a while convergence, a host often has to launch the RDA multiple times and materialize the on-chip states to DRAM, which further degrades performance.
Moreover, as RDAs become larger, their coarse-grained hierarchal structure introduces fragmentation in resource allocation.
Traditional mapping strategies---performing one-to-one allocation of resource tiles to program constructs---results in severe underutilization of resources, and heterogeneity in RDA resources further decreases mapping efficiency.

In this paper, we introduce \name{}---a compiler for scaling performance and enhancing mapping efficiency for RDAs. 
We propose a distributed p2p control paradigm that minimizes synchronization overhead for RDAs at scale.
We show that \name{} can infer synchronization required for on-chip distributed execution from an application, written in a sequential, unpipelined programming abstraction.
We present a mapping strategy efficiently decomposing a program over the distributed collection of compute and memory resources with global optimizations, which significantly reduces resource fragmentation for RDAs with heterogeneous resources (\Cref{sec:background}).
The distributed control improves RDA's performance by extending the linear region in the performance-resource curve in \Cref{fig:scaling}.
Whereas, the mapping strategy and optimizations reduce the fragmentation and, hence, pushes the scaling curve left.

Using a recently published RDA--Plasticine~\cite{plasticine}--as the target architecture, our evaluation shows that \name improves Plasticine's scaling efficiency (\Cref{ssec:scalability})\yz{quantify this}.
We qualitatively evaluate the impacts of individual optimizations on performance and resource (\Cref{ssec:evalopts}), and compare to a state-of-art Tesla V100 GPU (\Cref{ssec:abs-performance}).
We begin with a brief background on RDA architecture and the input of \name{} (\Cref{sec:background}), before detailing the design of our compiler (\Cref{sec:compiler}). 
 % 1.5 pages
\chapter{Background (WIP)}

\section{Execution Schedule of Reconfigurable Architectures} 
\begin{figure*}
\begin{subfigure}[b]{0.34\textwidth}
\inputminted{python}{code/spatialeg2.py}
\caption {
}
\end{subfigure}
\hfill
\begin{subfigure}[b]{0.65\textwidth}
\centering
\includegraphics[width=1.0\textwidth]{figs/pipeexec.pdf}
\caption {
}
\end{subfigure}
\caption[Hiearchical pipelining and parallelization on spatial architecture]{
Hierarchical pipelining and parallelization in spatial architecture.
(a) illustrates the runtime and throughput of a hierarchically pipelined and parallelized program on
a reconfigurable spatial architecture. 
At inner level, instructions within each basic
block are fine-grained pipelined across iterations of the inner most loop. 
At outer level, the inner loops are coarse-grained pipelined across the outer loop iterations.
Exploiting multiple levels of pipeline parallelism gives a total throughput of $x+y$ operations per
  cycle, where \emph{x} and \emph{y} are number of operations in the basic blocks.
(b) Vectorizing the inner most loops B and C by \texttt{n} increases the throughput to $(x+y)n$.
(c) Parallelizing the outer loop A by \texttt{m} further increases the throughput to $(x+y)mn$.
}
\label{fig:pipeexec}
\end{figure*}

\begin{figure*}
\centering
\includegraphics[width=0.4\textwidth]{figs/peakutil.pdf}
\caption[Average utilization vs. peak compute density tradeoff]{
 Average utilization vs. peak compute density tradeoff among different architectures.
}
\label{fig:peakutil}
\end{figure*}

\begin{figure*}
\centering
\includegraphics[width=1\textwidth]{figs/perfmodel.pdf}
\caption[High-level performance model of spatial architectures]{
High-level performance model of spatial architectures
}
\label{fig:perfmodel}
\end{figure*}

The key advantage of reconfigurable spatial accelerators, compared to processor-based architectures, 
is the ability to explore multiple levels of pipeline parallelism. 
In traditional Von Neumann architectures~\cite{vonneumann}, like CPUs and GPUs,
a computer consists of a processing unit that performs
computation, a memory unit that stores the program states, and a control unit that tracks execution
states and fetch the instruction to execute. This computing model inherently assumes that
instructions with in a program are executed in-time, maximizing the flexibility to 
context switching between different workloads dynamically.

Reconfigurable accelerators are a direct violation of the von Neumann execution model; 
instructions are statically imbedded in the datapath and executed in-space as supposed to in-time.
One of the disadvantage of reconfigurable hardware is paying the resource cost for infrequently
executed instructions, making it unsuitable for control-heavy workloads that traditional
processors are efficient at.
On the other hand, RDAs are particularly competitive in providing high-throughput, 
low-latency, and energy-efficiency acceleration for these applications.
Data-analytical workloads encompass a wide domains of applications, including image processing,
recognition, machine translation, digital signal processing, network processing, etc.
These applications exhibits a rich amount of data-level parallelism with relatively static control
flow.

\Cref{fig:pipeexec} shows an example of exploiting hierarchical parallization and pipelining on
a spatial architecture, where overall throughput equals to the product of total parallelization factors 
and pipelining depth.
By exploring multiple dimensions of concurrency in the program, spatial architecture is more likely
to achieve a good compute throughput for a wide range of applications.
For applications that are expensive to parallelize due to irregular access patterns, spatial
architectures can increase concurrency on the pipelining dimension.  For application with
embarrassingly parallel workloads, reconfigurable accelerator can 

Another benefit of pipelined execution is easier to achieve good memory performance.
Data accessed by different stage of the pipelines are stored in discrete scratchpads 
instead of a shared cache; improving the effective on-chip bandwidth and capacity.
Using explicitly managed scratchpad also tends to improve locality and 
eliminate cache performance issues, such thrashing.
Across kernels, pipelined execution reduces the amount of off-chip accesses for intermediate
data.
SIMT architectures, like GPUs, relying on high-bandwidth DRAM technology, such has HBM, to sustain
the compute throughput of massively parallelized threads.
While providing over 10x more bandwidth than traditional DDR technologies, HBM is very limited in
capacity, around 16GB as supposed to on the orders of TB for DDR.
As a result, the limited off-chip capacity often restricts the type of applications that
GPUs can support.

%\begin{table*}
  %\centering
%\begin{tabular}{lccc}
  %\toprule
 %Concurrency Level & Instruction & Data & Task/Kernel  \\ \midrule
 %Parallelsim & CPU,\rda & CPU,GPU,\rda & CPU,\rda  \\
 %Pipelining & \rda & \rda & \rda \\
 %\bottomrule
%\end{tabular}
%\caption[Concurrency level explored by different architectures]{
  %Concurrency level explored by different architectures
%}
%\label{tab:conclevel}
%\end{table*}

\section{Plasticine}

\begin{figure*}
\centering
\includegraphics[width=0.8\textwidth]{figs/plasticine.pdf}
\caption[Plasticine chip-level architecture]{Plasticine chip-level architectural diagram}
\label{fig:plasticine}
\end{figure*}

\section{Spatial}

\begin{figure}
\centering
%\newsavebox{\outerProduct}
%\begin{lrbox}{\outerProduct}
\lstinputlisting[language=Spatial,linewidth=0.6\columnwidth]{code/OuterProduct.scala}
%\end{lrbox}
%\begin{tabular}{m{0.01cm} l} & \usebox{\outerProduct}\\ \end{tabular}
  %\inputminted[fontsize=\footnotesize]{scala}{code/OuterProduct.scala}
  \caption{Example of Outer Product in Spatial.}
\label{fig:spatial_app}
\end{figure}

%To target spatial architectures, we use Spatial, an open source domain specific language for reconfigurable accelerators \cite{spatial_koeplinger}.
We use Spatial~\cite{spatial_koeplinger}, an domain specific language for reconfigurable accelerators, 
as the front-end of Plasticine.
Spatial describes applications with nested loops and an explicit memory hierarchy that captures data movement on-chip and off-chip. 
This exposes design parameters that are essential for achieving high performance on spatial architectures, including blocking size, loop unrolling factors, inner-loop pipelining, and coarse-grained pipelining of arbitrarily nested loops. 
To enable loop-level parallelization and pipelining, Spatial automatically banks and buffers intermediate memories between loops. 
An example of outer product---element-wise multiplication of two vectors resulting in a matrix---in Spatial is shown in Figure~\ref{fig:spatial_app}.
%In this example we assume inputs \emph{vecA}, \emph{vecB} and outputs \emph{matC} do not fit on chip.
%First, \emph{C2} and \emph{C4} load tiles of vectors of size \emph{tsA} and \emph{tsB} to on-chip scratchpads \emph{tileA} and \emph{tileB}. 
%Next, loop \emph{C5} computes the outer products and store it to scratchpad \emph{tileC}. 
%Finally, \emph{C6} stores partial results back to DRAM. 
\if 0
Spatial enables inner loop pipelining in \emph{C5} and coarse-grained pipelining between stages of the outer loop (e.g. \emph{C4}, \emph{C5}, and \emph{C6} are pipelined across iterations of \emph{C3}). 
The parallelization factor of the inner most loop (\emph{ip} for \emph{C2}, \emph{C5}, and \emph{C6}) translates to SIMD pipeline and vector network vectorization factor. 
In \emph{C1} and \emph{C2}, \emph{op1} and \emph{op2} are outer loop parallelization factors that allow the programmer to unroll the outer loops and parallelize compute, which can better saturates DRAM bandwidth or balances compute pipelines. 
When scratchpad producers or consumers are parallelized, the scratchpad must be banked to sustain the required bandwidth. 
Scratchpads only contain one level of banking hierarchy. 
Therefore, when more than one dimension of the scratchpad is banked, the high-dimensional banks are mapped across multiple scratchpads. 
In this example, if both \emph{ii} and \emph{jj} (used in the write address of \emph{tileC}) on line 31 are parallelized, \emph{tileC} will be mapped to multiple scratchpads. 
This mapping strategy makes broadcast communication common between producers, banks, and consumers when outer loops are unrolled.
\fi
For spatial architectures, Design Space Exploration (DSE) of parameters
(e.g., \emph{op1}, \emph{op2}, \emph{ip}, \emph{tsA}, \emph{tsB}) is critical to achieve good resource utilization and performance \cite{dse_koeplinger}.

 % 1.25 pages -- 3 pages till here
\chapter{Compiler} \label{sec:compiler}

In this section, we introduce the compiler framework---\name---that targets Plasticine
architecture from high-level programs described in the Spatial language. 

In the following sections, \Cref{sec:control} describes conversion from an imperative paradigm with
a nested control hierarchy to the distributed streaming dataflow execution.
\Cref{sec:resalloc} details program-partitioning passes that decompose program over distributed resources.
\Cref{sec:opt} enumerates several optimizations in \name, and \Cref{sec:par} discuss about PaR and
heuristic generation.
 % 
\input{text/compiler2} % 5 pages
\section{Evaluation} \label{sec:eval}
In this section, we evaluate the real-time RNN serving tasks on various platforms.
We start with the methodology of our experiments, followed by a discussion of performance and power comparisons across
these platforms.

\subsection{Methodology} \label{sec:methodology}
To evaluate RNN serving, we use the LSTM and GRU tasks from Baidu DeepBench as our benchmarks.
We evaluate the benchmarks across processor-based architectures including CPU and GPU, 
and spatial architectures including FPGA and CGRA.
Table \ref{tab:spec} shows the detailed specifications of the targeting hardware, 
which includes state-of-the-art high performance platforms in each of the commercialized categories.
Table \ref{tab:appconf} summarizes application configurations of each platform.

\paragraph{CPU} We implement the applications in TensorFlow 1.10, and evaluate our implementations on 
Intel Xeon Scalable Processor (Skylake) CPU.
We use the \texttt{LSTMBlockFusedCell} and \texttt{GRUBlockCell} kernels in TensorFlow.
We further enable AVX2 vector instructions for CPU evaluation. Due to lack of low-precision
support in both tool chain and platform, we use single-precision for our implementation.

\paragraph{GPU} We use TensorFlow with cuDNN Library to target NVIDIA Tesla V100 GPU from Google Cloud. 
cuDNN is a GPU-accelerator Library from NVIDIA that is specialized for deep learning.
We use 16-bit precision for our implementation on GPU.
On both CPU and GPU platforms, we run \emph{TensorFlow} profilers and collect the time spent 
only on evaluating the RNN cells.

\paragraph{Plasticine} We implement the applications in Spatial, which targets Plasticine.
Although Spatial has FPGA back-end support, Stratix 10 is not commercially available at the time of the submission of this work.
The current FPGA targets that Spatial support are not comparable to Stratix 10 both in terms of memory and compute capacity. 
Therefore, we only use Spatial to target Plasticine for this evaluation. However, our approach should generally benefit
an implementation on a high performance FPGA like Stratix 10.
We choose Plasticine configuration that matches the peak 8-bit FLOPS and
on-chip scratchpad capacity of a Stratix 10 FPGA. The exact configuration of Plasticine is shown in Table \ref{tab:conf}.
In order to minimize the overhead of low-precision support, Plasticine only supports 8-bit, 16-bit, and 32-bit element-wise 
operations, and mixed precision reduction operation. 
For our evaluation, the element-wise operations are performed in 8-bit precision, 
the first stage of the reduction is performed in 16-bit, 
while the remaining of the reduction and accumulation are performed in 32 bit operations.

To measure the performance, we use a cycle accurate simulator for Plasticine. 
We modified the simulator to model the proposed micro-architectural changes to support low-precision operations.
We use the area and power of individual CUs and network switches from the original Plasticine paper, 
and compute total area of configuration shown in Table \ref{tab:conf}. 
As discussed in Section \ref{sec:arch}, we reduce the number of stages in PCU from 6 stages to 4 stages with fused low-precision
operations and folded reduction tree. 
Low preicision function units can be used to compose full precision units. 
We conservatively estimate the area and power of PCU stays the same with our proposed change and reduced two stages. 
We also increase the PMU to PCU ratio to better match the compute to memory
ratio for RNN inference applications. To match the memory capacity of Stratix 10, we shrink the scratchpad capacity of 
each PMU from 256kB to 84kB.
For power calculations, we generate activity tracing of the CUs from simulation, and then integrate 
with characterized power of individual PCU to compute the total power. The power and area characterizations are based off
synthesis at 28nm technology at 1GHz clock frequency.

\paragraph{Brainwave} Finally, we also compared our results to Microsoft Brainwave framework.
For this evaluation, we compare to Brainwave implemented
on top of Intel Stratix 10 FPGA. Brainwave is synthesized at 250MHz and all operations are performed in
blocked low-precision floating-point format described in section~\ref{sec:blaslstm}.
\begin{table}[t]
\caption{Plasticine configuration.}
\label{tab:conf}
\centering
\scriptsize
\begin{tabular}{L{3cm}rL{2.5cm}r}
\toprule
\# Row                     & 24   & \# Column        & 24  \\
\# PCU                     & 192  & \# PMU           & 384 \\
\# Lanes in PCU            & 16   & \# Stages in PCU & 4   \\
Scrachpad capacity per PMU & 84kB &                  &     \\
\bottomrule
\end{tabular}
\end{table}

%% Stratix 10 2800 M20K MBits 229, MLAB 15 MBits, = 30.5MB
%% Source https://www.intel.com/content/dam/www/programmable/us/en/pdfs/literature/hb/stratix-10/s10-overview.pdf
\begin{table}
\caption{Hardware specifications for target platforms.}
\label{tab:spec}
\scriptsize
\centering
\begin{tabular}{L{2.5cm}M{1.2cm}M{0.8cm}M{0.8cm}M{1cm}}
\toprule
  Specification        & Intel Xeon Skylake (Dual core) & Tesla V100 SXM2 & Stratix 10 280 FPGA & Plasticine\\
\midrule
Max Clock Rate (GHz) & 2.0/2.8*                  & 1.38/1.53*      & 1                   & 1 \\
On-chip memory** (MB) & 55                        & 20              & 30.5                & 31.5\\
Peak 32-bit TFLOPS   & --                      & 15.7            & 10                  & 12.5\\
Peak 8-bit TFLOPS    & --                        & --              & 48                  & 49\\
Technology ($nm$)    & 14                        & 12              & 14                  & 28\\
Die Area ($mm^2$)    & 64.4                      & 815             & 1200                & 494.37 \\
  TDP (W)    & 15                      & 300             & 148                & 160 \\
\bottomrule
\end{tabular}
* Base/Boosted Frequency
** Capacity of L3 cache for CPU, register file for GPU, and on-chip scratchpad for reconfigurable architectures.
  %* Computed with AVX512f instructions. \cite{markidis2018nvidia}
\end{table}

\begin{table}
\caption{Application configurations for target platforms.}
\label{tab:appconf}
\centering
\scriptsize
\begin{tabular}{L{1.8cm}M{1cm}M{1cm}M{1cm}M{1.5cm}}
\toprule
Platform                       & Intel Xeon Skylake & Tesla V100 SXM2 & Stratix 10 280 FPGA & Plasticine\\
\midrule
Software Framework             & TF+AVX2                   & TF+cuDNN        & Brainwave           & Spatial \\
Achieved Clock Frequency (GHz) & 2                         & 1.38            & 0.25                & 1 \\
Precision                      & f32                       & f16             & blocked precision   & mix f8+16+32\\
\bottomrule
\end{tabular}
\end{table}

\begin{table*}
\caption{Performance comparison of DeepBench Inference.}
\label{tab:eval}
\centering
\scriptsize

%%% =================== With Utilization  ========================
%\begin{tabular}{|L{0.6cm}|M{0.6cm}|M{0.6cm}|M{1.1cm}M{0.6cm}M{0.6cm}M{1.2cm}|M{1.1cm}M{0.6cm}M{0.6cm}M{1.2cm}|M{0.6cm}M{1.2cm}|M{1.2cm}|}
%\hline
  %\multicolumn{3}{|c|}{\sc Benchmarks}					&	\multicolumn{4}{c|}{\textsc{Latency} (ms)}							&	\multicolumn{4}{c|}{\sc Effective TFLOPS}							&	\multicolumn{2}{M{2cm}|}{\textsc{8-bit FLOPS Utilization} (\%)}			&	\sc Power (W)	\\ \hline
  %&	\sc H	&	\sc T	&	\sc Xeon Skylake	&	\sc Tesla V100	&	\sc BW &	\sc Plasticine	&	\sc Xeon Skylake	& Tesla V100	&	\sc BW	&	\sc Plasticine	&	\sc BW	&	\sc Plasticine	&	\sc Plasticine	\\ \hline
%%% =================== PASTE HERE ========================
%\multirow{5}{*}{\sc\bf LSTM}	&	256	&	150	&	15.75	&	1.69	&	0.425	&	0.042	&	0.010	&	0.09	&	0.4	&	3.8	&	0.8	&	7.7	&		\\
	%&	512	&	25	&	11.50	&	0.60	&	0.077	&	0.015	&	0.009	&	0.18	&	1.4	&	7.0	&	2.8	&	14.3	&		\\
	%&	1024	&	25	&	107.65	&	0.71	&	0.074	&	0.037	&	0.004	&	0.59	&	5.7	&	11.2	&	11.8	&	22.9	&		\\
	%&	1536	&	50	&	411.00	&	4.38	&	0.145	&	0.160	&	0.005	&	0.43	&	13.0	&	11.8	&	27.1	&	24.1	&		\\
	%&	2048	&	25	&	429.36	&	1.55	&	0.074	&	0.106	&	0.004	&	1.08	&	22.7	&	15.8	&	47.3	&	32.3	&		\\ \hline
%\multirow{6}{*}{\sc\bf GRU}	&	512	&	1	&	1.00	&	1.00	&	0.013	&	1.000	&	0.003	&	0.00	&	0.2	&	0.0	&	0.5	&	0.0	&		\\
	%&	1024	&	1500	&	449.00	&	33.77	&	3.792	&	1.720	&	0.042	&	0.56	&	5.0	&	11.0	&	10.4	&	22.4	&		\\
	%&	1536	&	375	&	2,730.00	&	13.12	&	0.951	&	0.910	&	0.004	&	0.81	&	11.2	&	11.7	&	23.3	&	23.8	&		\\
	%&	2048	&	375	&	5,040.00	&	17.70	&	0.954	&	1.580	&	0.004	&	1.07	&	19.8	&	12.0	&	41.2	&	24.4	&		\\
	%&	2560	&	375	&	7,590.00	&	23.57	&	0.993	&	2.460	&	0.004	&	1.25	&	29.7	&	12.0	&	61.9	&	24.5	&		\\
	%&	2816	&	750	&	25,850.00	&	55.48	&	1.987	&	6.430	&	0.003	&	1.29	&	35.9	&	11.1	&	74.9	&	22.7	&		\\
%%% =================== PASTE HERE ========================

%% =================== With Speedup and no power ========================
  \begin{tabular}{|L{0.6cm}|M{0.4cm}|M{0.4cm}|M{1.1cm}M{0.45cm}M{0.45cm}M{1.2cm}|M{1.1cm}M{0.45cm}M{0.45cm}M{1.2cm}|M{1.1cm}M{0.6cm}M{0.6cm}|M{1.2cm}|}
\hline
    \multicolumn{3}{|c|}{\sc Benchmarks}					&	\multicolumn{4}{c|}{\textsc{Latency} (ms)}							&	\multicolumn{4}{c|}{\sc Effective TFLOPS}							&	\multicolumn{3}{M{3cm}|}{\sc Plasticine Speedup (x)} & \sc Power (W)			\\ \hline
    &	\sc H	&	\sc T	&	\sc Xeon Skylake	&	\sc Tesla V100	&	\sc BW &	\sc Plasticine	&	\sc Xeon Skylake	& Tesla V100	&	\sc BW	&	\sc Plasticine	&	\sc Xeon Skylake	&	\sc Tesla V100	&	\sc BW	 & \sc Plasticine\\ \hline
%% =================== PASTE HERE ========================
\multirow{5}{*}{\sc\bf LSTM}	&	256	&	150	&	15.75	&	1.69	&	0.425	&	0.0419	&	0.010	&	0.09	&	0.37	&	3.8	&	376.3	&	40.4	&	10.2	&	28.5	\\
	&	512	&	25	&	11.50	&	0.60	&	0.077	&	0.0139	&	0.009	&	0.18	&	1.37	&	7.6	&	830.3	&	43.2	&	5.6	&	53.7	\\
	&	1024	&	25	&	107.65	&	0.71	&	0.074	&	0.0292	&	0.004	&	0.59	&	5.68	&	14.4	&	3,686.6	&	24.3	&	2.5	&	97.2	\\
	&	1536	&	50	&	411.00	&	4.38	&	0.145	&	0.1224	&	0.005	&	0.43	&	13.01	&	15.4	&	3,357.8	&	35.8	&	1.2	&	102.7	\\
	&	2048	&	25	&	429.36	&	1.55	&	0.074	&	0.1060	&	0.004	&	1.08	&	22.62	&	15.8	&	4,050.6	&	14.6	&	0.7	&	104.5	\\ \hline
\multirow{6}{*}{\sc\bf GRU}	&	512	&	1	&	0.91	&	0.39	&	0.013	&	0.0004	&	0.003	&	0.01	&	0.25	&	7.6	&	2,182.3	&	942.4	&	31.2	&	61.9	\\
  &	1024	&	1500	&	3,810.00	&	33.77	&	3.792	&	1.4430	&	0.005	&	0.56	&	4.98	&	13.1	&	2,640.3	&	23.4	&	2.6	&	109.1	\\
  &	1536	&	375	&	2,730.00	&	13.12	&	0.951	&	0.7463	&	0.004	&	0.81	&	11.17	&	14.2	&	3,658.3	&	17.6	&	1.3	&	114.6	\\
	&	2048	&	375	&	5,040.00	&	17.70	&	0.954	&	1.2833	&	0.004	&	1.07	&	19.79	&	14.7	&	3,927.5	&	13.8	&	0.7	&	101.2	\\
	&	2560	&	375	&	7,590.00	&	23.57	&	0.993	&	1.9733	&	0.004	&	1.25	&	29.69	&	15.0	&	3,846.4	&	11.9	&	0.5	&	117.2	\\ \hline
\multicolumn{3}{|c|}{\textsc{\bf Geometric Mean}}					&		&		&		&		&		&		&		&		&	2,529.3	&	29.8	&	2.0	&		\\
%% =================== PASTE HERE ========================
\hline
\end{tabular}
\end{table*}

\begin{table}
\caption{Loop unrolling and vectorization parameters for spatial architectures.}
\label{tab:param}
\centering
\scriptsize
\begin{tabular}{|L{0.5cm}|M{0.4cm}|M{0.4cm}|M{0.3cm}|M{0.3cm}|M{0.3cm}|M{0.3cm}|M{0.3cm}|M{0.3cm}|M{0.3cm}|}
\hline
  \multicolumn{3}{|c|}{\sc Benchmarks}						&\multicolumn{3}{c|}{\sc Stratix 9 BW} &						\multicolumn{4}{c|}{\sc Plasticine}							\\\hline
	&	\sc H	&	\sc T	&	$ru$	&	$hv$	&	$rv$	&	$hu$	&	$hv$	&	$ru$	&	$rv$	\\\hline
%% =================== PASTE HERE ========================
\multirow{5}{*}{\sc\bf LSTM}	&	256	&	150	&	\multirow{11}{*}{6}	&	\multirow{11}{*}{400}	&	\multirow{11}{*}{40}	&	6	&	\multirow{11}{*}{1}	&	4	&	\multirow{11}{*}{64}	\\ \cline{7-7} \cline{9-9}	\cline{2-3}
	&	512	&	25	&		&		&		&	\multirow{4}{*}{4}	&		&	\multirow{10}{*}{8}	&		\\	\cline{2-3}
	&	1024	&	25	&		&		&		&		&		&		&		\\	\cline{2-3}
	&	1536	&	50	&		&		&		&		&		&		&		\\	\cline{2-3}
	&	2048	&	25	&		&		&		&		&		&		&		\\ \cline{7-7}	\cline{2-3}
\multirow{6}{*}{\sc\bf GRU}	&	512	&	1	&		&		&		&	\multirow{6}{*}{2}	&		&		&		\\	\cline{2-3}
	&	1024	&	1500	&		&		&		&		&		&		&		\\	\cline{2-3}
	&	1536	&	375	&		&		&		&		&		&		&		\\	\cline{2-3}
	&	2048	&	375	&		&		&		&		&		&		&		\\	\cline{2-3}
	&	2560	&	375	&		&		&		&		&		&		&		\\	\cline{2-3}
	&	2816	&	750	&		&		&		&		&		&		&		\\\hline	\cline{2-3}
%% =================== PASTE HERE ========================
\end{tabular}
\end{table}

\subsection{RNN Performance Analysis} \label{sec:results}
Table \ref{tab:eval} shows the performance comparison of LSTM and GRU with various numbers of hidden units (H) and step sizes (T) over
the four platforms. Overall, both CPU and GPU significantly underutilize the available compute FLOPS.
In addition, they cannot meet the latency requirement for real-time serving for all problem sizes.
Both BW and Plasticine deliver promising latencies within 5ms for all problem sizes.
When serving very large RNNs, BW provides better performance
	with up to 2x better than Plasticine on the largest GRU (H=2816).
% Plasticine's 8-bit multiplier-accumulator (MAC) units are implemented using mixed precision multipliers and adders,
% 	which are more expensive than BW's MAC implementation.
% As a result, Plasticine embeds fewer number of MAC units while scaled at the same peak TFLOPS as BW.
% This explains the performance gap between Plasticine and BW while serving the large GRU.
When serving small and medium size RNNs, Plasticine performs better than BW
	with up to 30x better performance on small GRU (H=512).
We also observe that Plasticine delivers consistent FLOPS when serving all the problem sizes.

\paragraph{Processor-Based Architectures}
For CPU experiments, the RNN kernels from TensorFlow itself is not multi-threaded.
Since we focus on real-time serving of RNN applications, we use batch size of 1 for all of our benchmarks,
	which expose no parallelism outside the kernel level.
Consequently, the machine is still very underutilized even with AVX2 instruction.
Although one could implement RNN directly in c++,
	the MVM sizes in RNNs are too small to benefit from multi-threading due to the synchronization overhead.
V100 with cuDNN library provides significant acceleration compared to CPU.
	Nevertheless, the latency is still high.
This is because GPUs are designed for throughput oriented rather than latency sensitive workloads.
Provided that the library is based on BLAS3 routines, which are matrix-matrix operation, MVMs in 
RNN serving suffer from significant resource underutilization.
In Table \ref{tab:eval}, V100 shows very poor performance on GRU (H=512). This is likely due to
the initialization overhead which should not be timed.
From our evaluation, neither processor-based architectures are suitable for providing low-latency serving on
RNN applications.

\paragraph{Spatial Architectures} Table \ref{tab:param} shows the selected design parameters for each 
problem size for BW and Plasticine.
On Stratix 10, BW uses 6 tile engines ($ru$) with native dimension of 400 ($hv$) and 40 lanes ($rv$).
Large $hv$ and $rv$ improve the data-to-control ratio by amortizing the scheduling overhead over a large vectorized instruction.
However, this design choice aggravates the underutilization for small RNN feature sizes at 256 and 512.
Our implementation effectively uses $hv$ of size 1 by performing dot product instead of MVM, 
which prevents fragmentation in the $H$ dimension.
	With $hv=1$, all the intermediate buffers are stored in registers.
In contrast, BW uses register files of size $hv$.
In addition, our proposed implementation captures additional gate-level, X, and H parallelism as well as pipelining at element-wise functions.
In contrast, BW schedules these operations in time and dispatches corresponding instructions to drive the compute units.

A CGRA is less flexible than an FPGA when performing arbitrary low-precision operations. 
In this example,
	we increase memory density of Plasticine by supporting quantile precisions as described in Section \ref{sec:arch:varprec}.
All weights are stored in 8 bit format, so as the multiplication operations of MVM. 
The reduction and accumulation operations are implemented in mix of 16 and 32 bit precisions.
Hence, the peak FLOPS when performing mixed precision map-reduce is much less than the peak FLOPS for blocked low-precision format in BW.
As a result, Plasticine performs worse than BW on the large RNNs.

In addition, Plasticine delivers very consistent FLOPS for
different problem sizes. For small problem size, the dot product can be fully unrolled with $rv * ru$. Therefore, we can
increase $hu$ to explore additional parallelism across the hidden units. For large problem size, dot product becomes the bottleneck of
the pipeline. Hence, we reduce $hu$ and increase $ru$ to balance the throughput between dot product and element-wise operations.
In this example, BW uses a single set of parameters for all problem sizes.
Although one can potentially tune parameters for different problem sizes,
	doing so will incur re-synthesis and place-and-route on an FPGA,
	which is an order of magnitude longer than the compilation time needed for a CGRA design.
In addition, to exhaust hardware resources with a smaller $hv$,
	one would have to increase the number of matrix vector tile engines $hu\times ru$ in BW.
As a result, decoders and schedulers associated with these units
	will drive up the control-to-data overhead and deliver less FLOPS for larger problem sizes.

\subsection{Area and Power Analysis} \label{sec:results}
Table \ref{tab:spec} shows the die area comparison of different platforms.
	While the GPU has a publicly-reported die area measurement \cite{markidis2018nvidia},
	Xeon Skylake and Stratix 10 only have estimated die areas based on
	their estimated transistor counts \cite{inteldie}.
With the rough area estimates, we can see that while CPU has the smallest area in this case,
	the performance gap is too large even after we scale up to a 28-core server.
The GPU also delivers bad performance per area mostly due to the low utilization of compute FLOPS.
Stratix 10 delivers the best performance for the large RNNs,
	but with the largest die area estimates of 30 billion transistors \cite{stratix10die}.
Plasticine's die area is based on the synthesis results at 28nm,
	which is one generation older than all the other platforms.
With technology scaling,
	Plasticine should possess double the amount of compute and memory resources at 14nm for the same die area,
	which will roughly match Stratix 10's performance on all the RNN problem sizes.
At the same time, Plasticine is more than 2x smaller than Stratix 10,
	which could also contribute at least 2x - 60x performance per area improvement for all problem sizes.
Table \ref{tab:spec} shows the thermal design power (TDP) of the four platforms,
	which is the peak power achievable for any workloads \cite{inteltdp, stratix10tdp, v100spec}.
BW also reports a measured peak power for the given set of benchmarks of 125W.
Table \ref{tab:eval} shows the simulated power for Plasticine for each benchmark.
	Overall, the peak power among benchmarks for Plasticine is 118W,
	which is slightly less than the peak power compared to BW.
% With technology scaling, Plasticine can match the performance of Stratix 10
% on the large problem size with roughly the same power.
 % 2 pages -- 3 pages left after this
\section{Related Work}
\label{sec:related}

\paragraph{Streaming Dataflow IRs}
Although many works claim to emit efficient and information-rich dataflow IRs for the downstream compilers, very few of them can capture the high-level parallel patterns and implementation details that are critical to RDA mappings. For example, TensorFlow \cite{tensorflow} emits dataflow IR composed of tensor operations. However, its IR lacks information on the parallel patterns within these operations. In contrast, most of the streaming languages \cite{streamit, synaid, maxj} are not able to extract nested loop-level parallelism from modern data-intensive applications. For example, StreamIt \cite{streamit}, a language tailored for streaming computing, also adopts distributed control as in \name{}. However, it lacks the necessary language features to describe deeply and irregularly nested loops that are common in modern data-intensive applications.

\paragraph{Hardware Architectures}
Spatial reconfigurable accelerators (\eg, Dyser~\cite{dyser} and Tartan~\cite{tartan}) have only one-level of hierarchy.
Hence, such accelerators' performance can be bottlenecked by their limited interconnect bandwidth and power budget.
Sparse Processing Unit (SPU)~\cite{sparseaccel} can sustain higher interconnect bandwidth by introducing on-chip hierarchy; however, it lacks support for polyhedral memory banking~\cite{poly_cong}, a pivotal optimization to achieve massive parallel accesses to on-chip memory. Plasticine~\cite{plasticine} provides us with the desired architecture features; however, its compiler lacks the necessary components to support efficient streaming execution. Given that Plasticine resembles many key features of the RDA model, we target Plasticine with \name{}.

\paragraph{Spatial Compilers}
Most previous works \cite{nowatzki, spatial-computation} only consider allocating resources at the same level. \name{} takes a more general assumption by co-allocating resources at multiple levels of an accelerator's hierarchy.

The Plasticine compiler~\cite{plasticine} is similar to \name{} that it also uses a token-based control protocol.
However, it performs worse than \name{} due to the following reasons.
First, the Plasticine compiler allocates VBs for every level of Spatial's (a high-level language) control hierarchy. 
The communication between parent and child controllers lead to both communication hotspots around the parent, and bubbles before entering a steady-state of the loop iterations. 
Second, the Plasticine compiler assigns a single memory PB for each logical memory in the Spatial program. 
Hence, it could not handle the case where a logical memory exceeds the capacity or bank limits of the physical PBs.
Third, the Plasticine compiler only supports polyhedral memory partitioning at the first dimension of the on-chip memory. 
Hence, its applicability to data-intensive applications with high-dimension tensor algebra is questionable.
Last, compared to \name{}'s separate allocation and assignment phases described in \Cref{sec:control} and \Cref{sec:decompose},
the Plasticine compiler allocates one VB for a specific type of PB and underutilizes resources within PBs.


% Ignore arch for now; we assume arch

% \paragraph{Plasticine Compiler}
% The Plasticine compiler described in the original paper also uses a token-based control protocal.
% However, the Plasticine compiler still allocates VB for each level of the controller hierarchy in Spatial
% and pass tokens between the parent and child controllers, which creates communication hot-spot around the
% parant controller and suffers from bubble during warmup phase of the loop iteartions.
% \yz{Rephrase this}
% In addition, their compiler assigns a single memory PB to each logical memory in the program; 
% the compiler cannot handle logical memory exceeding capacity or bank limit of the physical PBs.
% The simple strided banking scheme also disallows parallelized access on multiple dimension of the memory,
% which can greatly limits the application design space.
% Finally, unlike spaerate allocation and assignment phases described in \Cref{sec:alloc} and \Cref{sec:pruning},
% they allocate a VB for a specific type of PB, which under under utilizes resource whin PBs.


% \gist{
% CGRA: \\
% \cite{tartan} \\
% Imperative to Spatial Architectures: \\
% \cite{synaid}: 
% \begin{outline}
% \1 Target green array. Tiles of stack based 18-bit processors. Small local memory per
% core.
% \1 Small benchmarks: trigonometric, FFT, bithack, etc..
% \end{outline}
% \cite{zaidi}
% His thesis:\\
% \url{https://www.cl.cam.ac.uk/techreports/UCAM-CL-TR-870.pdf} \\
% Targeting FPGA. We probably don't want to cite this guy.

% Partitioning: \\
% \cite{nowatzki} - Partitioning data flow graph using ISL \\

% Summary:
% Looks like we are not the first one trying to map imperative language to spatial architecture. But there are still major differences.
% Some of the work is mapping to FPGAs, similar to high-level synthesis tools.
% Many of the other works like \cite{zaidi} only focus on the data-flow graph and do not handle memory accesses. Although we both trying to support imperative language on spatial accelerator, our focus is never to accelerate control heavy code on the accelerator, but rather to target data intensive application with more flexibility.

% \cite{synaid} is actually very relevant but the architecture is at a much smaller scale. It does have distributed on-chip memory but their memory is a stack used to store program for processors. None of the prior work has consider using distributed memory to compose logically single memory with strong-consistency and coherence. Another major difference is that these architecture does not have global network and DRAM that introduces variable latency. 
% So this changes how the application is mapped onto the accelerator. Our approach is completely distributed streaming approach while they tends to use statically scheduled approach.
% Finally the applications are very different. Most of their applications are very small kernels or image/audio encoding/decoding. Our benchmarks has a mix of full ML models + graph + others
% }

% Maybe related \cite{sparseaccel}

% Triggered Instruction \cite{ti}
% Tartan\cite{tartan}

% \gist{
% 	\begin{itemize}
% 		\item Plasticine compiler: \cite{plasticine}
% 		\item Tartan: \cite{tartan} Single op partitions w/ handshaking, direct template translation.
% 		\item Zaidi Thesis: \cite{zaidi} Conversion to bluespec, also uses token-based control. No partitioning / merging.
% 		\item Partitioning: \cite{nowatzki} Solver-based partitioning, very small arrays (4x4 dyser, etc.), with statically known delays and single-op partitions. Does not handle merging.
% 		\item SparseAccel: \cite{sparseaccel}\ Architecture paper, basically nothing about compilers.  (NR)
% 		\item streamit for RAW: \cite{streamit} No memory consistency due to pure stream structure, fine grained arch. Mesh of processors.
% 		\item Triggered Instructions \cite{ti} Fine grained processor architecture, with caches. (Micro paper: manual mapping, assembly, NR)
% 		\item Chlorophyll: \cite{synaid} Maps subset of C to GreenArrays (small spatial stack-processor architecture). Distinct Non-distributed arrays vs distributed arrays, optional partition annotations, static for loops (unrestricted while), uses Rosette backend solver (branch and bound, SA, Ant, Tabu Search, picked Simulated Annealing). Actual generation is synthesis based.
% 		\item Spatial computation \cite{spatial-computation} Maps ANSI C to Verilog, handshake + token based execution with speculative execution. Uses crossbar network for memory (although we assume pre-banked memory). Fine grained control, uses Pegasus for memory dependence graph and tokens for synchronization. Does not address partitioning / merging.
% 	\end{itemize}
% }
 % 1/2 page
%\section{Discussion \& Future Work}
\label{sec:discussion}

Currently our compiler map the entire program graph onto the accelerator spatially.
If the program graph cannot fit on the hardware, the programmer needs to break the applications
into multiple kernels, materialize intermediate results into DRAM, and reconfiguring the accelerator.
The partitioning algorithm described in \Cref{sec:splitmerge} can be used to find region of the program 
to partition within minimum live out variables. Doing this automatically can also enable overlapping
of reconfiguration with materializing and prefetching of the data.
The same algorithm can be used to partition the program on multiple chips. Since the control-flow
is completely distributed, the program tolerant arbitrary amount of delay from PCI links between
accelerators.
There is also the opportunity of mapping multiple logical memories with the same on-chip scratchpad. 
This can be achieved in pure software without hardware support. Compiler can perform live range analysis
on the memories and map logical memories with no overlapping live range on the same scratchpad. 
The last actor that access the old memory can pass the token to the first actor that accesses the new memory
to override previous data.
Fine-grain reconfiguration of the compute would requires hardware supports to load multiple configurations in PB.
In order to time-multiplex the computation hardware, such as using the same actor for different
layers of a DNN, currently the user need to write a loop around the reusable section and change dimension
of the kernel based on loop iterator. 
Finally, the distributed control-flow can be also applied on FPGAs to improve clock frequency on large designs.
 % 1/4 page
\balance\chapter{Conclusions (WIP)}

We show that the best network design depends on both applications and the underlying accelerator architecture.
Network performance correlates strongly with bandwidth for streaming accelerators, and scaling raw bandwidth is more area- and energy-efficient with a static network.
We show that the application mapping can be optimized to move less data by using a dynamic network as a fallback from a high-bandwidth static network.
%this contributes a 6.9x average performance per area and 2.3x average energy-efficiency improvement for a static-dynamic hybrid network.
This static-dynamic hybrid network provides a 1.8x energy-efficiency and
2.8x performance advantage over the purely static and purely dynamic networks, respectively.
  %Furthermore, we show that spatial architectures require larger switches as programs get bigger, imposing super-linear scaling of network size with chip area.


 % 1/4 page

\label{lastpage}

%\appendix
%%\chapter{Appendix}

%\section{Context Programming Restriction}

%We use the imperative context configuration for ease of understanding, but it is important to
%realize not all program expressible at this abstraction can be executed by the PCU.

%\begin{figure}
  %\begin{subfigure}[b]{0.5\textwidth}
    %\inputminted{python}{code/dotproduct.py}
    %\caption{Declarative configuration}
  %\end{subfigure}
  %\hfill
  %\begin{subfigure}[b]{0.4\textwidth}
    %\inputminted{python}{code/context_imp.py}
    %\caption{Imperative configuration}
  %\end{subfigure}
  %%\caption[Example PCU configuration]{
  %%}
  %%\label{fig:context}
%\end{figure}

%%
%% The acknowledgments section is defined using the "acks" environment
%% (and NOT an unnumbered section). This ensures the proper
%% identification of the section in the article metadata, and the
%% consistent spelling of the heading.
%\begin{acks}
%To Robert, for the bagels and explaining CMYK and color spaces.
%\end{acks}

%%
%% The next two lines define the bibliography style to be used, and
%% the bibliography file.
\bibliographystyle{ACM-Reference-Format}
\bibliography{paper}

\label{totalpage}

%%
%% If your work has an appendix, this is the place to put it.
%\appendix

%\section{Research Methods}

\end{document}
\endinput
%%
%% End of file `paper.tex'.
