%%
%% Configure the paper for preprint or camera-ready submission (but not both)
\def\setuppreprint{0}
\def\setupcameready{0}

%%
%% \BibTeX command to typeset BibTeX logo in the docs
%\AtBeginDocument{%
%  \providecommand\BibTeX{{%
%    \normalfont B\kern-0.5em{\scshape i\kern-0.25em b}\kern-0.8em\TeX}}}

%%
%% Rights management information.  This information is sent to you
%% when you complete the rights form.  These commands have SAMPLE
%% values in them; it is your responsibility as an author to replace
%% the commands and values with those provided to you when you
%% complete the rights form.
\if\setupcameready1
    \setcopyright{acmcopyright}
    \copyrightyear{2018}
    \acmYear{2018}
    \acmDOI{10.1145/1122445.1122456}
\else
    \setcopyright{none}
    \acmDOI{}
\fi

%%
%% These commands are for a PROCEEDINGS abstract or paper.
\if\setupcameready1
    \acmConference[Woodstock '18]{Woodstock '18: ACM Symposium on Neural
        Gaze Detection}{June 03--05, 2018}{Woodstock, NY}
    \acmBooktitle{Woodstock '18: ACM Symposium on Neural Gaze Detection,
        June 03--05, 2018, Woodstock, NY}
    \acmPrice{15.00}
    \acmISBN{978-1-4503-9999-9/18/06}
\else
    \acmISBN{}
\fi

%%
%% Submission ID.
%% Use this when submitting an article to a sponsored event. You'll
%% receive a unique submission ID from the organizers
%% of the event, and this ID should be used as the parameter to this command.
\if\setupcameready1
    \acmSubmissionID{123-A56-BU3}
\fi

%%
%% The majority of ACM publications use numbered citations and
%% references.  The command \citestyle{authoryear} switches to the
%% "author year" style.
%%
%% If you are preparing content for an event
%% sponsored by ACM SIGGRAPH, you must use the "author year" style of
%% citations and references.
%% Uncommenting
%% the next command will enable that style.
%\citestyle{acmauthoryear}

%% Set top matter
\if\setupcameready1
    \settopmatter{printacmref=true} % Reference format
    \settopmatter{printfolios=false} % Page numbers
\else
    \settopmatter{printacmref=false}
    \settopmatter{printfolios=true}
\fi

%%
%% Removes footnote with conference information in first column
\if\setupcameready0
    \renewcommand\footnotetextcopyrightpermission[1]{}
\fi

%% Remove running page headers
\if\setupcameready1
    \def\removepageheaders{0}
\else
    \def\removepageheaders{1}
\fi

%% Spacing
\setlength{\textfloatsep}{4pt}
%\setlength{\belowcaptionskip}{-4pt}

%%
%% Abbreviations
\newcommand{\ie}{{\em i.e.}}
\newcommand{\eg}{{\em e.g.}}
\newcommand{\ea}{{\em et al.}}
%\newcommand{\name}{\textlambda-NIC\xspace}
\newcommand{\name}{SARA\xspace}
\newcommand{\matchlambda}{Match+Lambda}

%% Bolding
\newcommand{\cemph}[1]{\textcolor{blue}{\texttt{#1}}}

%%
%% Comments
\if\setupcameready1
    \def\showcomments{0}
\else
    \if\setuppreprint1
        \def\showcomments{0}
    \else
        \def\showcomments{1}
    \fi
\fi
\newcommand{\xxx}[1]{\textcolor{red}{#1}}
\newcommand{\gist}[1]{{\color{blue} #1}}
\newcommand{\ms}[1]{\todo[color=teal]{Shahbaz: #1}}
\newcommand{\msil}[1]{\todo[color=blue, inline]{Shahbaz: #1}}
%\newcommand{\sch}[1]{\todo[color=teal]{Sean: #1}}
\newcommand{\yz}[1]{\todo[color=blue!40]{Yaqi: #1}}
\newcommand{\yzil}[1]{\todo[color=blue!40,inline]{Yaqi: #1}}
\newcommand{\nz}[1]{\todo[color=red!40]{Nathan: #1}}
\newcommand{\tz}[1]{\todo[color=green!40]{Tian: #1}}
\newcommand{\xu}[1]{\todo[color=pink!40]{Xu: #1}}
\newcommand{\ignore}[1]{}

%%
%% Hyphenations
\hyphenation{micro-second}
\hyphenation{time-scales}

%%
%% Sections and paragraph settings
\crefformat{section}{\S#2#1#3}
\crefformat{subsection}{\S#2#1#3}
\crefformat{subsubsection}{\S#2#1#3}
\makeatletter
\def\@parfont{\bfseries\itshape}
\def\@subsubsecfont{\sffamily\bfseries\section@raggedright}
\makeatother

%%
%% Listings
\definecolor{codegreen}{rgb}{0,0.6,0}
\definecolor{codegray}{rgb}{0.5,0.5,0.5}
\definecolor{codepurple}{rgb}{0.58,0,0.82}
\definecolor{backcolour}{rgb}{0.95,0.95,0.92}

\lstdefinestyle{match_lambda_style}{
    backgroundcolor=\color{backcolour},
    commentstyle=\color{codegreen},
    keywordstyle=\color{magenta},
    numberstyle=\tiny\color{codegray},
    stringstyle=\color{codepurple},
    basicstyle=\ttfamily\footnotesize,
    breakatwhitespace=false,
    breaklines=true,
    captionpos=b,
    keepspaces=true,
    numbers=left,
    numbersep=5pt,
    showspaces=false,
    showstringspaces=false,
    showtabs=false,
    tabsize=2
}

\lstset{style=match_lambda_style}

%%
%% Math operators
\DeclareMathOperator*{\argmax}{argmax}
\DeclareMathOperator*{\elmax}{elmax} % Elementwise maximum
\DeclareMathOperator*{\dest}{dest} % Destinations for an operator (its consumers)
\DeclareMathOperator*{\andf}{and}
\DeclareMathOperator*{\maximum}{maximum}

\newcommand{\projb}[1]{\text{proj}_\mathbf{B}(#1)}
\newcommand{\nnint}{\mathbb{Z}_{\ge 0}}
\newcommand{\ind}[1]{\mathbb{1}[#1]}

%% 
