\documentclass{article}

  % Recommended, but optional, packages for figures and better typesetting:
  \usepackage{microtype}
  \usepackage{graphicx}
  \usepackage{subfigure}
  \usepackage{booktabs} % for professional tables
  \usepackage{caption}
  %\usepackage{hyperref}
  \usepackage[hyphenbreaks]{breakurl}
  \usepackage[hyphens]{url}

  % hyperref makes hyperlinks in the resulting PDF.
  % If your build breaks (sometimes temporarily if a hyperlink spans a page)
  % please comment out the following usepackage line and replace
  % \usepackage{sysml2019} with \usepackage[nohyperref]{sysml2019} above.

  % Attempt to make hyperref and algorithmic work together better:
  \newcommand{\theHalgorithm}{\arabic{algorithm}}
  \newcommand\TODO[1]{\textcolor{red}{(TODO: #1)}}
  \newcommand{\gist}[1]{{\color{blue} GIST: #1 \\}}
  \newcommand\dash{\textnormal{-}}

  % Use the following line for the initial blind version submitted for review:
  \usepackage[accepted]{sysml2019}
  %\usepackage[nohyperref]{sysml2019}
  \usepackage{amsmath}
  \usepackage{listings}
  \usepackage{tabularx}
  \usepackage{multirow}
  \usepackage{array}
  \usepackage{mathtools}
  \DeclarePairedDelimiter{\ceil}{\lceil}{\rceil}

  \newcolumntype{M}[1]{>{\centering\arraybackslash}m{#1}}
  \newcolumntype{L}[1]{>{\raggedright}p{#1}}

  \usepackage{listings}           % Code listing
\usepackage{courier}
\definecolor{vbgray}{gray}{0.9}
\definecolor{darkgreen}{RGB} {0, 100, 0}
\definecolor{darkred}{RGB} {255, 0, 0}
\definecolor{orange}{RGB} {255, 128, 0}
\def\tick{{\color{darkgreen} \textbf{\tikz\fill[scale=0.5](0,.35) -- (.25,0) -- (1,.7) -- (.25,.15) -- cycle;}}}
\def\ytick{{\color{orange} \textbf{\tikz\fill[scale=0.5](0,.35) -- (.25,0) -- (1,.7) -- (.25,.15) -- cycle;}}}
\def\x{{\color{darkred} {$\bm{\times}$}}}

\definecolor{red}{RGB}{255,0,0}
\definecolor{vbgray}{gray}{0.9}
\definecolor{darkgreen}{RGB} {0, 100, 0}
\definecolor{darkred}{RGB} {255, 0, 0}
\definecolor{blue}{RGB} {0, 135, 255}
\definecolor{yellow}{RGB} {224, 173, 0}
\definecolor{codegreen}{RGB}{52,123,0}
\definecolor{codegray}{rgb}{0.6,0.5,0.5}
\definecolor{codepurple}{rgb}{0.58,0,0.82}
\definecolor{backcolour}{rgb}{0.95,0.95,0.95}
\definecolor{lightback}{rgb}{0.95,0.95,0.95}
\def\tick{{\color{darkgreen} \textbf{\tikz\fill[scale=0.5](0,.35) -- (.25,0) -- (1,.7) -- (.25,.15) -- cycle;}}}
\def\ytick{{\color{orange} \textbf{\tikz\fill[scale=0.5](0,.35) -- (.25,0) -- (1,.7) -- (.25,.15) -- cycle;}}}
\def\x{{\color{darkred} {$\bm{\times}$}}}

\lstdefinelanguage{Spatial}{
  basicstyle=\fontsize{7}{7}\selectfont\ttfamily,frame=tlbr,framesep=4pt,framerule=0pt,
  tabsize=2,
  basewidth={0.55em, 0.4em},%
  numbers=left,
  showspaces=false,
  keywordstyle=\bfseries,
  breaklines=true,
  columns=fixed,
  %xleftmargin=0.25in,
  firstnumber=auto,
  showstringspaces=false,
  escapechar=@,
  escapeinside={(*@}{@*)},
  morestring=[b]",
  morestring=[b]',
  morecomment=[l]{//},
  morecomment=[s]{/*}{*/},
  backgroundcolor=\color{backcolour},
  commentstyle=\color{codegreen}, %\bfseries,
  numberstyle=\tiny\color{codegray},
  stringstyle=\color{codepurple},
  keywordstyle=[2]\color{blue},
  keywords=[2]{val, def, type},
  keywordstyle=[3]\color{yellow}\bfseries,
  keywords=[3]{Float, Float8, Int, String, T, Void, Bit, Half, FltPt},
  keywordstyle=[4]\color{orange}\bfseries,
  keywords=[4]{Matrix, Array},
  keywordstyle=[5]\color{blue}\bfseries,
  keywords=[5]{StreamIn, StreamOut, DRAM, ArgIn, ArgOut, HostIO, RegFile, Reg, SRAM, SRAM1, SRAM2, FIFO, LIFO, LUT, LineBuffer},
  keywordstyle=[6]\bfseries,
  keywords=[6]{enq, deq, load, store, scatter, gather, :=, push, pop, peek},
  keywordstyle=[7]\color{magenta},
  keywords=[7]{until, par, by, value},
  keywordstyle=[8]\color{red}\bfseries,
  keywords=[8]{C0,C1,C2,C3,C4,C5,C6,C7,C8,C9,C10},
  keywordstyle=\color{magenta}\bfseries,
  morekeywords={Foreach,Reduce,MemReduce,MemFold,Fold,Accel,Stream,FSM,Sequential,if,else,Parallel,Pipe, DummyPipe}
}


  % If accepted, instead use the following line for the camera-ready submission:
  %\usepackage[accepted]{sysml2019}

  % The \sysmltitle you define below is probably too long as a header.
  % Therefore, a short form for the running title is supplied here:
  \sysmltitlerunning{Serving Recurrent Neural Networks Efficiently with a Spatial Accelerator}

  \begin{document}

  \twocolumn[
  \sysmltitle{Serving Recurrent Neural Networks Efficiently with a Spatial Accelerator}

  % It is OKAY to include author information, even for blind
  % submissions: the style file will automatically remove it for you
  % unless you've provided the [accepted] option to the sysml2019
  % package.

  % List of affiliations: The first argument should be a (short)
  % identifier you will use later to specify author affiliations
  % Academic affiliations should list Department, University, City, Region, Country
  % Industry affiliations should list Company, City, Region, Country

  % You can specify symbols, otherwise they are numbered in order.
  % Ideally, you should not use this facility. Affiliations will be numbered
  % in order of appearance and this is the preferred way.
  \sysmlsetsymbol{equal}{*}

  \begin{sysmlauthorlist}
  \sysmlauthor{Tian Zhao}{sf}
  \sysmlauthor{Yaqi Zhang}{sf}
  \sysmlauthor{Kunle Olukotun}{s}

  \end{sysmlauthorlist}

  \sysmlaffiliation{sf}{Department of Electrical Engineering, Stanford University, Stanford, USA}

  \sysmlcorrespondingauthor{Tian Zhao}{tianzhao@stanford.edu}
  \sysmlcorrespondingauthor{Yaqi Zhang}{yaqiz@stanford.edu}
  \sysmlcorrespondingauthor{Kunle Olukotun}{kunle@stanford.edu}

  % You may provide any keywords that you
  % find helpful for describing your paper; these are used to populate
  % the "keywords" metadata in the PDF but will not be shown in the document
  \sysmlkeywords{Model Serving, Parallel System}

  \vskip 0.3in
  \prefacesection{Abstract}

%In this talk, I will talk about architectural design and compilation techinques that improves the scaling efficiency of a RDA developed 
%at Stanford--Plasticine.
%Staring with a static-dynamic hybrid network, we show that the hybrid network can improves energy efficiency while providing
%guaranteed success in placement and routing.
%Next, I will talk about compiler techinques that convert applications' complex control hierarchies, such as nested loops and branch conditions, into a streaming dataflow representation that can be efficiently executed by Plasticine with distributed on-chip resources.
%The compiler implements (a) a peer-to-peer (p2p) control paradigm inferred from an imperative programming style that minimizes synchronization overhead, and (b) a mapping strategy that decomposes the computation and memory in a program across a distributed heterogeneous resources.
%By applying these techniques, we show that Plasticine is able to outperform state of art accelerators, such as GPUs and FPGAs, in
%both performance and performance/Watt in various dense, sparse, and streaming applications.

With the slowdown of Moore’s Law, specialized hardware accelerators are gaining tractions 
for delivering 100-1000x performance improvement over general-purpose processors. 
As the performance scaling in multicores is coming to a limit~\cite{multicorescale}, a new
class of accelerators--reconfigurable dataflow architectures (RDAs)--is promising in 
offering high-throughput and energy-efficient acceleration that keeps up with the performance demand.
Instead of dynamically fetching instructions like in traditional processors, RDAs have flexible datapath 
that can be statically configured to spatially parallelize and pipeline the program across
distributed on-chip resources. 
The pipelined execution model and explicitly-managed scratchpad in RDAs eliminate
the performance, area, and energy overhead in dynamic scheduling and conventional memory hierarchy.

To adapt to the compute intensity in modern data-analytic workloads, particularly in the deep learning
domain, RDAs are increasing to a scale that was unprecedented before, compared to the classic coarse-grained
reconfigurable architectures (CGRAs).
With an area footprint of $133\text{mm}^2$ at 28nm, 
Plasticine is a hierarchical RDA with 12.3 TFLOPs of compute power~\cite{plasticine}.
Prior work has shown an up to 76x performance/watt benefit from Plasticine over a Stradix V FPGA 
due to advantage in clock frequency and resource density.
The increase in scale introduces new challenges in on-chip network design to maintain 
the throughput and energy efficiency of RDAs.
Furthermore, targeting and managing RDAs at this scale also require new strategies in mapping, 
memory management, and flexible control to fully utilize the compute power of the accelerator. 

In this work, we focus on two aspects of the software-hardware co-design that impact the usability
and scalability of the Plasticine accelerator. 
One of the biggest challenges that hinders the adoption of these
accelerators is the low-level declarative configuration interface that requires the programmers to
have detailed knowledge about the underlying microarchitecture implementation and hardware
constraints. To address the programmability challenge, we introduce a compiler stack that provides a high-level
programming interface that efficiently translates imperative control constructs to streaming
dataflow execution with minimum synchronization overhead on an on-chip distributed architecture. The
compiler handles the hardware constraints systematically with resource virtualization. To address
the performance challenges, we present a comprehensive study on the on-chip network design for 
reconfigurable dataflow architectures that sustain performance in a scalable fashion with high energy efficiency.

  ]

  % this must go after the closing bracket ] following \twocolumn[ ...

  % This command actually creates the footnote in the first column
  % listing the affiliations and the copyright notice.
  % The command takes one argument, which is text to display at the start of the footnote.
% The \sysmlEqualContribution command is standard text for equal contribution.
  % Remove it (just {}) if you do not need this facility.
  % https://github.com/stanford-ppl/papers.git
  \printAffiliationsAndNotice{}
  % leave blank if no need to mention equal contribution
  % \printAffiliationsAndNotice{\sysmlEqualContribution} % otherwise use the standard text.

  \chapter{Introduction (WIP)}

With the end of Dennard Scaling~\cite{dennard}, the amount of performance one can extract from a CPU is reaching a limit.
To provide general-purpose flexibility, CPU spends the majority of energy on overheads, including dynamic-instruction execution, branch prediction, and a cache hierarchy, and less than 20\% of the energy on the actual computation~\cite{mark}.
Even worse, the power wall is limiting the entire multicore family
to reach the doubled performance improvement per generation enabled by technology scaling in the past\cite{multicorescale}.

  \chapter{Background (WIP)}

\section{Execution Schedule of Reconfigurable Architectures} 
\begin{figure*}
\begin{subfigure}[b]{0.34\textwidth}
\inputminted{python}{code/spatialeg2.py}
\caption {
}
\end{subfigure}
\hfill
\begin{subfigure}[b]{0.65\textwidth}
\centering
\includegraphics[width=1.0\textwidth]{figs/pipeexec.pdf}
\caption {
}
\end{subfigure}
\caption[Hiearchical pipelining and parallelization on spatial architecture]{
Hierarchical pipelining and parallelization in spatial architecture.
(a) illustrates the runtime and throughput of a hierarchically pipelined and parallelized program on
a reconfigurable spatial architecture. 
At inner level, instructions within each basic
block are fine-grained pipelined across iterations of the inner most loop. 
At outer level, the inner loops are coarse-grained pipelined across the outer loop iterations.
Exploiting multiple levels of pipeline parallelism gives a total throughput of $x+y$ operations per
  cycle, where \emph{x} and \emph{y} are number of operations in the basic blocks.
(b) Vectorizing the inner most loops B and C by \texttt{n} increases the throughput to $(x+y)n$.
(c) Parallelizing the outer loop A by \texttt{m} further increases the throughput to $(x+y)mn$.
}
\label{fig:pipeexec}
\end{figure*}

\begin{figure*}
\centering
\includegraphics[width=0.4\textwidth]{figs/peakutil.pdf}
\caption[Average utilization vs. peak compute density tradeoff]{
 Average utilization vs. peak compute density tradeoff among different architectures.
}
\label{fig:peakutil}
\end{figure*}

\begin{figure*}
\centering
\includegraphics[width=1\textwidth]{figs/perfmodel.pdf}
\caption[High-level performance model of spatial architectures]{
High-level performance model of spatial architectures
}
\label{fig:perfmodel}
\end{figure*}

The key advantage of reconfigurable spatial accelerators, compared to processor-based architectures, 
is the ability to explore multiple levels of pipeline parallelism. 
In traditional Von Neumann architectures~\cite{vonneumann}, like CPUs and GPUs,
a computer consists of a processing unit that performs
computation, a memory unit that stores the program states, and a control unit that tracks execution
states and fetch the instruction to execute. This computing model inherently assumes that
instructions with in a program are executed in-time, maximizing the flexibility to 
context switching between different workloads dynamically.

Reconfigurable accelerators are a direct violation of the von Neumann execution model; 
instructions are statically imbedded in the datapath and executed in-space as supposed to in-time.
One of the disadvantage of reconfigurable hardware is paying the resource cost for infrequently
executed instructions, making it unsuitable for control-heavy workloads that traditional
processors are efficient at.
On the other hand, RDAs are particularly competitive in providing high-throughput, 
low-latency, and energy-efficiency acceleration for these applications.
Data-analytical workloads encompass a wide domains of applications, including image processing,
recognition, machine translation, digital signal processing, network processing, etc.
These applications exhibits a rich amount of data-level parallelism with relatively static control
flow.

\Cref{fig:pipeexec} shows an example of exploiting hierarchical parallization and pipelining on
a spatial architecture, where overall throughput equals to the product of total parallelization factors 
and pipelining depth.
By exploring multiple dimensions of concurrency in the program, spatial architecture is more likely
to achieve a good compute throughput for a wide range of applications.
For applications that are expensive to parallelize due to irregular access patterns, spatial
architectures can increase concurrency on the pipelining dimension.  For application with
embarrassingly parallel workloads, reconfigurable accelerator can 

Another benefit of pipelined execution is easier to achieve good memory performance.
Data accessed by different stage of the pipelines are stored in discrete scratchpads 
instead of a shared cache; improving the effective on-chip bandwidth and capacity.
Using explicitly managed scratchpad also tends to improve locality and 
eliminate cache performance issues, such thrashing.
Across kernels, pipelined execution reduces the amount of off-chip accesses for intermediate
data.
SIMT architectures, like GPUs, relying on high-bandwidth DRAM technology, such has HBM, to sustain
the compute throughput of massively parallelized threads.
While providing over 10x more bandwidth than traditional DDR technologies, HBM is very limited in
capacity, around 16GB as supposed to on the orders of TB for DDR.
As a result, the limited off-chip capacity often restricts the type of applications that
GPUs can support.

%\begin{table*}
  %\centering
%\begin{tabular}{lccc}
  %\toprule
 %Concurrency Level & Instruction & Data & Task/Kernel  \\ \midrule
 %Parallelsim & CPU,\rda & CPU,GPU,\rda & CPU,\rda  \\
 %Pipelining & \rda & \rda & \rda \\
 %\bottomrule
%\end{tabular}
%\caption[Concurrency level explored by different architectures]{
  %Concurrency level explored by different architectures
%}
%\label{tab:conclevel}
%\end{table*}

\section{Plasticine}

\begin{figure*}
\centering
\includegraphics[width=0.8\textwidth]{figs/plasticine.pdf}
\caption[Plasticine chip-level architecture]{Plasticine chip-level architectural diagram}
\label{fig:plasticine}
\end{figure*}

\section{Spatial}

\begin{figure}
\centering
%\newsavebox{\outerProduct}
%\begin{lrbox}{\outerProduct}
\lstinputlisting[language=Spatial,linewidth=0.6\columnwidth]{code/OuterProduct.scala}
%\end{lrbox}
%\begin{tabular}{m{0.01cm} l} & \usebox{\outerProduct}\\ \end{tabular}
  %\inputminted[fontsize=\footnotesize]{scala}{code/OuterProduct.scala}
  \caption{Example of Outer Product in Spatial.}
\label{fig:spatial_app}
\end{figure}

%To target spatial architectures, we use Spatial, an open source domain specific language for reconfigurable accelerators \cite{spatial_koeplinger}.
We use Spatial~\cite{spatial_koeplinger}, an domain specific language for reconfigurable accelerators, 
as the front-end of Plasticine.
Spatial describes applications with nested loops and an explicit memory hierarchy that captures data movement on-chip and off-chip. 
This exposes design parameters that are essential for achieving high performance on spatial architectures, including blocking size, loop unrolling factors, inner-loop pipelining, and coarse-grained pipelining of arbitrarily nested loops. 
To enable loop-level parallelization and pipelining, Spatial automatically banks and buffers intermediate memories between loops. 
An example of outer product---element-wise multiplication of two vectors resulting in a matrix---in Spatial is shown in Figure~\ref{fig:spatial_app}.
%In this example we assume inputs \emph{vecA}, \emph{vecB} and outputs \emph{matC} do not fit on chip.
%First, \emph{C2} and \emph{C4} load tiles of vectors of size \emph{tsA} and \emph{tsB} to on-chip scratchpads \emph{tileA} and \emph{tileB}. 
%Next, loop \emph{C5} computes the outer products and store it to scratchpad \emph{tileC}. 
%Finally, \emph{C6} stores partial results back to DRAM. 
\if 0
Spatial enables inner loop pipelining in \emph{C5} and coarse-grained pipelining between stages of the outer loop (e.g. \emph{C4}, \emph{C5}, and \emph{C6} are pipelined across iterations of \emph{C3}). 
The parallelization factor of the inner most loop (\emph{ip} for \emph{C2}, \emph{C5}, and \emph{C6}) translates to SIMD pipeline and vector network vectorization factor. 
In \emph{C1} and \emph{C2}, \emph{op1} and \emph{op2} are outer loop parallelization factors that allow the programmer to unroll the outer loops and parallelize compute, which can better saturates DRAM bandwidth or balances compute pipelines. 
When scratchpad producers or consumers are parallelized, the scratchpad must be banked to sustain the required bandwidth. 
Scratchpads only contain one level of banking hierarchy. 
Therefore, when more than one dimension of the scratchpad is banked, the high-dimensional banks are mapped across multiple scratchpads. 
In this example, if both \emph{ii} and \emph{jj} (used in the write address of \emph{tileC}) on line 31 are parallelized, \emph{tileC} will be mapped to multiple scratchpads. 
This mapping strategy makes broadcast communication common between producers, banks, and consumers when outer loops are unrolled.
\fi
For spatial architectures, Design Space Exploration (DSE) of parameters
(e.g., \emph{op1}, \emph{op2}, \emph{ip}, \emph{tsA}, \emph{tsB}) is critical to achieve good resource utilization and performance \cite{dse_koeplinger}.


  \section{RNN Computation Analysis} 
\label{sec:app}
In this section, we first discuss the limitation of BLAS-based LSTM on processor and spatial architectures.
Next, we discuss our implementation of loop-based LSTM on spatial architectures.
Table \ref{tab:legend_app} contains specifications for symbols and parameters
  used in this section.
\begin{table}[t]
  \vskip 0.15in
  \centering
  \scriptsize
  \begin{tabular}{p{0.6cm}m{3cm}m{3cm}}
  \toprule
    Symbol & Processor & Reconfigurable Hardware \\
    \midrule
    \includegraphics[width=0.03\columnwidth]{figs/innerloop.png} & Kernel & Inner Loop \\
    \includegraphics[width=0.03\columnwidth]{figs/onchip.png} & Memory Hiearchy & On-chip Scratchpad \\
    \includegraphics[width=0.03\columnwidth]{figs/reg.png} & Register File & Register \\
    \includegraphics[width=0.1\columnwidth]{figs/unrollparm.png} & & Unrolling factor using multiple hardware compute blocks \\ \midrule
    \includegraphics[width=0.03\columnwidth]{figs/vec.png} & \multicolumn{2}{L{6.5cm}}{Element-wise Operation} \\
    \includegraphics[width=0.03\columnwidth]{figs/outerloop.png} & \multicolumn{2}{L{6.5cm}}{Outer Loop} \\
    \includegraphics[width=0.1\columnwidth]{figs/vecparam.png} & \multicolumn{2}{L{6.5cm}}{Vectorization parameter for AVX or SIMD instructions} \\
    \midrule
    \midrule
    Parameter & \multicolumn{2}{l}{Specification} \\
    \midrule
    $hv$      & \multicolumn{2}{l}{Vectorization parameter on H} \\
    $hu$      & \multicolumn{2}{l}{Unrolling factor on H} \\
    $rv$      & \multicolumn{2}{l}{Vectorization parameter on R} \\
    $ru$      & \multicolumn{2}{l}{Unrolling factor on R} \\
    $G$       & \multicolumn{2}{l}{Number of gates in an RNN. For LSTM, G=4} \\
  \bottomrule
  \end{tabular}
  \caption{Specifications for symbols and parameters in Section \ref{sec:app}.}
  \label{tab:legend_app}
  \vskip -0.1in
  \end{table}

\begin{figure*}
  \centering
  \includegraphics[width=1.5\columnwidth]{figs/cpugpulstm.pdf}
  \caption{Compute and memory layout of TensorFlow \texttt{BasicLSTM} cell on CPU (a) and \texttt{CudnnLSTM} cell on GPU (b).}\label{fig:tf_lstm}
\end{figure*}

\subsection{BLAS-based LSTM on Processor Architecture}
Modern Machine Learning frameworks, e.g.
  TensorFlow \cite{abadi2016tensorflow},
  divide the computation graph of an LSTM cell into BLAS kernels.
Then, the BLAS kernel is accelerated by calling low-level
optimized BLAS subroutines such as Intel BLAS Library on CPU
and NVBLAS Library on GPU.
Figure \ref{fig:tf_lstm} (a) shows the computation graph of a \texttt{BasicLSTM} cell in TensorFlow.
This implementation can lead to large memory footprint since all the intermediate results are
  materialized in memory.
A common strategy to tackle the issue is through fusing blocked kernels.
With TensorFlow's abstraction, this can only be achieved by
  expressing the entire RNN cell as an optimized kernel.
For example, TensorFlow provides \texttt{LSTMBlockFusedCell} and \texttt{GRUBlockCell} modules,
  which are the fastest TensorFlow implementations of RNN cells for CPU.
In practice, such implementation can provide significant performance improvement
  over the \texttt{BasicLSTM} implementation.
However, it is still very hard to saturate CPU compute capacity, potentially
due to the high synchronization overhead across threads.
Figure \ref{fig:tf_lstm} (b) shows the computation layout of TensorFlow with
cuDNN library \cite{chetlur2014cudnn} on GPU. cuDNN is an NVIDIA GPU
library for accelerating deep neural networks.
To minimize the data movement,
  cuDNN fuses all the vector-vector (VV) operations after MVM. Specifically, the bias add in
  Equation \ref{eq:1}, \ref{eq:2}, \ref{eq:3}, \ref{eq:4},
  and all the operations in Equation \ref{eq:5}, \ref{eq:6},
  are fused into one kernel.
Nevertheless, there are still intermediate buffers of size $H$
  between the MVM kernel and the element-wise operations.

Compared to the \texttt{BasicLSTM} implementation,
  \texttt{CudnnLSTM} eliminates most of large intermediate memories.
However, the MVMs of Equation \ref{eq:1}, \ref{eq:2}, \ref{eq:3}, \ref{eq:4} are all accelerated
with BLAS3 kernels, which performs only matrix-matrix level operations.
This turns MVM and VV bias add into Matrix Matrix Multiplication (MMM) and Matrix Matrix
Addition (MMA), which leads to serious underutilization of GPU.

Moreover, a processor-based architecture introduces large energy overhead of instruction
  decoding and scheduling.
GPU especially suffers from its power-hungry, high-throughput memory hierarchy.
For these reasons, both the CPU and GPU architectures are not suitable
  for energy-efficient, low-latency RNNs serving platforms.

\subsection{BLAS-based LSTM on Spatial Architecture}

Previous work has studied the capability of using an FPGA as a low-latency serving platform.
An FPGA has the flexibility of resizing MVM and VV units based on the application size.
In addition, MVM and VV units can be implemented with hardware pipelines,
  which removes the instruction scheduling and control overhead on a processor-based
  architecture.
The latest version of Intel Stratix 10 FPGA further boosts the compute power of FPGA
  with increasing number of hardened digital signal processing (DSP) blocks
  and on-chip memory capacity.
Microsoft Brainwave (BW) \cite{fowers2018configurable}
  is a state-of-the-art FPGA-based deep learning framework.

Figure \ref{fig:bw_lstm} shows BW's compute and memory layout.
In contrast to the CPU and GPU implementations, BW blocks the MVM along both
  row and column dimensions.
It then fuses the inner tiled MVM with element-wise non-linear functions.
Specifically for a matrix of size $H\times R$ and a vector of size $R$,
BW parallelizes the compute of multiple column tiles ($ru$, \# MV Tiles in the original paper) of size
$hv\times rv$ with multiple tiled engines, as shown in Figure \ref{fig:bwt} (a). 
Each tile engine contains $hv$ (native dimension) number of dot
product engines vectorized by $rv$ (lanes) and achieves one tile per cycle throughput. 
Parallel tiles along the row dimension are then fed into a pipelined reduction and accumulation
unit.
Immediately after the accumulation, the multi-function units (MFUs) execute the
element-wise operations on the $hv$ vector chunk produced by the accumulator.
Although BW's implementation still keeps the vectorized intermediate results, the size $hv$ is much
smaller than $H$ in \texttt{BasicLSTM} cell.
Nonetheless, with parallelization in $ru$,
  BW allocates lots of vectorized intermediate buffers that can still lead to energy inefficiency.
BW performs one MVM operation in $\ceil[\big]{\frac{H}{hv}}\ceil[\big]{\frac{R}{rv\cdot ru}}$
  iterations.

The MVM operations are executed on each gate of the LSTM sequentially.
Similarly, element-wise operations $hv$ using $\sigma, \tanh, \circ, +$ for the non-linear 
operators are also scheduled to execute on the vectorized multi-function units with size
of $hv$, as shown with the arrow in time in Figure \ref{fig:bw_lstm}.
To avoid DRAM communication overhead and improve compute density, Brainwave embeds MVM in a blocked floating-point format, 
where the vector of $hv$ values share a single 5-bit exponent and have distinct signs and 2-5 bit mantissa for
each value. As a result, they can achieve very dense low-precision compute and storage, with one
adder per $hv$ values and $hv$ multipliers for a vector of $hv$. The remaining operations are
performed in 16-bit precision.

When matrix dimensions cannot be divided by $hv$ and $rv\cdot ru$, Brainwave suffers from
underutilization of the compute FLOPS, as shown in Figure \ref{fig:bwt} (a).
The underutilization is worse with small problem sizes.
In addition, BW computes $W_xX$ and $W_hH$ separately rather than computing them with concatenated larger matrices,
  which can further aggravate the problem.
This might be because BW's abstraction does not allow partial updates of an vector
  but only $X$ is updated at the end of the step.


\subsection{Loop-based LSTM}
We have made the following observations
  from analyzing BLAS-based LSTM implementations:
\begin{enumerate}
\item 
  Constructing an LSTM cell's computation graph using BLAS subroutines
  introduces large intermediate buffers even when the kernels themselves are blocked.
  Each element on RNN cells' non-reduction dimension of the MVM ($H$)
  can be computed completely independently within one time step. This
  exposes the opportunity of fine-grain loop tiling and fusion across the
  entire LSTM kernel. 
\item MVM is the computation bottleneck in serving RNN cells. Spatial
  architecture allows us to distribute most of the compute resource to MVM 
  by parallelizing and pipelining MVM with element-wise operations.
\item Using low-precision operations can boost compute density and keep RNN
  weights on-chip to avoid high-latency DRAM communication. We need to introduce
  efficient low-precision support in the target spatial architecture without introduce
  too much overhead.
\end{enumerate}

To address the issue of large intermediate buffers, we fine-grain tile and fuse
MVM with non-linear functions. We refer to the computation for generating 
every single element in $c_t$ and $h_t$ as LSTM-1 operation, which can be computed
independently in a single step. LSTM-1 is composed of four independent dot products of 
the row vectors of the weight matrices with the input vector immediately followed by the 
element-wise operations on output of the dot product. The resulting $c$ and $t$ vectors are 
produced by computing LSTM-1 operations for $H+D$ iterations.

\label{sec:blaslstm}
 \begin{figure}
  \centering
  \includegraphics[width=0.8\columnwidth]{figs/bwlstm.pdf}
  \caption{Compute and memory layout of LSTM in Brainwave.}\label{fig:bw_lstm}
   \vspace*{-0.3in}
\end{figure}

\begin{figure}
 \centering
  \includegraphics[width=0.9\columnwidth]{figs/splstm.pdf}
  \caption{Compute and memory layout of a loop-based LSTM design.}\label{fig:spatial_lstm}
  \vspace*{-0.2in}
\end{figure}
\begin{figure}
  \centering
  \includegraphics[width=\columnwidth]{figs/bwfrag.pdf}
   \caption{Fragmentation in an MVM-based design (a) and a loop-based design (b) in MVM.}\label{fig:bwt}
\end{figure}

As shown in Figure \ref{fig:spatial_lstm}, each MVM unit is replaced by a MapReduce unit to 
compute the tiled dot product.
Each MapReduce is vectorized by $rv$ with pipelined map function followed by a pipelined reduction tree.
$ru$ is the number of parallel MapReduce units. Results of $ru$ MapReduce blocks are reduced and 
accumulated with another reduction tree (not shown in Figure). Next, the dot product result is passed 
through a chain of function units for executing bias add and non-linear functions. Dot products, bias adds, 
and non-linear functions of the four gates can also be parallelized. Finally, the results of the four gates are pipelined
through a set of function units for element-wise operation in LSTM cell.
At the outer loop, LSTM-1 runs for $\frac{H}{hu}$ iterations,
  where $hu$ is the number of parallel LSTM-1 implementations.

In the loop-based design, all intermediate buffers are scalars as supposed to vectors. 
Regarding utilization, the loop-based LSTM design suffers from less underutilization due to unaligned problem size 
compared to the tiled MVM approach in BW. Figure \ref{fig:bwt} shows sources of such underutilizations.
An MVM approach design would suffer from 2-D fragmentation on both the $H$ and $D$ dimensions (Figure \ref{fig:bwt} (a)), whereas
the loop-based design only suffers from 1-D fragmentation on the $R$ dimension (Figure \ref{fig:bwt} (b)).

%Using BLAS terminology,
  %an LSTM-1 operation can be thought of as a new BLAS level-1 routine,
  %i.e. element-wise operations.
%In contrast, previous works mostly focus on optimizing level-2 or 3 routines.
%We find that optimizing at the lowest BLAS level allows us to utilize hardware resources more efficiently.
%In addition, using parallel patterns and loop constructs as the basic building block
  %offers sufficient abstraction and does not lead to much engineering overhead.
\begin{figure}
  \centering
  \newsavebox{\lstm}
  \begin{lrbox}{\lstm}
    \lstinputlisting[language=Spatial,linewidth=1.0\columnwidth]{code/lstm.scala}
  \end{lrbox}
  \begin{tabular} {c}
    \usebox{\lstm} \\
  \end{tabular}
  \caption{Example of LSTM in Spatial.}
\label{fig:spatial_app}
\end{figure}

Figure \ref{fig:spatial_app} shows a loop-based LSTM design implemented in Spatial.
\textbf{Foreach} is a loop construct with a lambda that takes loop iterator as input.
\textbf{Reduce} is a construct that executes MapReduce by taking a map function followed
by a reduction function. User declare explicit on-chip scratchpads and registers with
\textbf{SRAM} and \textbf{Reg}.
To enable fine-tuning an RNN application, we exposes loop vectorization factor $rv, hv$ and
unrolling factors $hu, ru$.

  \input{text/lowprec.tex}
  \section{Evaluation} \label{sec:eval}
In this section, we evaluate the real-time RNN serving tasks on various platforms.
We start with the methodology of our experiments, followed by a discussion of performance and power comparisons across
these platforms.

\subsection{Methodology} \label{sec:methodology}
To evaluate RNN serving, we use the LSTM and GRU tasks from Baidu DeepBench as our benchmarks.
We evaluate the benchmarks across processor-based architectures including CPU and GPU, 
and spatial architectures including FPGA and CGRA.
Table \ref{tab:spec} shows the detailed specifications of the targeting hardware, 
which includes state-of-the-art high performance platforms in each of the commercialized categories.
Table \ref{tab:appconf} summarizes application configurations of each platform.

\paragraph{CPU} We implement the applications in TensorFlow 1.10, and evaluate our implementations on 
Intel Xeon Scalable Processor (Skylake) CPU.
We use the \texttt{LSTMBlockFusedCell} and \texttt{GRUBlockCell} kernels in TensorFlow.
We further enable AVX2 vector instructions for CPU evaluation. Due to lack of low-precision
support in both tool chain and platform, we use single-precision for our implementation.

\paragraph{GPU} We use TensorFlow with cuDNN Library to target NVIDIA Tesla V100 GPU from Google Cloud. 
cuDNN is a GPU-accelerator Library from NVIDIA that is specialized for deep learning.
We use 16-bit precision for our implementation on GPU.
On both CPU and GPU platforms, we run \emph{TensorFlow} profilers and collect the time spent 
only on evaluating the RNN cells.

\paragraph{Plasticine} We implement the applications in Spatial, which targets Plasticine.
Although Spatial has FPGA back-end support, Stratix 10 is not commercially available at the time of the submission of this work.
The current FPGA targets that Spatial support are not comparable to Stratix 10 both in terms of memory and compute capacity. 
Therefore, we only use Spatial to target Plasticine for this evaluation. However, our approach should generally benefit
an implementation on a high performance FPGA like Stratix 10.
We choose Plasticine configuration that matches the peak 8-bit FLOPS and
on-chip scratchpad capacity of a Stratix 10 FPGA. The exact configuration of Plasticine is shown in Table \ref{tab:conf}.
In order to minimize the overhead of low-precision support, Plasticine only supports 8-bit, 16-bit, and 32-bit element-wise 
operations, and mixed precision reduction operation. 
For our evaluation, the element-wise operations are performed in 8-bit precision, 
the first stage of the reduction is performed in 16-bit, 
while the remaining of the reduction and accumulation are performed in 32 bit operations.

To measure the performance, we use a cycle accurate simulator for Plasticine. 
We modified the simulator to model the proposed micro-architectural changes to support low-precision operations.
We use the area and power of individual CUs and network switches from the original Plasticine paper, 
and compute total area of configuration shown in Table \ref{tab:conf}. 
As discussed in Section \ref{sec:arch}, we reduce the number of stages in PCU from 6 stages to 4 stages with fused low-precision
operations and folded reduction tree. 
Low preicision function units can be used to compose full precision units. 
We conservatively estimate the area and power of PCU stays the same with our proposed change and reduced two stages. 
We also increase the PMU to PCU ratio to better match the compute to memory
ratio for RNN inference applications. To match the memory capacity of Stratix 10, we shrink the scratchpad capacity of 
each PMU from 256kB to 84kB.
For power calculations, we generate activity tracing of the CUs from simulation, and then integrate 
with characterized power of individual PCU to compute the total power. The power and area characterizations are based off
synthesis at 28nm technology at 1GHz clock frequency.

\paragraph{Brainwave} Finally, we also compared our results to Microsoft Brainwave framework.
For this evaluation, we compare to Brainwave implemented
on top of Intel Stratix 10 FPGA. Brainwave is synthesized at 250MHz and all operations are performed in
blocked low-precision floating-point format described in section~\ref{sec:blaslstm}.
\begin{table}[t]
\caption{Plasticine configuration.}
\label{tab:conf}
\centering
\scriptsize
\begin{tabular}{L{3cm}rL{2.5cm}r}
\toprule
\# Row                     & 24   & \# Column        & 24  \\
\# PCU                     & 192  & \# PMU           & 384 \\
\# Lanes in PCU            & 16   & \# Stages in PCU & 4   \\
Scrachpad capacity per PMU & 84kB &                  &     \\
\bottomrule
\end{tabular}
\end{table}

%% Stratix 10 2800 M20K MBits 229, MLAB 15 MBits, = 30.5MB
%% Source https://www.intel.com/content/dam/www/programmable/us/en/pdfs/literature/hb/stratix-10/s10-overview.pdf
\begin{table}
\caption{Hardware specifications for target platforms.}
\label{tab:spec}
\scriptsize
\centering
\begin{tabular}{L{2.5cm}M{1.2cm}M{0.8cm}M{0.8cm}M{1cm}}
\toprule
  Specification        & Intel Xeon Skylake (Dual core) & Tesla V100 SXM2 & Stratix 10 280 FPGA & Plasticine\\
\midrule
Max Clock Rate (GHz) & 2.0/2.8*                  & 1.38/1.53*      & 1                   & 1 \\
On-chip memory** (MB) & 55                        & 20              & 30.5                & 31.5\\
Peak 32-bit TFLOPS   & --                      & 15.7            & 10                  & 12.5\\
Peak 8-bit TFLOPS    & --                        & --              & 48                  & 49\\
Technology ($nm$)    & 14                        & 12              & 14                  & 28\\
Die Area ($mm^2$)    & 64.4                      & 815             & 1200                & 494.37 \\
  TDP (W)    & 15                      & 300             & 148                & 160 \\
\bottomrule
\end{tabular}
* Base/Boosted Frequency
** Capacity of L3 cache for CPU, register file for GPU, and on-chip scratchpad for reconfigurable architectures.
  %* Computed with AVX512f instructions. \cite{markidis2018nvidia}
\end{table}

\begin{table}
\caption{Application configurations for target platforms.}
\label{tab:appconf}
\centering
\scriptsize
\begin{tabular}{L{1.8cm}M{1cm}M{1cm}M{1cm}M{1.5cm}}
\toprule
Platform                       & Intel Xeon Skylake & Tesla V100 SXM2 & Stratix 10 280 FPGA & Plasticine\\
\midrule
Software Framework             & TF+AVX2                   & TF+cuDNN        & Brainwave           & Spatial \\
Achieved Clock Frequency (GHz) & 2                         & 1.38            & 0.25                & 1 \\
Precision                      & f32                       & f16             & blocked precision   & mix f8+16+32\\
\bottomrule
\end{tabular}
\end{table}

\begin{table*}
\caption{Performance comparison of DeepBench Inference.}
\label{tab:eval}
\centering
\scriptsize

%%% =================== With Utilization  ========================
%\begin{tabular}{|L{0.6cm}|M{0.6cm}|M{0.6cm}|M{1.1cm}M{0.6cm}M{0.6cm}M{1.2cm}|M{1.1cm}M{0.6cm}M{0.6cm}M{1.2cm}|M{0.6cm}M{1.2cm}|M{1.2cm}|}
%\hline
  %\multicolumn{3}{|c|}{\sc Benchmarks}					&	\multicolumn{4}{c|}{\textsc{Latency} (ms)}							&	\multicolumn{4}{c|}{\sc Effective TFLOPS}							&	\multicolumn{2}{M{2cm}|}{\textsc{8-bit FLOPS Utilization} (\%)}			&	\sc Power (W)	\\ \hline
  %&	\sc H	&	\sc T	&	\sc Xeon Skylake	&	\sc Tesla V100	&	\sc BW &	\sc Plasticine	&	\sc Xeon Skylake	& Tesla V100	&	\sc BW	&	\sc Plasticine	&	\sc BW	&	\sc Plasticine	&	\sc Plasticine	\\ \hline
%%% =================== PASTE HERE ========================
%\multirow{5}{*}{\sc\bf LSTM}	&	256	&	150	&	15.75	&	1.69	&	0.425	&	0.042	&	0.010	&	0.09	&	0.4	&	3.8	&	0.8	&	7.7	&		\\
	%&	512	&	25	&	11.50	&	0.60	&	0.077	&	0.015	&	0.009	&	0.18	&	1.4	&	7.0	&	2.8	&	14.3	&		\\
	%&	1024	&	25	&	107.65	&	0.71	&	0.074	&	0.037	&	0.004	&	0.59	&	5.7	&	11.2	&	11.8	&	22.9	&		\\
	%&	1536	&	50	&	411.00	&	4.38	&	0.145	&	0.160	&	0.005	&	0.43	&	13.0	&	11.8	&	27.1	&	24.1	&		\\
	%&	2048	&	25	&	429.36	&	1.55	&	0.074	&	0.106	&	0.004	&	1.08	&	22.7	&	15.8	&	47.3	&	32.3	&		\\ \hline
%\multirow{6}{*}{\sc\bf GRU}	&	512	&	1	&	1.00	&	1.00	&	0.013	&	1.000	&	0.003	&	0.00	&	0.2	&	0.0	&	0.5	&	0.0	&		\\
	%&	1024	&	1500	&	449.00	&	33.77	&	3.792	&	1.720	&	0.042	&	0.56	&	5.0	&	11.0	&	10.4	&	22.4	&		\\
	%&	1536	&	375	&	2,730.00	&	13.12	&	0.951	&	0.910	&	0.004	&	0.81	&	11.2	&	11.7	&	23.3	&	23.8	&		\\
	%&	2048	&	375	&	5,040.00	&	17.70	&	0.954	&	1.580	&	0.004	&	1.07	&	19.8	&	12.0	&	41.2	&	24.4	&		\\
	%&	2560	&	375	&	7,590.00	&	23.57	&	0.993	&	2.460	&	0.004	&	1.25	&	29.7	&	12.0	&	61.9	&	24.5	&		\\
	%&	2816	&	750	&	25,850.00	&	55.48	&	1.987	&	6.430	&	0.003	&	1.29	&	35.9	&	11.1	&	74.9	&	22.7	&		\\
%%% =================== PASTE HERE ========================

%% =================== With Speedup and no power ========================
  \begin{tabular}{|L{0.6cm}|M{0.4cm}|M{0.4cm}|M{1.1cm}M{0.45cm}M{0.45cm}M{1.2cm}|M{1.1cm}M{0.45cm}M{0.45cm}M{1.2cm}|M{1.1cm}M{0.6cm}M{0.6cm}|M{1.2cm}|}
\hline
    \multicolumn{3}{|c|}{\sc Benchmarks}					&	\multicolumn{4}{c|}{\textsc{Latency} (ms)}							&	\multicolumn{4}{c|}{\sc Effective TFLOPS}							&	\multicolumn{3}{M{3cm}|}{\sc Plasticine Speedup (x)} & \sc Power (W)			\\ \hline
    &	\sc H	&	\sc T	&	\sc Xeon Skylake	&	\sc Tesla V100	&	\sc BW &	\sc Plasticine	&	\sc Xeon Skylake	& Tesla V100	&	\sc BW	&	\sc Plasticine	&	\sc Xeon Skylake	&	\sc Tesla V100	&	\sc BW	 & \sc Plasticine\\ \hline
%% =================== PASTE HERE ========================
\multirow{5}{*}{\sc\bf LSTM}	&	256	&	150	&	15.75	&	1.69	&	0.425	&	0.0419	&	0.010	&	0.09	&	0.37	&	3.8	&	376.3	&	40.4	&	10.2	&	28.5	\\
	&	512	&	25	&	11.50	&	0.60	&	0.077	&	0.0139	&	0.009	&	0.18	&	1.37	&	7.6	&	830.3	&	43.2	&	5.6	&	53.7	\\
	&	1024	&	25	&	107.65	&	0.71	&	0.074	&	0.0292	&	0.004	&	0.59	&	5.68	&	14.4	&	3,686.6	&	24.3	&	2.5	&	97.2	\\
	&	1536	&	50	&	411.00	&	4.38	&	0.145	&	0.1224	&	0.005	&	0.43	&	13.01	&	15.4	&	3,357.8	&	35.8	&	1.2	&	102.7	\\
	&	2048	&	25	&	429.36	&	1.55	&	0.074	&	0.1060	&	0.004	&	1.08	&	22.62	&	15.8	&	4,050.6	&	14.6	&	0.7	&	104.5	\\ \hline
\multirow{6}{*}{\sc\bf GRU}	&	512	&	1	&	0.91	&	0.39	&	0.013	&	0.0004	&	0.003	&	0.01	&	0.25	&	7.6	&	2,182.3	&	942.4	&	31.2	&	61.9	\\
  &	1024	&	1500	&	3,810.00	&	33.77	&	3.792	&	1.4430	&	0.005	&	0.56	&	4.98	&	13.1	&	2,640.3	&	23.4	&	2.6	&	109.1	\\
  &	1536	&	375	&	2,730.00	&	13.12	&	0.951	&	0.7463	&	0.004	&	0.81	&	11.17	&	14.2	&	3,658.3	&	17.6	&	1.3	&	114.6	\\
	&	2048	&	375	&	5,040.00	&	17.70	&	0.954	&	1.2833	&	0.004	&	1.07	&	19.79	&	14.7	&	3,927.5	&	13.8	&	0.7	&	101.2	\\
	&	2560	&	375	&	7,590.00	&	23.57	&	0.993	&	1.9733	&	0.004	&	1.25	&	29.69	&	15.0	&	3,846.4	&	11.9	&	0.5	&	117.2	\\ \hline
\multicolumn{3}{|c|}{\textsc{\bf Geometric Mean}}					&		&		&		&		&		&		&		&		&	2,529.3	&	29.8	&	2.0	&		\\
%% =================== PASTE HERE ========================
\hline
\end{tabular}
\end{table*}

\begin{table}
\caption{Loop unrolling and vectorization parameters for spatial architectures.}
\label{tab:param}
\centering
\scriptsize
\begin{tabular}{|L{0.5cm}|M{0.4cm}|M{0.4cm}|M{0.3cm}|M{0.3cm}|M{0.3cm}|M{0.3cm}|M{0.3cm}|M{0.3cm}|M{0.3cm}|}
\hline
  \multicolumn{3}{|c|}{\sc Benchmarks}						&\multicolumn{3}{c|}{\sc Stratix 9 BW} &						\multicolumn{4}{c|}{\sc Plasticine}							\\\hline
	&	\sc H	&	\sc T	&	$ru$	&	$hv$	&	$rv$	&	$hu$	&	$hv$	&	$ru$	&	$rv$	\\\hline
%% =================== PASTE HERE ========================
\multirow{5}{*}{\sc\bf LSTM}	&	256	&	150	&	\multirow{11}{*}{6}	&	\multirow{11}{*}{400}	&	\multirow{11}{*}{40}	&	6	&	\multirow{11}{*}{1}	&	4	&	\multirow{11}{*}{64}	\\ \cline{7-7} \cline{9-9}	\cline{2-3}
	&	512	&	25	&		&		&		&	\multirow{4}{*}{4}	&		&	\multirow{10}{*}{8}	&		\\	\cline{2-3}
	&	1024	&	25	&		&		&		&		&		&		&		\\	\cline{2-3}
	&	1536	&	50	&		&		&		&		&		&		&		\\	\cline{2-3}
	&	2048	&	25	&		&		&		&		&		&		&		\\ \cline{7-7}	\cline{2-3}
\multirow{6}{*}{\sc\bf GRU}	&	512	&	1	&		&		&		&	\multirow{6}{*}{2}	&		&		&		\\	\cline{2-3}
	&	1024	&	1500	&		&		&		&		&		&		&		\\	\cline{2-3}
	&	1536	&	375	&		&		&		&		&		&		&		\\	\cline{2-3}
	&	2048	&	375	&		&		&		&		&		&		&		\\	\cline{2-3}
	&	2560	&	375	&		&		&		&		&		&		&		\\	\cline{2-3}
	&	2816	&	750	&		&		&		&		&		&		&		\\\hline	\cline{2-3}
%% =================== PASTE HERE ========================
\end{tabular}
\end{table}

\subsection{RNN Performance Analysis} \label{sec:results}
Table \ref{tab:eval} shows the performance comparison of LSTM and GRU with various numbers of hidden units (H) and step sizes (T) over
the four platforms. Overall, both CPU and GPU significantly underutilize the available compute FLOPS.
In addition, they cannot meet the latency requirement for real-time serving for all problem sizes.
Both BW and Plasticine deliver promising latencies within 5ms for all problem sizes.
When serving very large RNNs, BW provides better performance
	with up to 2x better than Plasticine on the largest GRU (H=2816).
% Plasticine's 8-bit multiplier-accumulator (MAC) units are implemented using mixed precision multipliers and adders,
% 	which are more expensive than BW's MAC implementation.
% As a result, Plasticine embeds fewer number of MAC units while scaled at the same peak TFLOPS as BW.
% This explains the performance gap between Plasticine and BW while serving the large GRU.
When serving small and medium size RNNs, Plasticine performs better than BW
	with up to 30x better performance on small GRU (H=512).
We also observe that Plasticine delivers consistent FLOPS when serving all the problem sizes.

\paragraph{Processor-Based Architectures}
For CPU experiments, the RNN kernels from TensorFlow itself is not multi-threaded.
Since we focus on real-time serving of RNN applications, we use batch size of 1 for all of our benchmarks,
	which expose no parallelism outside the kernel level.
Consequently, the machine is still very underutilized even with AVX2 instruction.
Although one could implement RNN directly in c++,
	the MVM sizes in RNNs are too small to benefit from multi-threading due to the synchronization overhead.
V100 with cuDNN library provides significant acceleration compared to CPU.
	Nevertheless, the latency is still high.
This is because GPUs are designed for throughput oriented rather than latency sensitive workloads.
Provided that the library is based on BLAS3 routines, which are matrix-matrix operation, MVMs in 
RNN serving suffer from significant resource underutilization.
In Table \ref{tab:eval}, V100 shows very poor performance on GRU (H=512). This is likely due to
the initialization overhead which should not be timed.
From our evaluation, neither processor-based architectures are suitable for providing low-latency serving on
RNN applications.

\paragraph{Spatial Architectures} Table \ref{tab:param} shows the selected design parameters for each 
problem size for BW and Plasticine.
On Stratix 10, BW uses 6 tile engines ($ru$) with native dimension of 400 ($hv$) and 40 lanes ($rv$).
Large $hv$ and $rv$ improve the data-to-control ratio by amortizing the scheduling overhead over a large vectorized instruction.
However, this design choice aggravates the underutilization for small RNN feature sizes at 256 and 512.
Our implementation effectively uses $hv$ of size 1 by performing dot product instead of MVM, 
which prevents fragmentation in the $H$ dimension.
	With $hv=1$, all the intermediate buffers are stored in registers.
In contrast, BW uses register files of size $hv$.
In addition, our proposed implementation captures additional gate-level, X, and H parallelism as well as pipelining at element-wise functions.
In contrast, BW schedules these operations in time and dispatches corresponding instructions to drive the compute units.

A CGRA is less flexible than an FPGA when performing arbitrary low-precision operations. 
In this example,
	we increase memory density of Plasticine by supporting quantile precisions as described in Section \ref{sec:arch:varprec}.
All weights are stored in 8 bit format, so as the multiplication operations of MVM. 
The reduction and accumulation operations are implemented in mix of 16 and 32 bit precisions.
Hence, the peak FLOPS when performing mixed precision map-reduce is much less than the peak FLOPS for blocked low-precision format in BW.
As a result, Plasticine performs worse than BW on the large RNNs.

In addition, Plasticine delivers very consistent FLOPS for
different problem sizes. For small problem size, the dot product can be fully unrolled with $rv * ru$. Therefore, we can
increase $hu$ to explore additional parallelism across the hidden units. For large problem size, dot product becomes the bottleneck of
the pipeline. Hence, we reduce $hu$ and increase $ru$ to balance the throughput between dot product and element-wise operations.
In this example, BW uses a single set of parameters for all problem sizes.
Although one can potentially tune parameters for different problem sizes,
	doing so will incur re-synthesis and place-and-route on an FPGA,
	which is an order of magnitude longer than the compilation time needed for a CGRA design.
In addition, to exhaust hardware resources with a smaller $hv$,
	one would have to increase the number of matrix vector tile engines $hu\times ru$ in BW.
As a result, decoders and schedulers associated with these units
	will drive up the control-to-data overhead and deliver less FLOPS for larger problem sizes.

\subsection{Area and Power Analysis} \label{sec:results}
Table \ref{tab:spec} shows the die area comparison of different platforms.
	While the GPU has a publicly-reported die area measurement \cite{markidis2018nvidia},
	Xeon Skylake and Stratix 10 only have estimated die areas based on
	their estimated transistor counts \cite{inteldie}.
With the rough area estimates, we can see that while CPU has the smallest area in this case,
	the performance gap is too large even after we scale up to a 28-core server.
The GPU also delivers bad performance per area mostly due to the low utilization of compute FLOPS.
Stratix 10 delivers the best performance for the large RNNs,
	but with the largest die area estimates of 30 billion transistors \cite{stratix10die}.
Plasticine's die area is based on the synthesis results at 28nm,
	which is one generation older than all the other platforms.
With technology scaling,
	Plasticine should possess double the amount of compute and memory resources at 14nm for the same die area,
	which will roughly match Stratix 10's performance on all the RNN problem sizes.
At the same time, Plasticine is more than 2x smaller than Stratix 10,
	which could also contribute at least 2x - 60x performance per area improvement for all problem sizes.
Table \ref{tab:spec} shows the thermal design power (TDP) of the four platforms,
	which is the peak power achievable for any workloads \cite{inteltdp, stratix10tdp, v100spec}.
BW also reports a measured peak power for the given set of benchmarks of 125W.
Table \ref{tab:eval} shows the simulated power for Plasticine for each benchmark.
	Overall, the peak power among benchmarks for Plasticine is 118W,
	which is slightly less than the peak power compared to BW.
% With technology scaling, Plasticine can match the performance of Stratix 10
% on the large problem size with roughly the same power.

  \chapter{Related Work} \label{sec:related}

Multiple decades of research have resulted in a rich body of literature, both in CGRAs~\cite{cgraSurvey1, cgraSurvey2} and on-chip networks~\cite{ocn-synthesis}. We discuss relevant prior work under the following categories:

\section{Traditional CGRAs and their Compilers}
Emerging around the 1990s and rapidly developed in 2000s, CGRAs were first introduced as near-ASIC efficient accelerators with
post-fabrication reconfigurability that provide a high-speed 
DSP datapath in a very long instruction word (VLIW) design~\cite{cgrasurvey,cgra1,cgra2,adres,trips,pactxpp,dyser,ti}.
Most of the early CGRAs are small reconfigurable fabrics tightly integrated with a 
processor~\cite{dyser, adres,morphosys,piperanch}. 
The processor frequently invokes and reconfigures the CGRA to accelerate compute-intensive regions of the program.
For example, DySER communicates data and enables memory accesses through instruction registers set and
received by a processor~\cite{dyser}.
ADRES uses a shared register file to communicate with a VLIW core~\cite{adres}.
MorphoSys extends the RISC ISA to configure the accelerator and initiate memory transfer
to the accelerator's on-chip buffer~\cite{morphosys}.
The accelerator, often time, handles a subgraph of a basic block at a time, leaving the control
handling to a CPU~\cite{dyser, adres, piperanch}. 
PipeRanch~\cite{piperanch}, for example, require a straight-line single assignment program, 
achieved by fully unrolling loops and inlining functions.
This strategy, however, does not work with applications requiring data-dependent control flow and irregular memory
accesses, such as sparse applications, and dense applications with intensive computation that are
too expensive to fully unroll.

The co-processor model and low-level ISAs in these CGRAs introduce challenges in the programmability
of the hardware.
\cite{dyser} identifies a series of manually performed transformations required to efficiently convert the input program to the expected program graph of a DySER array.
Prior works also have looked into using integer linear programming (ILP) to address constraints in traditional CGRAs. 
\cite{nowatzki} uses ILP to optimize the scheduling, placement, and routing of a dataflow graph within a basic block on the DySER fabric.

Nonetheless, this level of CGRA-processor integration undermines the processor performance due to frequent accelerator
interruptions. 
Furthermore, it is hard to scale in compute and memory density demanded by today's data-analytic
workloads.
Plasticine is a recently proposed large-scale CGRA 
designed as an independent accelerator with dedicated accelerator DRAM~\cite{plasticine}.
The host communicates to the accelerator via a PCIe bus, similar to an FPGA and a GPU.
With the flexibility of a CGRA, Plasticine delivers comparable compute and memory density to
a fixed-function ML accelerator.

At high-level, Plasticine resembles a DySER array replacing the function units with a compute or
memory tile, each supplying a compute density comparable to the entire DySER array.
As a large reconfigurable accelerator, Plasticine takes a longer time to reconfigure, around 10s
of $\mu s$ as supposed to 10s of cycles in traditional CGRAs~\cite{dyser}.
As a result, the expected reconfiguration window for Plasticine is around milliseconds or even
seconds.
To keep the accelerator busy for such a long period of time, Plasticine must support more flexible
control flow in the program to minimize host intervention that compromises the utilization of the hardware.
The consideration of flexible control supports with scalable performance and memory
bandwidth is what differentiates \name's approach from conventional CGRAs' mapping strategy.
While Chlorophyll targeting a GreenArrays 144 chip also support branches, their compute tile is a processor that can execute instructions unlike the SIMD pipeline in Plasticine~\cite{synaid}. 
Chlorophyll focuses on data partitioning across distributed scratchpads. However,
they do not explore pipeline parallelism across compute tiles.
Instead of accelerating a single basic block at a time in traditional CGRAs, 
\name pipelines the entire program with nested control flow onto Plasticine.
By exploring multiple-degrees of concurrency, including instruction-level, data-level, and kernel-level
pipelining and parallelism, \name achieves a good utilization on kernels with both massive and fine-grained
basic blocks.
%The difference in scale determines a completely different mapping strategy that \name exploits compared
%to traditional compiler works on CGRAs.

Wave DPU is another large-scale RDA developed commercially that targets deep learning applications.
It uses a dataflow execution model with a globally asynchronous and locally synchronous (GALS) scheme
that ensures a highly scalable architecture~\cite{wavecomputing}.
With three levels of hierarchy, Wave DPU includes a mesh of CGRA clusters partitioned into compute
machines that provide multiple memory and data I/O ports to an SoC AXI4 NOC; each cluster
further contains a grid of accumulation PEs and arithmetic units.
Like Plasticine, Wave DPU is also an independent accelerator that does not require a CPU to facilitate
execution.
Wave claims to be able to execute any language that can be compiled to an LLVM IR.
From the description of the hardware, it is unclear whether Wave can handle data-dependent control flow and
irregular memory accesses outside of the deep learning domain.

%\subsection{Streaming Dataflow IRs}
%Many recent work in high-level DSL uses a dataflow oriented programming abstraction.
%Example includes ML frameworks like TensorFlow~\cite{tensorflow} that uses a dataflow graph to
%describe the network architecture.
%These dataflow graphs, however, cannot be directly mapped to a dataflow architecture, due to the
%mismatching in their node definitions.
%A node in TensorFlow data graph corresponds to a kernel, which itself is 
%one or multiple invocations of accelerator kernel that contains a sub dataflow graph.
%\name provides an imperative programming front-end for Plasticine to implement these kernels in a
%high-level abstraction that enables cross-kernel optimizations.

%StreamIt is a DSL for streaming applications targeting the RAW architecture,
%a microprocessor array exposing scalable ISAs that maps instructions across
%processors.
%StreamIt have explicit streaming interface across processing pipelines, similar to the output of
%\name's imperative to streaming transformation.
%Unlike Plasticine, the compute engine for RAW is a processor core that can execute arbitrary
%controlflow.

%Although many works claim to emit efficient and information-rich dataflow IRs for the downstream compilers, 
%very few of them can capture the high-level parallel patterns and implementation details that are critical 
%to RDA mappings. 
%For example, TensorFlow \cite{tensorflow} emits dataflow IR composed of tensor operations. 
%However, its IR lacks information on the parallel patterns within these operations. 
%In contrast, most of 
%the streaming languages \cite{streamit, synaid, maxj} are not able to extract nested loop-level parallelism 
%from modern data-intensive applications. 
%For example, StreamIt \cite{streamit}, a language tailored for streaming computing, also adopts distributed 
%control as in \name{}. 
%However, it lacks the necessary language features to describe deeply and irregularly nested loops that are 
%common in modern data-intensive applications.

%\paragraph{Hardware Architectures}
%Spatial reconfigurable accelerators (\eg, Dyser~\cite{dyser} and Tartan~\cite{tartan}) have only one-level of hierarchy.
%Hence, such accelerators' performance can be bottlenecked by their limited interconnect bandwidth and power budget.
%Sparse Processing Unit (SPU)~\cite{sparseaccel} can sustain higher interconnect bandwidth by introducing on-chip hierarchy; however, it lacks support for polyhedral memory banking~\cite{poly_cong}, a pivotal optimization to achieve massive parallel accesses to on-chip memory. Plasticine~\cite{plasticine} provides us with the desired architecture features; however, its compiler lacks the necessary components to support efficient streaming execution. Given that Plasticine resembles many key features of the RDA model, we target Plasticine with \name{}.

%\paragraph{Machine Learning Compilers}
%\begin{outline}
%\1 TensorFlow
%\1 Theano: A Python framework for fast computation of mathematical expressions.
%\1 Cognitive computing programming paradigm: a corelet language for composing networks of neurosynaptic
%cores. I
%\cite{onnc}
%\end{outline}

%Library approach
%\begin{outline}
%\1 Mxnet: A flexible and efficient machine learning library for heterogeneous distributed systems
%\1 Anintroductiontocomputational networks and the computational network toolkit
%\1 Neural Network Transformation and Co-design under Neuromorphic Hardware Constraints.
%\1 Caffe: Convolu- tional architecture for fast feature embedding.
%\1 Cambricon: An instruction set architecture for neu- ral networks.
%\1 Tangram: Optimized Coarse-Grained Dataflow for Scalable NN Accelerators
%\end{outline}

%Kernel-specific loop optimization
%\begin{outline}
%\1 Tangram: Optimized Coarse-Grained Dataflow for Scalable NN Accelerators
%\end{outline}

%\paragraph{ML Accelerators}
%\begin{outline}
%%% Dataflow accelerators
%\1 A runtime reconfigurable dataflow processor for vision
%\1 Tangram: Optimized Coarse-Grained Dataflow for Scalable NN Accelerators
%\1 %% domain-specific loop fashion interchange
%\1 Truenorth: Design and tool flow of a 65 mw 1 million neuron programmable neurosynaptic chip. from
%IBM
%\1 Memristive Boltzmann machine: A hardware accelerator for combinatorial optimization and deep
%learning. 
%\1 Scalable hierarchical network-on-chip architecture for spiking neural network hardware
%implementations. 
%\1 Diannao: A small-footprint high-throughput accelerator for ubiquitous machine-learning.
%\1 Pudiannao: A polyvalent machine learning accelerator. In ACM SIGARCH Computer Architecture News, Vol.
%43. ACM, 369–381.
%\1 Eyeriss: An energy-efficient reconfigurable accelerator for deep convolutional neural networks. I
%\1 EIE: efficient inference engine on compressed deep neural network.
%\1 Design and evolution of modular neural network architectures
%\1 In-Datacenter Performance Analysis of a Tensor Processing Unit. 
%\1 RENO: A high- efficient reconfigurable neuromorphic computing accelerator design.
%\1 Convolution engine: balancing efficiency \& flexibility in specialized computing.
%\1 Minerva: Enabling low-power, highly accurate deep neural network accelerators. 
%\1 DNPU: An 8.1TOPS/W Reconfigurable CNN-RNN Processor for General Purpose Deep Neural Networks. 
%\1 C-Brain: A deep learning accelerator that tames the diversity of CNNs through adaptive data-level
%\1 parallelization. 
%\end{outline}

%\subsection{Spatial Compilers}
%Most previous works \cite{nowatzki, spatial-computation} only consider allocating resources at the same level. 
%\name{} takes a more general assumption by co-allocating resources at multiple levels of an accelerator's hierarchy.

%The Plasticine compiler~\cite{plasticine} is similar to \name{} that it also uses a token-based control protocol.
%However, it performs worse than \name{} due to the following reasons.
%First, the Plasticine compiler allocates VBs for every level of Spatial's (a high-level language) control hierarchy. 
%The communication between parent and child controllers lead to both communication hotspots around the parent, and bubbles before entering a steady-state of the loop iterations. 
%Second, the Plasticine compiler assigns a single memory PB for each logical memory in the Spatial program. 
%Hence, it could not handle the case where a logical memory exceeds the capacity or bank limits of the physical PBs.
%Third, the Plasticine compiler only supports polyhedral memory partitioning at the first dimension of the on-chip memory. 
%Hence, its applicability to data-intensive applications with high-dimension tensor algebra is questionable.
%Last, compared to \name{}'s separate allocation and assignment phases described in
%\Cref{sec:control} and \Cref{sec:resalloc},
%the Plasticine compiler allocates one VB for a specific type of PB and underutilizes resources within PBs.

\section{On-Chip Networks}

\subsection{Tiled Processor Interconnects} Architectures such as Raw~\cite{raw} and Tile~\cite{tile} use scalar operand networks~\cite{son}, which combine static and dynamic networks. Raw has one static and two dynamics interconnects: the static interconnect is used to route normal operand traffic, one dynamic network is used to route runtime-dependent values which could not be routed on the static network, and the second dynamic network is used for cache misses and other exceptions. Deadlock avoidance is guaranteed only in the second dynamic network, which is used to recover from deadlocks in the first dynamic network. However, as described in Section~\ref{sec:network}, wider buses, and larger flit sizes create scalability issues with two dynamic networks, including higher area and power. In addition, our static VC allocation scheme ensures deadlock freedom in our single dynamic network, obviating the need for deadlock recovery.
The dynamic Raw network also does not preserve operand ordering, requiring an operand reordering mechanism at every tile.

TRIPS~\cite{trips} is a tiled dataflow architecture with dynamic execution. TRIPS does not have a static interconnect but contains two dynamic networks~\cite{trips-network}: an operand network to route operands between tiles, and an on-chip network to communicate with cache banks. Wavescalar~\cite{wavescalar} is another tiled dataflow architecture with four levels of hierarchy, connected by dynamic interconnects that vary in topology and bandwidth at each level. The Polymorphic Pipeline Array~\cite{ppa} is a tiled architecture built to target mobile multimedia applications. While compute resources are either statically or dynamically provisioned via hardware virtualization support, communication uses a dynamic scalar operand network.

\subsection{CGRA Interconnects}
Many previously proposed CGRAs use a word-level static interconnect, which has better compute density than bit-based routing~\cite{bus-fpga}. CGRAs such as HRL~\cite{hrl}, DySER~\cite{dyser}, and Elastic CGRAs~\cite{elasticCGRAs} commonly employ two static interconnects: a word-level interconnect to route data and a bit-level interconnect to route control signals.
Several works have also proposed a statically scheduled interconnect~\cite{van2009static, dimitroulakos2006exploring, wave} using a modulo schedule. While this approach is effective for inner loops with predictable latencies and fixed initiation intervals, variable latency operations and hierarchical loop nests add scheduling complexity that prevents a single modulo schedule. HyCube~\cite{hycube} has a similar statically scheduled network, with the ability to bypass intermediate switches in the same cycle. This allows operands to travel multiple hops in a single cycle, but creates long wires and combinational paths and adversely affects the clock period and scalability.
%However, applications with a hierarchical nesting of loops provide two poor options to arrive at a single modulo schedule: the II can be extended to cover the entirety of the innermost loop, or the outer loop can have resources reserved in the inner loop schedule.
%The first option is frequently not realistic because scheduled hardware has a hard cap on the II, and loops can be of arbitrary length.
%If outer loops have resources reserved in the inner loop schedule, then the schedule will become congested with reserved, but infrequently used resources.
\subsection{Design Space Studies} Several prior studies focus on tradeoffs with various network topologies, but do not characterize or quantify the role of dynamism in interconnects.
The Raw design space study~\cite{dse-raw} uses an analytical model for applications as well as architectural resources to perform a sensitivity analysis of compute and memory resources focused on area, power, and performance, without varying the interconnect. The ADRES design space study~\cite{dse-adres} focuses on area and energy tradeoffs with different network topologies with the ADRES~\cite{adres} architecture, where all topologies use a fully static interconnect. KressArray Xplorer~\cite{dse-kressarray} similarly explores topology tradeoffs with the KressArray~\cite{kress} architecture. Other studies explore topologies for mesh-based CGRAs~\cite{dse-date} and more general CGRAs supporting resource sharing~\cite{dse-tvlsi}. Other tools like Sunmap~\cite{sunmap} allow end-users to construct and explore various topologies.

\subsection{Compiler-Driven NoCs}
Other prior works have used compiler techniques to optimize various facets of NoCs.
Some studies have explored statically allocating virtual channels~\cite{staticVC-isca, staticVC-nocs} to multiple concurrent flows to mitigate head-of-line blocking. These studies propose an approach to derive deadlock-free allocations based on the turn model~\cite{turnModel}. While our approach also statically allocates VCs, our method to guarantee deadlock freedom differs from the aforementioned study as it does not rely on the turn model.
Ozturk et al. \cite{ozturk2010compiler} propose a scheme to increase the reliability of NoCs for chip multiprocessors by sending packets over multiple links.
Their approach uses integer linear programming to balance the total number of links activated (an energy-based metric) against the amount of packet duplication (reliability).
Ababei et al. \cite{ababei2011energy} use a static placement algorithm and an estimate of reliability to attempt to guide placement decisions for NoCs.
Kapre et al. \cite{kapre2011noc} develop a workflow to map applications to CGRAs using several transformations, including efficient multicast routing and node splitting, but do not consider optimizations such as non-minimal routing.

%Our approach provides an extension of these techniques by using a ``closed-loop'' iterative process.
%Instead of doing each step (placement, routing, deadlock avoidance) sequentially, we are able to use heuristics to quantify the impact that each step has on the badness of the overall design.
%For example, this allows us to fix routing pathologies by re-placement because we can identify the specific placement decisions that led to the poor set of routes.



% \gist{
% CGRA: \\
% \cite{tartan} \\
% Imperative to Spatial Architectures: \\
% \cite{synaid}: 
% \begin{outline}
% \1 Target green array. Tiles of stack based 18-bit processors. Small local memory per
% core.
% \1 Small benchmarks: trigonometric, FFT, bithack, etc..
% \end{outline}
% \cite{zaidi}
% His thesis:\\
% \url{https://www.cl.cam.ac.uk/techreports/UCAM-CL-TR-870.pdf} \\
% Targeting FPGA. We probably don't want to cite this guy.

% Partitioning: \\
% \cite{nowatzki} - Partitioning data flow graph using ISL \\

% Summary:
% Looks like we are not the first one trying to map imperative language to spatial architecture. But there are still major differences.
% Some of the work is mapping to FPGAs, similar to high-level synthesis tools.
% Many of the other works like \cite{zaidi} only focus on the data-flow graph and do not handle memory accesses. Although we both trying to support imperative language on the spatial accelerator, our focus is never to accelerate control heavy code on the accelerator, but rather to target data-intensive application with more flexibility.

% \cite{synaid} is actually very relevant but the architecture is at a much smaller scale. It does have distributed on-chip memory but their memory is a stack used to store programs for processors. None of the prior work has considered using distributed memory to compose logically single memory with strong-consistency and coherence. Another major difference is that these architecture does not have a global network and DRAM that introduces variable latency. 
% So this changes how the application is mapped onto the accelerator. Our approach is a completely distributed streaming approach while they tend to use a statically scheduled approach.
% Finally the applications are very different. Most of their applications are very small kernels or image/audio encoding/decoding. Our benchmarks have a mix of full ML models + graph + others
% }

% Maybe related \cite{sparseaccel}

% Triggered Instruction \cite{ti}
% Tartan\cite{tartan}

% \gist{
%     \begin{itemize}
%         \item Plasticine compiler: \cite{plasticine}
%         \item Tartan: \cite{tartan} Single op partitions w/ handshaking, direct template translation.
%         \item Zaidi Thesis: \cite{zaidi} Conversion to bluespec, also uses token-based control. No partitioning / merging.
%         \item Partitioning: \cite{nowatzki} Solver-based partitioning, very small arrays (4x4 dyser, etc.), with statically known delays and single-op partitions. Does not handle merging.
%         \item SparseAccel: \cite{sparseaccel}\ Architecture paper, basically nothing about compilers.  (NR)
%         \item streamit for RAW: \cite{streamit} No memory consistency due to pure stream structure, fine-grained arch. Mesh of processors.
%         \item Triggered Instructions \cite{ti} Fine-grained processor architecture, with caches. (Micro paper: manual mapping, assembly, NR)
%         \item Chlorophyll: \cite{synaid} Maps subset of C to GreenArrays (small spatial stack-processor architecture). Distinct Non-distributed arrays vs distributed arrays, optional partition annotations, static for loops (unrestricted while), uses Rosette backend solver (branch and bound, SA, Ant, Tabu Search, picked Simulated Annealing). The actual generation is synthesis based.
%         \item Spatial computation \cite{spatial-computation} Maps ANSI C to Verilog, handshake + token-based execution with speculative execution. Uses a crossbar network for memory (although we assume pre-banked memory). Fine-grained control uses Pegasus for memory dependence graph and tokens for synchronization. Does not address partitioning/merging.
%     \end{itemize}
% }

  \chapter{Conclusions (WIP)}

We show that the best network design depends on both applications and the underlying accelerator architecture.
Network performance correlates strongly with bandwidth for streaming accelerators, and scaling raw bandwidth is more area- and energy-efficient with a static network.
We show that the application mapping can be optimized to move less data by using a dynamic network as a fallback from a high-bandwidth static network.
%this contributes a 6.9x average performance per area and 2.3x average energy-efficiency improvement for a static-dynamic hybrid network.
This static-dynamic hybrid network provides a 1.8x energy-efficiency and
2.8x performance advantage over the purely static and purely dynamic networks, respectively.
  %Furthermore, we show that spatial architectures require larger switches as programs get bigger, imposing super-linear scaling of network size with chip area.



  \begin{acks}
We thank Gedeon Nyengele and Kartik Prabhu for their assistance in simulation and data collection.
This paper is based on research partially supported by a Herbert Kunzel Stanford Graduate Fellowship.
This material is based on research sponsored by Air Force Research Laboratory (AFRL) and Defense
Advanced Research Projects Agency (DARPA) under agreement number FA8650-18-2-7865. The U.S.
Government is authorized to reproduce and distribute reprints for Governmental purposes
notwithstanding any copyright notation thereon. The views and conclusions contained herein are those
of the authors and should not be interpreted as necessarily representing the official policies or
endorsements, either expressed or implied, of Air Force Research Laboratory (AFRL) and Defense
Advanced Research Projects Agency (DARPA) or the U.S. Government.
This research is also supported in part by affiliate members and other supporters of the
Stanford DAWN project: Ant Financial, Facebook, Google, Infosys, Intel, Microsoft, NEC, Teradata,
SAP and VMware.
\end{acks}


  \bibliographystyle{sysml2019}
  \bibliography{references}
  %%%%%%%%%%%%%%%%%%%%%%%%%%%%%%%%%%%%%%%%%%%%%%%%%%%%%%%%%%%%%%%%%%%%%%%%%%%%%%%
  %%%%%%%%%%%%%%%%%%%%%%%%%%%%%%%%%%%%%%%%%%%%%%%%%%%%%%%%%%%%%%%%%%%%%%%%%%%%%%%
\end{document}
