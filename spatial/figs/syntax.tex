
\begin{table*}
\centering
\caption{A subset of Spatial's syntax.
%An overview of Spatial's syntax for host interfaces, control structures, scheduling directives, memory templates, streaming interfaces, and design space parameters. 
Square brackets (e.g. \texttt{[T]}) represent a template's type parameter. Parameters followed by a `\texttt{+}' denote arguments which can be given one or more times, while a `\texttt{*}' denotes that an argument is optional. 
\texttt{DRAMs}, \texttt{Foreach}, \texttt{Reduce}, and \texttt{MemReduce} can all have arbitrary dimensions. 
%\texttt{DRAMs} can be allocated with an arbitrary number of dimensions. \texttt{Foreach}, \texttt{Reduce}, and \texttt{MemReduce} support multi-dimensional iteration domains. 
}
\label{t:syntaxTable}


\newsavebox{\counter}
\begin{lrbox}{\counter}
\begin{lstlisting}[language=SpatialTable]
min* until max by stride* par factor*
\end{lstlisting}
\end{lrbox}

\newsavebox{\fsmSignature}
\begin{lrbox}{\fsmSignature}
\begin{lstlisting}[language=SpatialTable]
FSM(init){continue}{action}{next}
\end{lstlisting}
\end{lrbox}

\newsavebox{\foreachSignature}
\begin{lrbox}{\foreachSignature}
\begin{lstlisting}[language=SpatialTable]
Foreach(counter+){body}
\end{lstlisting}
\end{lrbox}

\newsavebox{\reduceSignature}
\begin{lrbox}{\reduceSignature}
\begin{lstlisting}[language=SpatialTable]
Reduce(accum)(counter+){func}{reduce}
\end{lstlisting}
\end{lrbox}

\newsavebox{\memreduceSignature}
\begin{lrbox}{\memreduceSignature}
\begin{lstlisting}[language=SpatialTable]
MemReduce(accum)(counter+){func}{reduce}
\end{lstlisting}
\end{lrbox}

\newsavebox{\streamStar}
\begin{lrbox}{\streamStar}
\begin{lstlisting}[language=SpatialTable]
Stream(*){body}
\end{lstlisting}
\end{lrbox}

\newsavebox{\parallelSignature}
\begin{lrbox}{\parallelSignature}
\begin{lstlisting}[language=SpatialTable]
Parallel{body}
\end{lstlisting}
\end{lrbox}

\newsavebox{\pipeSignature}
\begin{lrbox}{\pipeSignature}
\begin{lstlisting}[language=SpatialTable]
DummyPipe{body}
\end{lstlisting}
\end{lrbox}

\newsavebox{\ifSignature}
\begin{lrbox}{\ifSignature}
\begin{lstlisting}[language=SpatialTable]
if (cond){body} 
[else if (cond){body} ] 
[else {body} ]
\end{lstlisting}
\end{lrbox}

\newsavebox{\sequentialTag}
\begin{lrbox}{\sequentialTag}
\begin{lstlisting}[language=SpatialTable]
Sequential.(Foreach|Reduce|MemReduce)
\end{lstlisting}
\end{lrbox}

\newsavebox{\pipeTag}
\begin{lrbox}{\pipeTag}
\begin{lstlisting}[language=SpatialTable]
Pipe(ii*).(Foreach|Reduce|MemReduce)
\end{lstlisting}
\end{lrbox}

\newsavebox{\streamTag}
\begin{lrbox}{\streamTag}
\begin{lstlisting}[language=SpatialTable]
Stream.(Foreach|Reduce|MemReduce)
\end{lstlisting}
\end{lrbox}

\newsavebox{\parallelTag}
\begin{lrbox}{\parallelTag}
\begin{lstlisting}[language=SpatialTable]
Parallel.(Foreach|Reduce|MemReduce)
\end{lstlisting}
\end{lrbox}

\newsavebox{\fifoSyntax}
\begin{lrbox}{\fifoSyntax}
\begin{lstlisting}[language=SpatialTable]
FIFO[T](depth)
\end{lstlisting}
\end{lrbox}

\newsavebox{\filoSyntax}
\begin{lrbox}{\filoSyntax}
\begin{lstlisting}[language=SpatialTable]
LIFO[T](depth)
\end{lstlisting}
\end{lrbox}

\newsavebox{\lineBufferSyntax}
\begin{lrbox}{\lineBufferSyntax}
\begin{lstlisting}[language=SpatialTable]
LineBuffer[T](r, c)
\end{lstlisting}
\end{lrbox}

\newsavebox{\lutSyntax}
\begin{lrbox}{\lutSyntax}
\begin{lstlisting}[language=SpatialTable]
LUT[T](dims+)(elements+)
\end{lstlisting}
\end{lrbox}

\newsavebox{\regSyntax}
\begin{lrbox}{\regSyntax}
\begin{lstlisting}[language=SpatialTable]
Reg[T](reset*)
\end{lstlisting}
\end{lrbox}

\newsavebox{\regfileSyntax}
\begin{lrbox}{\regfileSyntax}
\begin{lstlisting}[language=SpatialTable]
RegFile[T](dims+)
\end{lstlisting}
\end{lrbox}

\newsavebox{\sramSyntax}
\begin{lrbox}{\sramSyntax}
\begin{lstlisting}[language=SpatialTable]
SRAM[T](dims+)
\end{lstlisting}
\end{lrbox}

\newsavebox{\argInSyntax}
\begin{lrbox}{\argInSyntax}
\begin{lstlisting}[language=SpatialTable]
ArgIn[T]
\end{lstlisting}
\end{lrbox}

\newsavebox{\argOutSyntax}
\begin{lrbox}{\argOutSyntax}
\begin{lstlisting}[language=SpatialTable]
ArgOut[T]
\end{lstlisting}
\end{lrbox}

\newsavebox{\hostIOSyntax}
\begin{lrbox}{\hostIOSyntax}
\begin{lstlisting}[language=SpatialTable]
HostIO[T]
\end{lstlisting}
\end{lrbox}

\newsavebox{\dramSyntax}
\begin{lrbox}{\dramSyntax}
\begin{lstlisting}[language=SpatialTable]
DRAM[T](dims+)
\end{lstlisting}
\end{lrbox}

\newsavebox{\streamInSyntax}
\begin{lrbox}{\streamInSyntax}
\begin{lstlisting}[language=SpatialTable]
StreamIn[T](bus)
\end{lstlisting}
\end{lrbox}

\newsavebox{\streamOutSyntax}
\begin{lrbox}{\streamOutSyntax}
\begin{lstlisting}[language=SpatialTable]
StreamOut[T](bus)
\end{lstlisting}
\end{lrbox}

\newsavebox{\parameterSyntax}
\begin{lrbox}{\parameterSyntax}
\begin{lstlisting}[language=SpatialTable]
default (min,max)
default (min,stride,max)
\end{lstlisting}
\end{lrbox}

\newsavebox{\accelSyntax}
\begin{lrbox}{\accelSyntax}
\begin{lstlisting}[language=SpatialTable]
Accel{body}
\end{lstlisting}
\end{lrbox}

\newsavebox{\accelStarSyntax}
\begin{lrbox}{\accelStarSyntax}
\begin{lstlisting}[language=SpatialTable]
Accel(*){body}
\end{lstlisting}
\end{lrbox}

\fontsize{8}{10}
\selectfont
\begin{tabular}{lll}
%\toprule
\begin{tabular}{l}

\multicolumn{1}{l}{\bf{(a) Control Structures}}  \\
\midrule 

\multirow{1}{*}{\usebox{\counter}} \\
A counter over [\argg{min},\argg{max}) ([0,\argg{max}) if \argg{min} is unspecified).  \\
~~~~~\argg{stride}: optional counter stride, default is 1 \\
~~~~~\argg{factor}: optional counter parallelization, default is 1 \\
%& \\
\vspace{-9pt}\\

% \multirow{3}{*}{\usebox{\ifSignature}} \\
% \\
% \vspace{-3pt}\\
% Data-dependent execution. \\
% Doubles as a multiplexer if all bodies return scalar values. \\
% ~~~~~\textbf{\fontsize{8}{\texttt{cond}}}: condition for execution of associated body \\
% ~~~~~\textbf{\fontsize{8}{\texttt{body}}}: arbitrary expression \\
% \vspace{-9pt} \\

\multirow{1}{*}{\usebox{\fsmSignature}} \\ %& \multirow{6}{*}{\usebox{\fsmExample}} \\
%& An arbitrary state machine with an internal state of type \texttt{T}. \\ 
An arbitrary finite state machine, similar to a \emph{while} loop. \\
~~~~~\argg{init}: the FSM's initial state \\
~~~~~\argg{continue}: the ``while'' condition for the FSM \\
~~~~~\argg{action}: arbitrary expression, executed each iteration \\
~~~~~\argg{next}: function calculating the next state \\

\vspace{-9pt}\\

\multirow{1}{*}{\usebox{\foreachSignature}} \\ %& \multirow{6}{*}{\usebox{\fsmExample}} \\
%& An arbitrary state machine with an internal state of type \texttt{T}. \\ 
A parallelizable \emph{for} loop. \\
~~~~~\argg{counter}: counter(s) defining the loop's iteration domain \\
~~~~~\argg{body}: arbitrary expression, executed each loop iteration \\
\vspace{-9pt}\\

\multirow{1}{*}{\usebox{\reduceSignature}} \\
A scalar reduction loop, parallelized as a tree. \\
~~~~~\argg{accum}: the reduction's accumulator register \\
~~~~~\argg{counter}: counter(s) defining the loop's iteration domain \\
~~~~~\argg{func}: arbitrary expression which produces a scalar value \\
~~~~~\argg{reduce}: associative reduction between two scalar values \\
\vspace{-9pt}\\

\multirow{1}{*}{\usebox{\memreduceSignature}} \\
Reduction over addressable memories. \\
~~~~~\argg{accum}: an addressable, on-chip memory for accumulation \\
~~~~~\argg{counter}: counter(s) defining the loop's iteration domain \\
~~~~~\argg{func}: arbitrary expression returning an on-chip memory \\
~~~~~\argg{reduce}: associative reduction between two scalar values \\
\vspace{-9pt}\\


\multirow{1}{*}{\usebox{\streamStar}} \\
A streaming loop which never terminates. \\
~~~~~\argg{body}: arbitrary expression, executed each loop iteration \\
\vspace{-9pt}\\

\multirow{1}{*}{\usebox{\parallelSignature}} \\
Overrides normal compiler scheduling. All statements \\
in the body are instead scheduled in a \emph{fork-join} fashion. \\
~~~~~\argg{body}: arbitrary sequence of controllers \\
\vspace{-9pt}\\

\multirow{1}{*}{\usebox{\pipeSignature}} \\
A ``loop'' with exactly one iteration. \\
Inserted by the compiler, generally not written explicitly. \\
~~~~~\argg{body}: arbitrary expression \\
\end{tabular} & 

\begin{tabular}{l}
\multicolumn{1}{l}{\bf{(b) Optional Scheduling Directives}}  \\
\midrule 

\multirow{1}{*}{\usebox{\sequentialTag}} \\
Sets loop to run sequentially. \\
\vspace{-10pt}\\

\multirow{1}{*}{\usebox{\pipeTag}} \\
Sets loop to be pipelined. \\
~~~~~\argg{ii}: optional overriding initiation interval \\
\vspace{-10pt}\\

\multirow{1}{*}{\usebox{\streamTag}} \\
Sets loop to be streaming. \\
% \vspace{-10pt}\\

% \multirow{1}{*}{\usebox{\parallelTag}} \\
% Informs the compiler that the loop is parallelizable. \\
\\

% \multicolumn{1}{l}{\bf{(d) On-Chip Memories}}  \\
% \midrule

% \multirow{1}{*}{\usebox{\fifoSyntax}} \\
% FIFO (queue) with a capacity of \textbf{\fontsize{8}{\texttt{depth}}} elements of type \textbf{\fontsize{8}{\texttt{T}}} \\ 
% \vspace{-10pt}\\

% \multirow{1}{*}{\usebox{\filoSyntax}} \\
% A LIFO (stack) with a capacity of \textbf{\fontsize{8}{\texttt{depth}}} elements of type \textbf{\fontsize{8}{\texttt{T}}} \\
% \vspace{-10pt}\\ 

% \multirow{1}{*}{\usebox{\lineBufferSyntax}} \\
% On-chip buffered scratchpad containing \textbf{\fontsize{8}{\texttt{r}}} buffers of \textbf{\fontsize{8}{\texttt{c}}} elements \\ 
% \vspace{-10pt}\\

% \multirow{1}{*}{\usebox{\lutSyntax}} \\
% Read-only Lookup Table containing supplied \textbf{\fontsize{8}{\texttt{elements}}} of type \textbf{\fontsize{8}{\texttt{T}}} \\ 
% \vspace{-10pt}\\

% \multirow{1}{*}{\usebox{\regSyntax}} \\
% Register holding a value of type \textbf{\fontsize{8}{\texttt{T}}}, with optional \textbf{\fontsize{8}{\texttt{reset}}} value \\ 
% \vspace{-10pt}\\

% \multirow{1}{*}{\usebox{\regfileSyntax}} \\
% Register file of elements of type \textbf{\fontsize{8}{\texttt{T}}} with given dimensions\\ 
% \vspace{-10pt}\\

% \multirow{1}{*}{\usebox{\sramSyntax}} \\
% On-chip scratchpad of elements of type \textbf{\fontsize{8}{\texttt{T}}} with given dimensions\\ 
% \\


\multicolumn{1}{l}{\bf{(c) Shared Host/Accelerator Memories}}  \\
\midrule

\multirow{1}{*}{\usebox{\argInSyntax}} \\
Accelerator register initialized by the host \\
\vspace{-10pt}\\

\multirow{1}{*}{\usebox{\argOutSyntax}} \\
Accelerator register visible to host after accelerator execution \\
\vspace{-10pt}\\

\multirow{1}{*}{\usebox{\hostIOSyntax}} \\
Accelerator register the host may read and write at any time. \\
\vspace{-10pt}\\

\multirow{1}{*}{\usebox{\dramSyntax}} \\
Burst-addressable, host-allocated off-chip memory. \\
%Memory is accessible by both the accelerator and the host. \\ 
\\


\multicolumn{1}{l}{\bf{(d) External Interfaces}}  \\
\midrule

\multirow{1}{*}{\usebox{\streamInSyntax}} \\
Streaming input from a \argg{bus} of external pins. \\ 
\vspace{-10pt}\\

\multirow{1}{*}{\usebox{\streamOutSyntax}} \\
Streaming output to a \argg{bus} of external pins. \\ 
\\

\multicolumn{1}{l}{\bf{(e) Host Interfaces}}  \\
\midrule 
\multirow{1}{*}{\usebox{\accelSyntax}} \\
A blocking accelerator design. \\
\vspace{-10pt} \\

\multirow{1}{*}{\usebox{\accelStarSyntax}} \\
A non-blocking accelerator design. \\
\\


\multicolumn{1}{l}{\bf{(f) Design Space Parameters}}  \\
\midrule 
\multirow{2}{*}{\usebox{\parameterSyntax}} \\
\\
A compiler-aware design parameter with given \argg{default} value. \\
DSE explores the range [\argg{min}, \argg{max}] with optional \argg{stride}. \\

\end{tabular} \\

\end{tabular}

\vspace{-5pt}
\end{table*}

