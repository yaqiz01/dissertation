\chapter{Introduction (WIP)}

\section{The Rising of Hardware Acceleration}

With the end of Dennard Scaling~\cite{dennard}, the amount of performance one can extract from a CPU is reaching a limit.
To provide general-purpose flexibility, CPU spends the majority of resources and energy on overheads, 
including dynamic-instruction fetching and decoding, branch prediction, and a cache hierarchy, 
with less than 20\% of 
the energy on the actual computation~\cite{mark}.
Even worse, power wall is limiting the entire multicore family
to reach the doubled performance per generation enabled by technology scaling in the 
past~\cite{multicorescale}.

For this reason, hardware acceleration is emerging in various compute-intensive application domains 
to provide orders of magnitude acceleration, enabling algorithms that were otherwise
infeasible~\cite{genomicaccel, bioaccel, fpgadeeplearn, fpgacripto}.
Examples include widely adopted General-Purpose Graphics Processing Units (GPGPUs) 
in computational genomics, signal procesing, graph processing, and
deep learning, etc.~\cite{genomicaccel, bioaccel, fpgacloudsurvey}.
Moreover, many recent efforts are spent on leveraging application domain knowledge in hardware design to enable 
continued performance scaling while meeting the power budget~\cite{turinglecture}.
As artificial intelegence receiving great success in industry and business,
past years have seen an explosion in maching learning accelerators;
these accelerators contain specialized circuits for ML kernels that dramatically improves the compute
efficiency~\cite{dadiannao,tpu,eie,chen2017eyeriss,tangram,truenorth}.

Nonetheless, it is non-trivial to achieve a good utilization of these accelerators.
While the peak FLOPS of GPU has increased by over 20x in the past 10 years, the achievable FLOPS is
not increasing accordingly~\cite{floptrend, gpuperfana}.
The massive threads in GPU require embarrassedly parallel workloads to fully saturate the compute
throughput. GPU's bulk-synchronous nature also causes poor cache efficiency; data are spilled
off-chip before getting reused~\cite{gpuinefficiency}.
On the outer hand, 
while extremely efficient for certain models, machine learning accelerators are highly
specialized for specific kernels, especially general matrix multiply (GEMM) and convolution.
However, ML algorithms evolve much faster than the development and manufacture cycle of hardware, 
leading to inefficiency in or even unsupported by these accelerators.
For example, hybrid models, such as Mask R-CNN~\cite{maskrcnn} and DeepLab~\cite{deeplab}, include 
combinations of GEMM-compatible and incompatible operations, both computationally expensive.
Fix-functional accelerators focusing on GEMM operations, such as TPU~\cite{tpu}, 
have to rely on CPUs for unsupported operations or
convert non-GEMM operations to GEMM operations, resulting in a severe performance gap between the
peak and effective FLOPS~\cite{effflexdnnaccel}.

Reconfigurable spatial architectures overcome this limitation by changing its datapath based on applications' needs.
Applications are configured at the circuit-level without dynamic instruction fetching and decoding,
hence improving energy-efficiency~\cite{calhoun,fpgaPower}.
In addition to instruction, data, and task-level parallelism explored by processor architectures, spatial architectures also explore instruction and task-level
pipelining that further increase the compute throughput~\cite{spatial-computation}.
Exploring pipeline parallelism enable spatial architectures to achieve a high-throughput
without massively parallelize every program stage.
Flexible datapath also permits resource distribution proportional to the compute intensity of
program stages.
A key performance optimization on reconfigurable accelerators is an application-level design space
exploration 
searching for the best resource distribution scheme that balances the compute pipeline~\cite{dse_koeplinger}.

One of the mainstream reconfigurable spatial architecture is 
Field Programmable Gate Arrays (FPGAs) that support fine-grain, 
bit-level reconfigurability with a soft logic fabric~\cite{fpga-survey}.
The flexible interconnect and lookup table-based logic gate can be configured to implement arbitrary
datapath.
FPGAs have been used to deploy services commercially~\cite{microsoft, baidu, deephi} and can be rented on the AWS F1 cloud~\cite{aws}. 
Although around for a long time, FPGAs are not broadly accepted among high-level application programmers 
due to their low-level programming interface and long compilation time.
Fine tuning applications on FPGAs requires expertise in digital design and takes a long
development cycle, which hinders their accessibility to the general software community.
As an application-level accelerator, FPGAs also suffers from overhead from fine-grained reconfigurability; 
studies have shown that over 60\% of chip size of a FPGA are spent on routing 
resources~\cite{fpgaSurvey, calhoun, fpgaPower}. 

In contrast to fine-grained reconfigurable architectures,
Coarse-Grained Reconfigurable Arrays (CGRAs) are spatial architectures with 
coarse-grained building blocks, such as ALUs, register files, and memory controllers, 
distributed in a programmable, word-level static interconnect~\cite{adres, kress, dyser, piperench, tartan, 
hrl, hycube}.

Lately, Reconfigurable Dataflow Accelerators (RDAs)~\cite{plasticine, ti, streamdataflow,
neuflow,cnndataflow,dataflowarch} are emerging as a new class of spatial accelerators that 
retain the desired level of flexibility and energy efficiency without 
the area overhead and low clock frequency due to bit-level reconfigurability.

Plasticine is a recently proposed RDA with high resource density and compute throughput.
Recent studies~\cite{plasticine} have demonstrated a promising acceleration of dense, sparse
, and streaming applications using RDAs.
%, operating at a fixed and high clock frequency at large chip sizes.


\section{The Need of Flexible Interconnects}
Applications are mapped to RDAs by distributing computations spatially across multiple processing blocks 
and executing them in a pipelined, data-driven fashion. On traditional Networks on Chip (NoCs), communication 
is the result of explicit message passing by parallel workers or cache misses; these are bursty and
relatively infrequent. On RDAs, however, applications are distributed by parallelizing and pipelining; 
pipelining introduces frequent and throughput-sensitive communication. Because different applications are 
parallelized and pipelined differently, they have different communication requirements.

RDAs need the right amount of interconnect flexibility to achieve a good resource utilization; 
an inflexible interconnect constrains the space of
valid application mappings and hinders resource utilization. 
Furthermore, 
in the quest to increase compute density, RDA data paths now 
contain increasingly coarse-grained processing blocks such as pipelined, vectorized functional 
units~\cite{plasticine, piperench, xilinx-acap}.

Plasticine, as an example, has a 512-bit vector bus, which necessitates coarser communication 
and higher on-chip interconnect bandwidth to avoid creating performance bottlenecks. 
Although many hardware accelerators with large, vectorized data paths have fixed local networks~\cite{brainwave}, there is a need for more
flexible global networks to adapt to future applications.
Consequently, interconnect design for these RDAs involves achieving a balance between the often conflicting requirements of high bandwidth and high flexibility.

\section{The Gap between High-Level DSLs and Dataflow Accelerators}
Recent years has seen a raise in interests in researching in hardware accelerators, primarily
motivated by AI applications. 
On one hand, huge success of adoption of AI in various domain has motivated various machine learning
specialized hardware in the architecture community.
On the other hand, research in machine learning algorithms has been expedited by the increasing
powerful hardware accelerators, enabling the next-level algorithms that was not feasible in just
decades ago.
Research in software infrastructure for these accelerators, however, are just emerging.
Unlike CPUs, these accelerators do not support a standard ISAs, alleviating the overhead from
layers of abstractions and the burden of backward compatibility. 
Nonetheless, lack of a common abstraction makes it very hard to sharing compiler infrastructure
across accelerators. 

\section{Contribution}
\section{Outline}
