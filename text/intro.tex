\chapter{Introduction (WIP)}

\section{The Rising of Hardware Acceleration}

With the end of Dennard Scaling~\cite{dennard}, the amount of performance one can extract from a CPU is reaching a limit.
To provide general-purpose flexibility, CPU spends the majority of energy on overheads, including dynamic-instruction execution, branch prediction, and a cache hierarchy, and less than 20\% of the energy on the actual computation~\cite{mark}.
Even worse, the power wall is limiting the entire multicore family
to reach the doubled performance improvement per generation enabled by technology scaling in the past\cite{multicorescale}.

For this reason, many recent efforts are spent on leveraging application domain knowledge in hardware design to enable 
continued performance scaling while meeting the power budget\cite{turinglecture}.
Examples include widely adopted General-Purpose Graphics Processing Units (GPGPUs) in the deep-learning domain and machine learning (ML) accelerators, such as Tensor Processing Units~\cite{tpu} and EIE~\cite{eie}, providing orders of magnitude acceleration over a CPU.
However, massive threads in GPU and the highly specialized datapath in ML accelerators often cause severe
underutilization of the hardware due to variation in application and data characteristics\cite{tz_rnn}.

Reconfigurable spatial architectures overcome this limitation by changing its datapath based on applications' needs.
Applications are configured at the circuit-level without dynamic instruction fetching and decoding, hence improving energy-efficiency.
In addition to instruction, data, and task-level parallelism explored by processor architectures, spatial architectures also explore instruction and task-level
pipelining that further increase the compute throughput.
Pipelining at various granularity enables spatial architectures to achieve high throughput on program phases without massive parallel workload, and allocate resource proportional to compute intensities of program phases.

One example of a reconfigurable spatial architecture is Field Programmable Gate Arrays (FPGAs) that support fine-grain, 
bit-level reconfiguration with a soft logic fabric~\cite{fpga-survey}.
The flexible interconnect and lookup table-based logic gate can be configured to implement arbitrary
datapath.
FPGAs are used to deploy services commercially~\cite{microsoft, baidu, deephi} and can be rented on the AWS F1 cloud~\cite{aws}. 
Although around for a long time, FPGAs are not broadly used on high-level applications due to their low-level 
programming interface and long compilation time.
Fine tuning applications on FPGAs requires expertise in digital design knowledge and takes a long
development cycle, which hinders their accessibility to general software programmers.
As an application-level accelerator, FPGAs also suffers from overhead incurred by fine-grained reconfigurability; 
studies have shown that an FPGA can have over 
have hampered widespread adoption for several 
years~\cite{bolsens, calhoun, fpgaPower, fpgaSurvey}. 

Lately, Reconfigurable Dataflow Accelerators (RDAs)~\cite{plasticine, ti} are emerging as a new class of spatial accelerators that retain the desired level of flexibility and energy efficiency without the fine-grained reconfigurability overhead.
RDA provides high resource density and compute throughput with a hierarchical streaming network, operating at a fixed and high clock frequency at large chip sizes.
Recent studies~\cite{plasticine} have demonstrated a promising acceleration of dense, sparse, and streaming applications using RDAs.

Spatially reconfigurable architectures are programmable, energy efficient application accelerators offering the flexibility of software and the efficiency of hardware.
Architectures such as Field Programmable Gate Arrays (FPGAs) achieve energy efficiency by providing statically reconfigurable compute elements and on-chip memories in a bit-level programmable interconnect; this interconnect can be configured to implement arbitrary datapaths. 
FPGAs are used to deploy services commercially~\cite{microsoft, baidu, deephi}
and can be rented on the AWS F1 cloud~\cite{aws}. 
However, FPGAs suffer from overhead incurred by fine-grained reconfigurability; their long compile times
and relatively low compute density have hampered widespread adoption for several years~\cite{bolsens, calhoun, fpgaPower, fpgaSurvey}. 
Therefore, recent spatial architectures use
increasingly coarse-grained building blocks, such as ALUs, register files, and memory controllers, distributed in a programmable, word-level static interconnect.
Several of these Coarse-Grained Reconfigurable Arrays (CGRAs) have recently been proposed \cite{adres, kress, dyser, piperench, tartan, hrl, ti, hycube, plasticine}.

\section{The Need of Flexible Interconnects}
CGRAs need the right amount of interconnect flexibility to achieve high resource utilization; an inflexible interconnect constrains the space of
valid application mappings and hinders resource utilization. Furthermore, in the quest to increase compute density, CGRA data paths now 
contain increasingly coarse-grained processing blocks such as pipelined, vectorized functional units~\cite{plasticine, piperench, xilinx-acap}.
These data paths typically have a vector width of 8--16x~\cite{plasticine}, which necessitates coarser communication and higher on-chip interconnect bandwidth to avoid
creating performance bottlenecks. 
Although many hardware accelerators with large, vectorized data paths have fixed local networks~\cite{brainwave}, there is a need for more
flexible global networks to adapt to future applications.
Consequently, interconnect design for these CGRAs involves achieving a balance between the often conflicting requirements of high bandwidth and high flexibility.

\section{The Gap between High-Level DSLs and Dataflow Accelerators}
Recent years has seen a raise in interests in researching in hardware accelerators, primarily
motivated by AI applications. 
On one hand, huge success of adoption of AI in various domain has motivated various machine learning
specialized hardware in the architecture community.
On the other hand, research in machine learning algorithms has been expedited by the increasing
powerful hardware accelerators, enabling the next-level algorithms that was not feasible in just
decades ago.
Research in software infrastructure for these accelerators, however, are just emerging.
Unlike CPUs, these accelerators do not support a standard ISAs, alleviating the overhead from
layers of abstractions and the burden of backward compatibility. 
Nonetheless, lack of a common abstraction makes it very hard to sharing compiler infrastructure
across accelerators. 

\section{Contribution}
\section{Outline}
