\section{Resource Allocation} \label{sec:resalloc}

The output of the imperative to dataflow transformation discussed in \Cref{sec:control} is a VUDFG that 
can execute on a Plasticine with physical units (PUs) that have infinite resources.
The \emph{Resource Allocation} phase enforces and addresses constraint violations given 
the specification of the Plasticine units. 
At the end of this phase, \name assigns each VU in the VUDFG graph to a PU type with required
resources; the placer then takes the type assignments and determines the final placement.

Accelerators often have heterogeneity in compute resources to improve efficiency for common
special operations.
In Plasticine, PMUs and AGs have specialized compute pipelines for address calculation that are 
less capable than the compute pipeline in PCUs.
However, heterogeneity tends to reduce average utilization because different applications, and even the same application with different data sizes, can vary highly in the desired ratio among different
resource~\cite{tz_rnn}.
A compute-bound application, for example, can heavily underutilize the AGs and PMUs.
To address this problem, \name models the virtual to physical assignment as a constraint satisfaction problem; 
each VU consumes a set of resources and can only be assigned to a PU if the PU processes the required resources. 
%Instead of using heuristics to assign certain a type of VU to a type of PU, we
\Cref{tab:resource} shows the types of resources \name models in Plasticine's heterogeneous units.
For example, special connection to off-chip memory interface is
also treated as a type of resource in the AG, which forces virtual contexts accessing DRAM to map to AGs. 
On the other side, regular contexts with non-vectorized fixed-point operations can also be mapped to
spare AGs, which improves utilization.
\begin{table*}
  \centering
\begin{tabular}{lccccc}
  \toprule
  Feature & PCU & PMU & AG & Host Unit & Aggregation Function\\ \midrule
  Vector lane width & 16 & 16 & 1 & 1 & \multirow{2}{*}{MAX}\\
  \# pipeline register (PR) & 8 & 8 & 4 & 0 & \\ \hline
  \# stages & 6 & 10 & 5 & 0 & \multirow{6}{*}{SUM}\\
  Scratchpad banks & 0 & 16 & 0 & 0 &  \\
  Scratchpad capacity & 0 & 256kB & 0 & 0 & \\
  MergeBuffer & 1 & 0 & 0 & 0 & \\
  Splitter & 1 & 0 & 0 & 0 & \\
  Scanner & 1 & 0 & 0 & 0 & \\ \hline
  Operation types & fix $\cup$ float & fix & fix & $\varnothing$ & $\cup$ \\ \hline
  Reduction tree & \cmark & \xmark & \xmark & \xmark & \multirow{3}{*}{OR}\\
  Access to DRAM Interface & \xmark & \xmark & \cmark & \xmark & \\
  Access to Host IO & \xmark & \xmark & \xmark & \cmark & \\ \hline
  \# Vector Input & 6 & 6 & 4 & 0 & \multirow{6}{*}{G}\\
  \# Scalar Inputs & 6 & 6 & 4 & 16 & \\
  \# Control Inputs & 16 & 16 & 4 & 16& \\
  \# Vector Outputs & 6 & 6 & 4 & 0 & \\
  \# Scalar Outputs & 6 & 6 & 4 & 16 & \\
  \# Control Outputs & 8 & 8 & 2 & 16 & \\
 \bottomrule
\end{tabular}
\caption[Resources specification of heterogeneous units and aggregation function]{
  A list of resources \name models in four types of configurable units in Plasticine. The host unit
  models the host registers I/Os.
  MergeBuffer, Splitter, and Scanner are new hardware units introduced in \cite{gorgon} and recent work
  to support database and sparsity in Plasticine.
  The aggregation function indicates how to compute the aggregated resource cost when two contexts are merged
  into a single virtual unit (VU). G indicates the aggreated value is the output of a graph traversal of the
  merged graph. How to count \# I/O is discussed later in \Cref{sec:compsplit}.
  While the aggreation function of \#PR of two merged contexts is MAX, the \#PR of a
  context is the maximum number of live variables of its dataflow graph, which is also an output of
  a topological traversal.
  This table reflects a different Plasticine configuration as the original Plasticine in \cite{plasticine}.
}
\label{tab:resource}
\end{table*}

\begin{algorithm}
  \Fn(\tcc*[h]{Allocation Algorithm}){alloc(V, P, pruners)}{
    \KwData{V: a set of VUs from the VUDFG}
    \KwData{P: a set of all PUs on the hardware}
    \KwData{pruners: a list of constraint pruners to check
    and fixes constraint violations}
    \tcc{Initialize a complete bipartite graph}
    G = \KwNew BipartiteGraph()\;
    G[V] = P\;
    \tcc{Constraint resolution}
    prune(G, pruners)\;
    \tcc{Global merging}
    merge(G)\;
    \tcc{Heuristic check on whether assigning all VUs in V is feasible}
    check(G)\;
    \tcc{Virtual to physical assignment}
    backtracking\_assign(G)\;
  }
  \vspace{0.5cm}
  \Fn(\tcc*[h]{A recursive pruning function}){prune(G, pruners)}{
    \KwData{G: bipartite graph between VUs and PUs}
    \KwData{pruners: a list of constraint pruners to check
    and fixes constraint violations}
    \KwResult{The function update G by removing VU-PU edges that violates constraints guarded by
    pruners. The function may fail and raise an exception.}
    \tcc{All PUs on the hardware}
    P = G.values()\;
    \For{pruner \KwTo pruners}{
      \For{v \KwTo G.keys()}{
        \For{p \KwTo G[v]} {
          \If{pruner.cost(v) > pruner.cost(p)} {
            G[v] -= p\;
          }
        }
        \If{G[v].empty()} {
          \tcc{Partition VU v based on resource constraints registered in pruner. 
          Not all resources can be partitioned and this step may fail.
          If succeeded, the function returns a new set of VUs.}
          V' = pruner.partition(v)\;
          G' = \KwNew BipartiteGraph()\;
          G'[V'] = P\;
          prune(G',pruners)\;
          G -= v\;
          G[V'] = G'[V']\;
        }
      }
    }
  }
  \caption{Resource allocation. The bipartite graph \texttt{G} contains a bi-directional many-to-many
  map. \texttt{G[key]} returns the set of values connecting to the key (dom(key)), and \texttt{G[value]}
  returns the set of keys connecting to the value.
  \texttt{G[key] = value} connects an edge between key and value.
  \texttt{G[KeySet] = ValueSet} creates all-to-all connection between \texttt{KeySet} and
  \texttt{ValueSet}.}
  \label{algo:resalloc}
\end{algorithm}

\begin{algorithm}
  \Fn(\tcc*[h]{Assignment feasibility check}){check(G)}{
    \KwData{G: bipartite graph}
    \KwResult{Whether it is possible to assign all VUs in V with a different PU in P}
    \tcc{For every value set in \texttt{G}}
    \For{V \KwTo G.values().toSet()} {
      K = $\varnothing$\;
      \For{v \KwTo V} {
        \For{k \KwTo G[v]} {
          \If{G[k] $\subset$ V} {
            K += k\;
          }
        }
      }
      \If{|K| > |V|} {
        \KwRet{failure()}\;
      }
    }
    \KwRet{success()}\;
  }
  \caption{Heuristic check on whether it is possible to assign all key with an value in a bipartite
  graph. Given there are only a few types of hardware tiles, $G.values().toSet()$ is
  relatively small. This algorithm roughly runs in $O(|G.keys()|\times|G.values()|)$, which is
  still much faster than the backtracking assignment with exponential runtime.}
  \label{algo:check}
\end{algorithm}
 
%% backtracking_assignment(vu, dom)
As shown in \Cref{algo:resalloc}, the \emph{resource allocation} phase contains three steps:
\emph{constraint resolution}, \emph{global merging}, and \emph{virtual to physical assignment}.
\name uses a VU-PU bipartite graph (\emph{G}) to keep track of potential valid assignments between the two.
Initially, \emph{G} is initialized to a complete bipartite graph, i.e., all VUs can be assigned to
all PUs.
We refer to all PUs connected to a VU \emph{v} as the domain of \emph{v} in G, i.e. \emph{dom(v)}.

\paragraph{Constraint Resolution}
A list of constraint pruners, each considering a set of on-chip resources, 
incrementally remove the VU-PU edges that violate the resource constraints.
If a VU \emph{v} has an empty domain after pruning, the pruner attempts to fix the violation by
decomposing the VU into multiple VUs. 
Not all resources are composable, and the partitioning transformation may fail.
If succeeded, the partitioner generates a new set of VUs \emph{V'}. \name starts a new complete bipartite
graph between \emph{V'} and all resources \emph{P}, and recursively prune on \emph{V'}.
If succeeded, the original graph \emph{G} is updated with \emph{V'} and their pruned resources.

\paragraph{Global Merging}
After all VUs have at least one PU in the bipartite graph, \name triggers a global optimization that merges 
small VUs into a larger VU to reduce fragmentation in allocation.
Each type of resource has an aggregation rule to compute how the resource cost changes if two VUs are merged
together, as shown in \Cref{tab:resource}. 
Most aggregation rules are simple, such as addition, logical or, max, or union.
The in- and out-degree costs are tricker and will be detailed in \Cref{sec:compsplit}.

\paragraph{Virtual to Physical Assignment}
Next, \name performs a quick heuristic check on the bipartite graph to see if there exists a
possible assignment for all VUs with sufficient PUs (\Cref{algo:check}), and provide feedback on the limiting resources, otherwise.
Finally, \name assigns each VU to a PU type with a backtracking search on the pruned bipartite
graph.

This approach can be easily extended to handle new heterogeneous tiles in the architecture by registering
the tile with existing or new types of resources with aggregation and partitioning rules.
The rest of this section goes over two types of partitioning transformations--compute
partitioning in \Cref{sec:compsplit} and memory partitioning in \Cref{sec:memsplit}.
%We have another partitioner encoding valid rule to decompose a BlackBox IP block available on the RDA.

\subsection{Compute Partitioning} 
\label{sec:compsplit}

The {\em compute-partitioning} phase addresses VUs using more compute resources than any PU can provide. 
If a VU contains multiple contexts, \name{} first moves the contexts into separate VUs.
If a single context exceeds the resource limit, \name breaks down the dataflow graph in the context into multiple contexts and puts them in separate VUs.
During partitioning, \name maps each subgraph of the large dataflow graph into a new context, mirrors the control states of the original context, and streams live variables in between.
We can formulate the problem of how to partition in the dataflow graph as an optimization problem, shown in
\Cref{tab:partprob}.
The partitioner ``fixes'' the VU \emph{v} based on a single PU specification, albeit there are many potential PUs 
the decomposed VU can be mapped to.
Currently, we use a heuristic to select a PU type from \emph{dom(v)} right before the compute pruning 
as a guiding constraint for partitioning.

\begin{table*}
  \centering
\begin{tabular}{lp{12cm}}
  \toprule
  \textbf{Problem} & Partition the dataflow graph into subgraphs such that all subgraphs satisfy the constraints of a
  hardware unit. \\[0.9cm]
  \textbf{Objective }& Minimize the number of partitions and connectivity across partitions. \\[0.5cm]
  \textbf{Constraints} & 
  \begin{minipage}{12cm}
  \begin{outline}
  \0 Each partition must not exceeds the limit on the number of \vspace{-0.2cm}
    \1 live in/out variables (I/O ports) \vspace{-0.2cm}
    \1 operations (pipeline stages), \vspace{-0.2cm}
    \1 and live variables across operations (pipeline registers), etc.\vspace{-0.2cm}
  \0 No \emph{new} cycles can form across partitions other than the cycles in the original
  dataflow graph.
  \end{outline}
  \end{minipage}
  \\
 \bottomrule
\end{tabular}
\caption[Formulation of the compute partitioning problem]{
Formulation of the compute partitioning problem
}
\label{tab:partprob}
\end{table*}

Because the global network is specialized to handle efficient broadcasts, 
the in/out-degree of a partition counts the number of unique live-in/out variables, as supposed to
the number of edges across partitions.
In addition, the partitioned subgraphs cannot form {\em new} cycles; contexts waits for all
input dependencies and therefore cycles across contexts cause deadlock. 
Nonetheless, the original graph might contain cycles representing loop-carried dependencies, such as
accumulation. For these cycles, \name initializes the back edge of the cycle with dummy data to
enable execution.
\Cref{fig:parteg} shows examples of valid and invalid partitioning solutions.
\Cref{fig:partcycleeg} shows another partitioning example of a dataflow graph with cycles.

\begin{figure}
  \centering
  \includegraphics[width=1\columnwidth]{figs/parteg.pdf}
  \caption[Compute partitioning examples]{
    Compute partitioning examples. Solution 1 and 2 are both valid. Solution 2 is
    better because it has less number of broadcast edges across partitions (3 as supposed to 4 in Solution 1). 
    Solution 3 is an illegal partitioning due to the cycle between partition 1 and 2.
  }
  \label{fig:parteg}

  \includegraphics[width=0.8\columnwidth]{figs/partcycleeg.pdf}
  \caption[Compute partitioning examples with cycle]{
    Compute partitioning examples with cycle in the dataflow graph.
    Solution 1 is valid because there is no 
    cycle between partitions after removing the back edge in
    the original graph.
    Solution 2 is invalid because there is still cycle between partition 1 and 2 after
    removing the back edge.
  }
  \label{fig:partcycleeg}
\end{figure}

\paragraph{Community Detection}
The formulation of compute partitioning is similar to the community detection
problem\cite{community} that has a similar objective. 
The major difference is that the latter often takes the number of output partitions as an
input to the algorithm, whereas our problem partitions until all subgraphs satisfy all constraints.
Moreover, community detection algorithms do not enforce the cycle constraints. 
Finally, the edge connectivity in community detection counts the number of edges across partitions, 
as supposed to broadcast edges as in our problem.

\paragraph{Retiming}
Imbalanced data paths across partitions can cause pipeline stalls at runtime if the long-live path is not
sufficiently buffered.
To ensure full-throughput pipelining, \name needs to insert retiming buffers along the imbalanced data path across
partitions.
Retiming introduces new VUs in addition to the partitioned VUs, which attributes to the cost in
\Cref{tab:partprob}'s objective.

In the following sections, we present two algorithms to resolve the problem described in
\Cref{tab:partprob}:
a fast traversal-based algorithm providing a decent solution, and a slow convex
optimization-based algorithm providing an optimal solution.

\subsubsection{Traversal-based Solution}
To address the cycle constraint, the traversal-based algorithm performs a topological sort of the dataflow graph.
The topological sort ignores the back edges of the cycles during traversal. 
Starting from one end of the sorted list, the algorithm iteratively adds nodes into a partition
until it fits no more nodes. The algorithm then repeats the process with a new partition.
This approach guarantees that no cycle is introduced with $O(V+E)$ complexity, 
where $V$ and $E$ are the numbers of vertices and edges in the dataflow graph.

The partitioning result is a function of the traversal order.
We experienced with depth-first search (DFS) and breadth-first search (BFS) with forward and
backward dataflow traversal orders.
For DFS, we re-sort the remaining list each time starting with a new partition.

\subsubsection{Solver-based Solution}
The convex optimization solution models the problem as a node-to-partition assignment problem.
\Cref{tab:solver-eqns} gives our formulation and \Cref{tab:solver-variables} explains the notations
used in \Cref{tab:solver-eqns}.

\newcommand{\lcell}[1]{\makecell[l]{#1}}
\newcommand{\ccell}[1]{\makecell[c]{#1}}
\begin{table*}
  \centering

  %\scriptsize
	%\begin{tabular}{c | c | c | c}
		%\textbf{Name} & \textbf{Type} & \textbf{Description} & \textbf{Definition / Default}\\\hline
		%$\mathcal{N}$ & Constant & Enumeration of nodes to partition, numbered $\{n_i\}_i$ & - \\
		%N & Constant, $\nnint$ & Number of operations to partition & $N = |\mathcal{N}|$\\
		%P & Constant, $\nnint$ & Number of partitions to consider & $N$, or from heuristic \\
		%$\mathcal{E}$ & Constant, $\{n_i \to n_j\}$& Directed edges representing dependence & - \\
		%B & Variable, $\{0, 1\}^{N \times P}$ & Boolean Partitioning Matrix& - \\
		%%p & Variable, $\nnint^N$ & Vector of mappings from node to assigned partition& $p = B \begin{bmatrix} 0 & 1 & \cdots & P-1\end{bmatrix}^T$\\
		%$\projb{\cdot}$ & $\nnint \to \mathbb{B}$ & Function to convert a positive integer into a boolean& Supplemental Materials\\
		%$\andf(\cdot, \cdot)$ & $\{0, 1\} \times \{0, 1\} \to \{0, 1\}$ & Boolean and of binary variables & Supplemental Materials \\ 
		%$d_p$ & Variable, $\nnint^P$ & Vector of partition delays & - \\
		%$d_n$ & Variable, $\nnint^N$ & Vector of node delays & - \\
		%$\dest(n)$ & $\mathcal{N} \to \mathcal{P}(\mathcal{N})$& The set of nodes which depend on $n$& $\{n' | n' \in \mathcal{N}\ s.t.\ (n \to n') \in \mathcal{E}\}$\\
		%$c_o$ & Constant, $\nnint$ & Maximum output arity of a partition & HW Spec \\
		%$c_i$ & Constant, $\nnint$ & Maximum input arity of a partition & HW Spec \\
		%$b_d$ & Constant, $\nnint$ & Maximum input buffer depth & HW Spec \\
		%$K$ & Constant, $\mathbb{R}_+$ & Very Large Constant, used for constraint activation & $P \times N$ \\
		%$\alpha_d$ & Hyperparameter, $\mathbb{R}_+$ & Retime merging probability multiplier& $\frac{1}{\max\{c_o, c_i\}}$ \\
	%\end{tabular}

  \begin{tabularx}{\textwidth}{cccc}
    \toprule
    \textbf{Name} & \textbf{Type} & \textbf{Description} & \textbf{Definition / Default}\\
    \midrule
    $\mathcal{N}$ & Constant & \ccell{Enumeration of nodes to\\ partition, numbered $\{n_i\}_i$} & - \\[0.5cm]
    N & Constant, $\nnint$ & \ccell{Number of operations\\ to partition} & $N = |\mathcal{N}|$\\[0.5cm]
    P & Constant, $\nnint$ & \ccell{Number of partitions\\ to consider} & $N$, or from heuristic \\[0.5cm]
    $\mathcal{E}$ & Constant, $\{n_i \to n_j\}$& \ccell{Directed edges representing\\ dependence} & - \\[0.5cm]
    B & Variable, $\{0, 1\}^{N \times P}$ & \ccell{Boolean Partitioning\\ Matrix}& - \\[0.5cm]
    %p & Variable, $\nnint^N$ & Vector of mappings from node to assigned partition& $p = B \begin{bmatrix} 0 & 1 & \cdots & P-1\end{bmatrix}^T$\\
    $\projb{\cdot}$ & $\nnint \to \mathbb{B}$ & \ccell{Function to convert a\\ positive integer into\\ a boolean}& Supplemental Materials\\[0.5cm]
    $\andf(\cdot, \cdot)$ & \ccell{$\{0, 1\} \times \{0, 1\}$\\$ \to \{0, 1\}$} & \ccell{Boolean and of\\ binary variables} & Supplemental Materials \\ [0.5cm]
    $d_p$ & Variable, $\nnint^P$ & Vector of partition delays & - \\[0.5cm]
    $d_n$ & Variable, $\nnint^N$ & Vector of node delays & - \\[0.5cm]
    $\dest(n)$ & $\mathcal{N} \to \mathcal{P}(\mathcal{N})$& \ccell{The set of nodes\\ which depend on $n$} & $\{n' | n' \in \mathcal{N}\ s.t.\ (n \to n') \in \mathcal{E}\}$\\[0.5cm]
    $c_o$ & Constant, $\nnint$ & \ccell{Maximum output arity\\ of a partition} & HW Spec \\[0.5cm]
    $c_i$ & Constant, $\nnint$ & \ccell{Maximum input arity\\ of a partition} & HW Spec \\[0.5cm]
    $b_d$ & Constant, $\nnint$ & \ccell{Maximum input\\ buffer depth} & HW Spec \\[0.5cm]
    $K$ & Constant, $\mathbb{R}_+$ & \ccell{Very Large Constant, used\\ for constraint activation} & $P \times N$ \\[0.5cm]
    $\alpha_d$ & Hyperparameter, $\mathbb{R}_+$ & \ccell{Retime merging\\ probability multiplier}& $\frac{1}{\max\{c_o, c_i\}}$ \\[0.5cm]
    \bottomrule
  \end{tabularx}
  \caption{Names and definitions used in the solver-based partitioning.}
  \label{tab:solver-variables}
\end{table*}

\begin{table*}
  \centering
  \begin{tabularx}{\textwidth}{cll}
    \toprule
		\textbf{Type} & \textbf{Description} & \textbf{Expression}\\\midrule
    \multirow{3}{*}{\lcell{Cost\\Function}} & Allocated Partitions & $\Sigma_i \projb{\Sigma_j B_{i, j}}$\\

    & \lcell{Additional Retiming\\Partitions}
    & $\alpha_d \Sigma_{n_i \to n_j \in \mathcal{E}} \projb{\max\{d_n(j) - d_n(i) - b_d, 0\}}$\\[0.3cm]
		\hline

    \multirow{12}{*}{\lcell{\\\\Constraint}} & Partition Assignment & $ \forall n_i \in \mathcal{N}:\ \Sigma_j B_{i, j} = 1$\\[0.1cm]

    &\lcell{Dependency\\Constraint} & $\forall n_i \to n_j \in \mathcal{E}:\ d_n(i) + 1[p_i \ne p_j] \le d_n(j)$\\[0.1cm]

    &\lcell{Output Arity\\ Constraint} 
    &\lcell{
      $\forall p \in [0, P):$ \\
      $\Sigma_{n_s \in \mathcal{N}} \andf(B_{s, p}, \projb{\max\{(\Sigma_{n_d \in \dest(n_s)} B_{d, p}) -$ \\
      $K \times B_{s, p}, 0\}}) \le c_o$
    }\\[0.7cm]

    &\lcell{Input Arity Constraint\\ (vectorized)} & $\Sigma_{n_i \in \mathcal{N}} \max\{\projb{\Sigma_{n_j \in \dest(n_i)} B_{j, :}} - B_{i, :}, 0\} \le c_i \times \vec{1}$\\

		&Delay Consistency& 
    \lcell{
    $\forall n_i \in \mathcal{N}:\ d_n(i) \le \min_j (d_p(j) + K - B_{i, j} \times K)$ \\
		$\forall n_i \in \mathcal{N}:\ d_n(i) \ge \max_j (d_p(j) + B_{i, j} \times K - K)$
    }\\[0.5cm]

		&Constant Validity& 
    \lcell{
      $\forall n_i \in \mathcal{N}:\ d_n(i) \le K$\\
		  $\forall i \in [0, P):\ d_p(i) \le K$
    } \\
    \bottomrule
	\end{tabularx}
  \caption{Solver formulation for partitioning*.}
	\label{tab:solver-eqns}
\end{table*}

\begin{table*}
	\begin{tabular}{c | c | c | c}
		\textbf{Name} & \textbf{Type} & \textbf{Description} & \textbf{Definition / Default}\\\hline
		$\mathcal{C}_r$& $[\mathcal{N} \to \mathbb{R}_+,\mathbb{R}_+, [\mathbb{R}_+] \to \mathbb{R}_+]$ & List of per-node values, limits, and reduction& Supplemental Materials\\&& functions for reducible constraints& \\
		F & $\{0, 1\}^{N \times P}$ & Feasibility matrix, whether a partition can support a node& HW Spec \\ 
	\end{tabular}
	\caption{\Cref{tab:solver-variables} extension for solver-based merging, which is a generalization of the partitioning problem.}
	\label{tab:merging-variables}

	\begin{tabular}{c | c | c}
		\textbf{Type} & \textbf{Description} & \textbf{Expression}\\\hline
		Constraint & Feasibility Constraint & $ \forall i, j \in [0, N) \times [0, P):\ B_{i, j} \le F_{i, j}$\\
		& Reducible Constraints & $\forall j \in [0, P).\ \forall (c(\cdot), c_v, r(\cdot)) \in \mathcal{C}:\ r([c(n_i) \times B_{i, j}]_{n_i \in \mathcal{N}}) \le c_v$\\
	\end{tabular}
  \caption{\Cref{tab:solver-eqns} extension for solver-based merging*. The Retiming Partition objective is not used for merging.}
	\label{tab:merge-eqns}
  \scriptsize
  \raggedright
  \vspace{-0.3cm}
  *Expressions are presented using the Disciplined Convex Programming ruleset \cite{DCP, DCP-online}. Explanations for selected expressions can be found in the supplemental material.
\end{table*}



At a high-level, we use a boolean matrix $B$ to keep track of the assignment. 
$B$ has dimension equals to the number of nodes in the dataflow graph by the maximum number of partitions, where$B[i,j]=1$ 
indicates node $i$ is assigned to partition $j$.
In \Cref{tab:solver-eqns}, \emph{partition assignment} restricts each node to have a single partition assignment.
The \emph{input and output arity constraints} show the formulations that limit the number of input
and output for a subgraph.
These are the two most challenging constraints as we need to identify broadcast edges across partitions.
To address the cycle constraint, we introduce a delay vector $d_n$ with a size equivalent to the number of nodes. 
The delay vector encodes a time schedule to execute each node, whose values are selected by the solver.
The \emph{dependency constraint} enforces a node can be scheduled no earlier than its input dependencies and
no later than its output dependents.
Since the operations within a partition have to be triggered atomically, there is another delay
vector $d_p$ for partitions. The \emph{delay consistency} enforces the schedule of a node equals to the
schedule of its assigned partition.
Finally, \emph{constant validity} limits the range of values the delay vectors can be chosen from.
In addition to enforcing the cycle constraint, 
these delay variables are also used to calculate where retiming is required and project the amount of introduced retiming VUs. 
Finally, we use the traversal-based solution to warm start the assignment matrix
$B$ and the delay vectors to reduce the solver runtime.

\subsubsection{Comparison}

\begin{figure*}
\centering
\hfill
\begin{subfigure}[b]{0.35\textwidth}
\includegraphics[width=1\textwidth]{figs/algo2.pdf}
\caption{Resource Comparison}
\end{subfigure}
\hfill
\begin{subfigure}[b]{0.64\textwidth}
\centering
\begin{tabular}{lccccc}
  \toprule
  Apps &bfs-bwd & bfs-fwd & dfs-bwd & dfs-fwd & gurobi \\ \midrule 
bs & 3s & 3s & 2s & 2s & 2h4m \\ 
mlp & <1s & <1s & <1s & <1s & 4s \\ 
rf & 1m37s & 1m7s & 29s & 27s & - \\ 

 \bottomrule
\end{tabular}
\caption{
  Compile time for spltting
}
\vspace{0.1cm}
\begin{tabular}{lccccc}
  \toprule
  Apps &bfs-bwd & bfs-fwd & dfs-bwd & dfs-fwd & gurobi \\ \midrule 
bs & 1s & 2s & 5s & 3s & 1m4s \\ 
mlp & 5s & 5s & 6s & 11s & 31m16s \\ 
rf & 11s & 10s & 2m7s & 1m0s & 14h27m \\ 

 \bottomrule
\end{tabular}
\caption{
  Compile time for merging
}
\vspace{0.65cm}
\end{subfigure}
\hfill
\caption[Partitioning and merging algorithm comparisons]{
  Partitioning and merging algorithm comparisons. (a) shows the normalized resource usage between
  different algorithms (the lower the better). (b) and (c) shows the compile time of each algorithm.
  Benchmarks include BlackScholes (bs), multi-layer perceptron (mlp), and random forests (rf).
}
\label{fig:split}
\end{figure*}

\Cref{fig:split} shows the comparison between the traversal-based and the solver-based solutions for both
compute partitioning and global merging.
Global merging is a global optimization merging small VUs into a large VU that can still fit in
a PU. 
The merging algorithm is very similar to the compute partitioning algorithm, where nodes in the
dataflow graph corresponds to the VUs in a VUDFG. 
The traversal and solver-based algorithms for partitioning can be extended to handle the merging
problem.
\Cref{sec:opt} will discuss merging problem in more detail.

We use a commercial solver, Gurobi~\cite{gurobi}, for the solver-based algorithm. 
The evaluation is performed on an Intel Xeon E7-8890 CPU at 2.5GHz with 1TB DDR4 RAM. 
Gurobi is parallelized across ten threads for each application.
To speed up convergence, we configure Gurobi with a 15\% optimality gap, i.e., 
the solver is allowed to early stop after the current solution is less than 15\% worse than the optimum
solution. 
%To speed up convergence, we use a 15\% optimality gap that stops the solver at a reasonable solution.
The solving time increases dramatically as getting close to 100\% optimum.

\Cref{fig:split} (a) shows the normalized resources in the number of VUs after partitioning and merging.
We can see that Gurobi provides almost the best solution for all applications when it can derive
an answer in a reasonable amount of time. The missing solver bar in random forest (\emph{rf}) partitioning
is due to timeout after a few days.
The traversal-based algorithms can sometimes match or even outperform the solver slightly.
However, because the partitioning result is a function of the traversal order, 
each traversal order has adversarial cases, where they can be up to 1.7x worse in resource than the best possible solution.
We found the forward (\emph{fwd}) traversal order schedules nodes as earlier as possible, reducing the number of
external live variables; the backward traversal minimizes the number of internal live variables
across partitions.
The depth-first-search (DFS) traversal order minimizes the number of live variables between partitions, 
albeit producing more imbalanced paths between partitions. 
On the other hand, breath-first-search (BFS) produces more balanced partitioning with more live variables and partitions.

There are two common graph patterns in the applications that require partitioning. 
The first is a dataflow graph from a large basic block, which contains a small set of external
live-in and -out variables and many intermediate temporary variables.
Such graphs typically end up with long-live variables across partitions that require retiming.
%\todo{show example and discuss the other pattern.}
The second is a balanced tree structure as the result of partitioning
a logical memory across PUs discussed later in \Cref{sec:memsplit}.
The first structure favors the DFS traversal order, minimizing the number of partitions. The second
structure favors the BFS traversal order, creating balanced partitions without additional
retiming VUs.

\Cref{fig:split} (b) and (c) shows the compile time for these algorithms. The single-threaded
the traversal-based algorithm runs in minutes, which is significantly faster than the parallelized solver that takes hours to days.
In general, the solver runtime becomes quickly unbounded with a large amount of VUs.
%\todo{show solver time with an increasing number of VBs}.

In summary, the solver solution provides a guaranteed close-to-optimum solution at an expensive
compile time. Nonetheless, the solver-solution treats the retiming and partitioning as a joint optimization,
whereas the traversal-based solution solves these two problems in two separate passes,
generating less optimal solutions.
Moreover, the solver-based solution tends to produce a better result for PUs with a tight I/O
bound (small number of I/Os and large number of stages).
The traversal-based solutions, on the other hand, can produce a decent solution in a short amount of
time.
However, the solution is a function of the traversal order; hence, the quality of the partitioning
is highly sensitive to the graph structures.
In practice, we can combine the two approaches and invoke the expensive solver only when the
traversal-based solution is insufficient. The quality of the traversal-based solution can be easily
estimated with the resource utilization of a partition.

\subsection{Memory Partitioning} \label{sec:memsplit}
The memory pruner addresses virtual on-chip memory exceeding the capacity and bandwidth limit of a
single PMU.
As we parallelize the computation, the on-chip memory must provide higher address bandwidth to sustain the
compute throughput.
On a processor-based architecture, this is often achieved with a separate first-level
cache for each processor core, as shown in \Cref{fig:memmodel} (a).
The cache implements a hardware coherence protocol that synchronizes the different copies of data
behind the scenes,
providing the abstraction of a shared memory.

\begin{figure}
  \begin{subfigure}[b]{0.35\textwidth}
  \centering
  \includegraphics[width=1\columnwidth]{figs/cpumemmodel.pdf}
  \caption{Processor Architecture}
  \end{subfigure}
  \hfill
  \begin{subfigure}[b]{0.45\textwidth}
  \centering
  \includegraphics[width=1\columnwidth]{figs/spatialmemmodel.pdf}
  \caption{Reconfigurable Spatial Architecture}
  \end{subfigure}
  \caption[Memory model of different architectures]{Memory model of different architectures}
  \label{fig:memmodel}
\end{figure}

The coherence protocol is both expensive in hardware complexity and hard to scale in bandwidth for
streaming pipelined execution.
Instead of making redundant copies of the data, we can partition the data across different memory
partitions to get additional address port on a reconfigurable accelerator, as shown in
\Cref{fig:memmodel} (b).
Each parallel worker broadcasts the requests to all memory partitions.
If the data accessed by two parallel workers live in the same bank, the bandwidth will be halved.
To avoid bank conflicts, an important static analysis often used on reconfigurable accelerators is called static partitioning, or static banking~\cite{poly_cong}.
For most address patterns, the compiler can derive a partitioning scheme such that every partition is accessed by a
single worker at any time, which guarantees bandwidth at runtime.
The output of the analysis is an expression for bank address (BA) and bank offset (BO), both are
functions of the requested logical address.
BA determines which partition each request is going to, and BO is the offset within the
partition.

\begin{figure}
  \centering
  \includegraphics[width=0.6\columnwidth]{figs/memsplit2.pdf}
  \caption[Memory partitioning]{Example of memory partitioning across physical units. In this example, the program parallelizes the outer loop by two and vectorizes the inner loop by 4, which generates two vectorized access lanes. We need eight scratchpad banks to sustain the read bandwidth. 
  \name groups the 8 banks in two virtual memory partitions, $P_A$ (bank 0-3) and $P_B$ (bank 4-7). The address generation contexts $A_0$ and $A_1$ contains the computation for logical address and address remapping, which output $BA$s and $BO$s. \name allocates one merge context per memory partition
  ($M_A$ and $M_B$),
  merging requests from all access lanes. Inside the merge context, the shuffle operator converts the BO from access-aligned to bank-aligned. $A_0$, for example, requests offsets 
  A, B, C, and D ($BO_0$) from banks 1,4,5,and 3 ($BA_0$). The shuffle operator picks the requests belonging to partition A and outputs a BO aligned with bank 0-3. If the bank $i$ in the partition has no request, the $i$th element in the output is marked as invalid.
  The merge context contains one shuffle operator per request lane, and uses a tree of $OR$
  operators to combine all bank-aligned BOs into the final $BO_A$.
  On the receiver side, the memory broadcasts its response to all receiver lanes.
  The receiver context uses an inverse shuffle operator to convert the response from bank-aligned back to access-aligned, using the $BA$ forwarded from the address context. The response is then merged with another OR tree. 
  }
  \label{fig:memsplit}
\end{figure}

\Cref{fig:memsplit} shows how static banking is achieved on the Plasticine architecture.
Banking analysis from Spatial specifies the number of banks required to sustain bandwidth for the
current parallelization factors, and expressions for BAs and BOs. 
\name groups the banks into virtual memories, with group size limited by the number of banks in a PMU.
For each access lane, \name allocates a context to compute the logical and remapped address, which
outputs a vectorized BA and BO for each access lane. BAs and BOs are broadcasted to all memory partitions.
For each memory partition, \name allocates a merge context, merging BOs from all requesting lanes.
The merge context uses a shuffle operator to transform the vectorized BO from bank order specified by BA
(access-aligned) to bank order assigned to the current partition (bank-aligned).
Because the static banking analysis guarantees no bank conflicts for all banks,
\name can use a OR tree to merge the bank-aligned BOs into the final BO that gets send to
banks in the partition. On the receiver side, \name uses an inverse shuffle to convert responses from
partitions from bank-aligned back to access-aligned.
\name uses another OR tree to merge the access-aligned responses, which produces the final data
vector requested by the access lane.
With large parallelization, both the request OR tree and the respond OR tree can be
partitioned across VUs if running out of stages in a VU, scaling in the network bandwidth of the
crossbar connection by burning more VUs.

\Cref{sec:banking_arch} discusses the architectural changes to Plasticine to support this general
banking scheme.

\subsection{Register Allocation} \label{sec:regalloc}

%\subsubsection{Blackbox IP Pruning} \label{sec:bbsplit}
%\yz{Cut this if out of the space}
%This step illustrates an example of integrating a customized partitioner for composable IP available on
%the architecture.
%The cost metrics and partition rule are specific to each IP.
%The example IP is a merge buffer, which can merge two sorted vector streams into a single stream with
%one vector per cycle throughput.
%The merge buffer pruner uses a tree of 2-way merge buffers across PUs to compose a multi-way merge buffer declared in the program.
