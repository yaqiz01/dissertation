\prefacesection{Acknowledgements}

As a start, I would like to thank the most impactful person in my graduate adventure---my advisor
Kunle Olukotun.
Retrospectively, Kunle provided many constructive suggestions that set forth a good foundation for my research.
While Kunle knows what is right, he allows us to explore and make mistakes.
A frequent question from Kunle during our one-on-ones is ``So, what's the plan?''.
Beyond the technical expertise, one of the most valuable skills I have acquired from graduate school is the ability to construct an executable plan for a new problem, realize the defects in the initial solution, and persistently refine the implementation, which results in innovations.

Many other Stanford professors have benefited my research from various perspectives.
I would like to thank Subhasish Mitra and Christos Kozyrakis for their past support for the
Plasticine project.
Interactions with Matei Zaharia and Chris R\'e during DAWN retreats also enlarged my scope on 
software advancement in machine learning research and frameworks.
I would also like to thank Juan Rivas to serve as my defense committee chair and evaluate the
contribution of the work for a broader audience.

I have encountered a group of exceptional colleagues at Stanford and
benefit tremendously in my professional skills\footnote{and other random knowledge} during interactions with my peers.
David Koeplinger answered many of my Scala questions and I have learned a great deal about compiler
infrastructures by reading his code.
Raghu Prabhakar had always been extremely helpful and instrumental in research discussions. 
He often put himself in others' shoes that he always was and continue to be the most popular person on the planet back at school and now at work.
Alex Rucker made great contributions to Plasticine's network design and introduced a high-level simulator that relieved me from RTL simulation, which undoubtedly expedited my graduation.
Matt Feldman enhanced the Spatial compiler and made it robust, which supported the research of mine and many
others, both inside and outside our group.
Matt also introduced his dog, Daisy, to befriend with everyone and 
provide de-stress after a hard-working Friday afternoon.
Tian Zhao provided generous support on experiment setup and baseline collections during many of my paper deadlines.
I had a great collaboration experience with Matt Vilim, who also introduced me to ergonomic keyboards.
Nathan Zhang contributed considerably to solver-related works and was always the joyful person in the group that positively affected everyone.
Muhammad Shahbaz was particularly helpful on paper editing and helped me identify the impact and contribution of my work.
I had a fun cross-domain collaboration with Rekha Singhal and Jeff Ullman to design new algorithms for our
hardware.

I would also like to thank everyone in Kunle's group and all of the DAWN members, including but not limited to, 
Tushar Swamy, Olivia Hsu, and Luigi Nardi. 
All of you are an vital part of my graduate school experience.
I appreciate the hard work our Admins---Dan Moreau and Andrea Brand-Sanchez---had put forth that
offered us a smooth experience for graduate school.

I would like to share my gratitude towards my undergraduate research advisor, Daniel J. Sorin, who taught the
majority of my architecture classes and inspired my research interests in this area.
I would not have experienced such a fruitful journey and acquainted so many outstanding people
without his mentorship and encouragement.

Last but not least, I would like to thank my friends and family for their continuous support during my
graduate school. 
Friends have always been my first source for help ever since I left home and studied abroad.
They have enriched my life beyond research, making the past five years both enjoyable and
unforgettable.
My parents' passion for research has always inspired me to pursue a Ph.D., and their encouragement and
acknowledgments equipped me with tremendous strength and confidence to complete this adventure.
Finally, I would like to thank my boyfriend, Charles, for always by my side during the happiest and the most stressful moments. 
He was especially understanding and supportive that he would
share my household burden during deadlines and accommodate my irregular working schedules.
My graduate school journey would be much more challenging and less rewarded without the support from all these people.
