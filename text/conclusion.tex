\chapter{Conclusions (WIP)} \label{sec:conclusion}

Reconfigurable dataflow accelerators (RDAs) are a promising class of spatial accelerators that deliver higher performance-to-resource efficiency while capturing a large application space.
However, to sustain these benefits of RDAs, scalability must be taken into account in both software
stack and hardware design.

On the software side, we address this challenge with a distributed asynchronous control scheme and resource
virtualization.
We develop a compiler, \name{}, that constructs a dataflow graph from an imperative program with nested
control flow. We use dataflow tokens to enforce sequential consistency across distributed memory
accesses, incurring minimum communication overhead.
Furthermore, \name efficiently compose distributed resource to provide an abstraction of larger
logical resources to the programmers.
Our optimizations effectively eliminate pipelining stall and reduce resource fragmentation with
heterogeneous resources.
Lastly, our evaluation shows that \name{} achieves of 2x averaged speedup and an area-normalized 4x
speedup over a Tesla V100 GPU.

On the hardware side, we show that the best network design depends on both applications and the underlying accelerator architecture.
Network performance correlates strongly with bandwidth for streaming accelerators, and scaling raw bandwidth is more area- and energy-efficient with a static network.
We show that the application mapping can be optimized to move less data by using a dynamic network as a fallback from a high-bandwidth static network.
This static-dynamic hybrid network provides a 1.8x energy-efficiency and
2.8x performance advantage over the purely static and purely dynamic networks, respectively.


