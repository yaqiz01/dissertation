\usepackage[T1]{fontenc}

%% Font of document
\usepackage{tgpagella}
%\usepackage{tgbonum} %% no good
%\usepackage{bookman} %% no good
%\usepackage{charter} 
%\usepackage{courier} 
%\usepackage{mathptmx}
\renewcommand{\baselinestretch}{1.5}

\usepackage{graphicx}

\usepackage{listings}           % Code listing
\usepackage{courier}
\definecolor{vbgray}{gray}{0.9}
\definecolor{darkgreen}{RGB} {0, 100, 0}
\definecolor{darkred}{RGB} {255, 0, 0}
\definecolor{orange}{RGB} {255, 128, 0}
\def\tick{{\color{darkgreen} \textbf{\tikz\fill[scale=0.5](0,.35) -- (.25,0) -- (1,.7) -- (.25,.15) -- cycle;}}}
\def\ytick{{\color{orange} \textbf{\tikz\fill[scale=0.5](0,.35) -- (.25,0) -- (1,.7) -- (.25,.15) -- cycle;}}}
\def\x{{\color{darkred} {$\bm{\times}$}}}

\definecolor{red}{RGB}{255,0,0}
\definecolor{vbgray}{gray}{0.9}
\definecolor{darkgreen}{RGB} {0, 100, 0}
\definecolor{darkred}{RGB} {255, 0, 0}
\definecolor{blue}{RGB} {0, 135, 255}
\definecolor{yellow}{RGB} {224, 173, 0}
\definecolor{codegreen}{RGB}{52,123,0}
\definecolor{codegray}{rgb}{0.6,0.5,0.5}
\definecolor{codepurple}{rgb}{0.58,0,0.82}
\definecolor{backcolour}{rgb}{0.95,0.95,0.95}
\definecolor{lightback}{rgb}{0.95,0.95,0.95}
\def\tick{{\color{darkgreen} \textbf{\tikz\fill[scale=0.5](0,.35) -- (.25,0) -- (1,.7) -- (.25,.15) -- cycle;}}}
\def\ytick{{\color{orange} \textbf{\tikz\fill[scale=0.5](0,.35) -- (.25,0) -- (1,.7) -- (.25,.15) -- cycle;}}}
\def\x{{\color{darkred} {$\bm{\times}$}}}

\lstdefinelanguage{Spatial}{
  basicstyle=\fontsize{7}{7}\selectfont\ttfamily,frame=tlbr,framesep=4pt,framerule=0pt,
  tabsize=2,
  basewidth={0.55em, 0.4em},%
  numbers=left,
  showspaces=false,
  keywordstyle=\bfseries,
  breaklines=true,
  columns=fixed,
  %xleftmargin=0.25in,
  firstnumber=auto,
  showstringspaces=false,
  escapechar=@,
  escapeinside={(*@}{@*)},
  morestring=[b]",
  morestring=[b]',
  morecomment=[l]{//},
  morecomment=[s]{/*}{*/},
  backgroundcolor=\color{backcolour},
  commentstyle=\color{codegreen}, %\bfseries,
  numberstyle=\tiny\color{codegray},
  stringstyle=\color{codepurple},
  keywordstyle=[2]\color{blue},
  keywords=[2]{val, def, type},
  keywordstyle=[3]\color{yellow}\bfseries,
  keywords=[3]{Float, Float8, Int, String, T, Void, Bit, Half, FltPt},
  keywordstyle=[4]\color{orange}\bfseries,
  keywords=[4]{Matrix, Array},
  keywordstyle=[5]\color{blue}\bfseries,
  keywords=[5]{StreamIn, StreamOut, DRAM, ArgIn, ArgOut, HostIO, RegFile, Reg, SRAM, SRAM1, SRAM2, FIFO, LIFO, LUT, LineBuffer},
  keywordstyle=[6]\bfseries,
  keywords=[6]{enq, deq, load, store, scatter, gather, :=, push, pop, peek},
  keywordstyle=[7]\color{magenta},
  keywords=[7]{until, par, by, value},
  keywordstyle=[8]\color{red}\bfseries,
  keywords=[8]{C0,C1,C2,C3,C4,C5,C6,C7,C8,C9,C10},
  keywordstyle=\color{magenta}\bfseries,
  morekeywords={Foreach,Reduce,MemReduce,MemFold,Fold,Accel,Stream,FSM,Sequential,if,else,Parallel,Pipe, DummyPipe}
}


\usepackage{outlines}
\usepackage{booktabs}
\usepackage{multirow}
%\usepackage[numbers,sort&compress]{natbib}
\usepackage{amsmath}
\usepackage{amsfonts}
\usepackage{siunitx}

\usepackage{makecell}

%% Code formatting
\usepackage{minted}

%% To enable sub figures
\usepackage{caption}
\usepackage{subcaption}

\usepackage{algorithm2e}
\SetStartEndCondition{ }{}{}%
\SetKwProg{Fn}{Function}{\string:}{}
%\SetKwProg{Fn}{}{}{}
\SetKw{KwTo}{in}
\SetKw{KwNew}{new}
\SetKwIF{If}{ElseIf}{Else}{if}{:}{elif}{else:}{}%
\SetKwFor{For}{for}{\string:}{}%
\SetKwFor{While}{while}{:}{}%
\AlgoDontDisplayBlockMarkers%
\SetAlgoNoEnd%
\SetAlgoNoLine%
\SetKwFunction{Range}{range}%%

\usepackage[hidelinks]{hyperref}
\usepackage[capitalise]{cleveref}
\usepackage{xspace} 
\usepackage{todonotes}

\setminted[python]{linenos, frame=lines, framesep=2mm, baselinestretch=0.8, fontsize=\footnotesize,escapeinside=||}
%\usemintedstyle{pastie}

\newcommand{\name}{SARA\xspace}
\newcommand{\rda}{RDA\xspace}
\newcommand{\rdas}{RDAs\xspace}
\newcommand{\gist}[1]{{\color{blue} #1}}
\renewcommand{\todo}[1]{{\color{red} TODO: #1}}
\newcommand{\term}[1]{\textbf{#1}}
\newcommand{\eg}{{\em e.g.}}
\newcommand{\CL}[1]{\textbf{\textcolor{red}{#1}}}
\definecolor{brightpurple}{RGB}{163,33,247}
\newcommand{\CK}[1]{\textbf{\textcolor{brightpurple}{#1}}}

\usepackage{pifont}
\newcommand{\cmark}{\textcolor{green!80!black}{\ding{51}}}
\newcommand{\xmark}{\textcolor{red}{\ding{55}}}

%%
%% Math operators
\DeclareMathOperator*{\argmax}{argmax}
\DeclareMathOperator*{\elmax}{elmax} % Elementwise maximum
\DeclareMathOperator*{\dest}{dest} % Destinations for an operator (its consumers)
\DeclareMathOperator*{\andf}{and}
\DeclareMathOperator*{\maximum}{maximum}

\newcommand{\projb}[1]{\text{proj}_\mathbf{B}(#1)}
\newcommand{\nnint}{\mathbb{Z}_{\ge 0}}
\newcommand{\ind}[1]{\mathbb{1}[#1]}
