\prefacesection{Abstract}

With the slowdown of Moore’s Law, specialized hardware accelerators are gaining traction
for delivering 100-1000x performance improvement over general-purpose
processors in a variety of applications domains, such as cloud computing, biocomputing, 
artificial intelligence, etc.~\cite{fpgacloudsurvey,bioaccel,genomicaccel}.
As the performance scaling in multicores is coming to a limit~\cite{multicorescale}, a new
class of accelerators--reconfigurable dataflow architectures (RDAs)--offers  
promising high-throughput and energy-efficient acceleration that keeps up with the performance demand.
Instead of dynamically fetching instructions like in traditional processors, RDAs have flexible datapath 
that can be statically configured to spatially parallelize and pipeline the program across
distributed on-chip resources. 
The pipelined execution model and explicitly-managed scratchpad in RDAs eliminate
the performance, area, and energy overhead in dynamic scheduling and conventional memory hierarchy.

To adapt to the compute intensity in modern data-analytic workloads, particularly in the deep learning
domain, RDAs are increasing to a scale that was unprecedented before.
With an area footprint of $133\text{mm}^2$ at 28nm, 
Plasticine is a large-scale RDA supplying 12.3 TFLOPs of computing power~\cite{plasticine}.
Prior work has shown an up to 76x performance/watt benefit from Plasticine over a Stradix V FPGA 
due to an advantage in clock frequency and resource density.
The increase in scale introduces new challenges in network-on-chip design to maintain 
the throughput and energy efficiency of RDAs.
Furthermore, targeting and managing RDAs at this scale require new strategies in mapping, 
memory management, and flexible control to fully utilize their compute power. 

In this work, we focus on two aspects of the software-hardware co-design that impact the usability
and scalability of the Plasticine accelerator. 
Although RDAs are flexible to support a wide range of applications, 
the largest challenge that hinders the adoption of these accelerators is 
the required low-level knowledge in microarchitecture design and hardware constraints in
order to efficiently implement a new application.
To address this challenge, we introduce a compiler stack--\name--that raises the programming abstraction of
Plasticine to an imperative-style domain-specific language with nested control
flow for general spatial architectures.
Besides architecture-agnostic, this abstraction contains explicit loop constructs, enabling
cross-kernel optimizations that are often not exploited when programming RDAs.
\name efficiently translates imperative control constructs to a streaming
dataflow graph that scale performance with distributed on-chip resources.
By virutalizing resources, \name systematically handles the physical constraints, hiding
the low-level physical limitations from the programmers.
To address the scalability challenge with increasing chip sizes in RDAs, 
we present a comprehensive study on the network-on-chip design space for RDAs~\cite{network}.
We found that network performance highly correlates to network bandwidth, as supposed to latency,
for RDAs with streaming dataflow execution model.
Lastly, we show that a static-dynamic hybrid network design can sustain performance in a 
scalable fashion with high energy efficiency.
